\documentclass[11pt,a4paper]{article}
\pdfoutput=1
\usepackage{gensymb}
\usepackage[utf8]{inputenc}
\usepackage[T1]{fontenc}
\usepackage[swedish]{babel}
\usepackage{amsmath} 
\usepackage{lmodern}
\usepackage{units}
\usepackage{icomma}
\usepackage{tikz}
\usepackage{pgf-pie}
\usepackage{color}
\usepackage{graphicx,caption}
\usepackage{hyperref}
\usepackage{filecontents}
\usepackage{subcaption}
\usepackage{bbm}
\usepackage{todonotes}
\usepackage{pdfpages}
\usepackage{float}
\usepackage[utf8]{inputenc}
%\usepackage[top=1.4in, bottom=1.3in, left=1.5in, right=1.5in]{geometry}
\usepackage{pgfgantt}
\usepackage{float}
\usepackage{csquotes}
\usepackage{caption}
\usepackage{subcaption}

\definecolor{lila}{RGB}{128,0,128} % - Ida
\definecolor{Mahogany}{RGB}{192, 64, 0} % - Mattias
\definecolor{turkos}{RGB}{0, 163, 215} % - Björn
\definecolor{WildStrawberry}{RGB}{255,67,164} % - Axel
\definecolor{green}{RGB}{25,150,50} % - Påja

\begin{document}

\includepdf{Figures/Framsida.pdf}

\pagenumbering{gobble}

\newpage

\begin{centering}
\Huge
''Tell me and I forget 

Teach me and I remember

Involve me and I learn''

\end{centering}
\bigskip
\begin{centering}

\Large
- Benjamin Franklin

\end{centering}
\newpage

\renewcommand\abstractname{Sammandrag}\begin{abstract}
\noindent \textcolor{green}{Matematik är ett kärnämne i skolan. Samtidigt är en vanlig upplevelse för många elever att det är svårt, tråkigt och oanvändbart. Med detta projekt ville vi ändra på detta genom att införa nya typer av matematiska problem för gymnasieskolan. Dessa problem låter eleverna arbeta mer med problemlösning. Problemen har en realistisk verklighetsanknytning, ger underlag för diskussion, går att lösa på olika sätt, och ska framför allt ge känslan av att matematik är användbart. Eftersom programmering snart ska införas i skolmatematiken är även en del av problemen konstruerade för att förbereda inför det.}

\textcolor{green}{Som en del av vårt resultat gjorde vi bland annat en undersökning som 58 lärare deltog i. 91\% av lärarna angav att de arbetade för att inkluderade mer problemlösning samtidigt som 42\% tyckte att det var svårt att hitta bra problem. Efter att problemen testades på gymnasieskolor svarade en del av eleverna på en enkät om sin upplevelse av det problem de arbetade med. Av de som svarade angav 80\% att de lärde sig något nytt och 75\% uttryckte på något sätt att problemet var roligt eller meningsfullt.}

\textcolor{green}{Vår slutsats blev att lärare idag vill inkludera mer problemlösning i undervisningen, men att det inte är lätt att hitta bra problem, samtidigt som en del har tidsbrist. Trots att vi inte lyckades testa problemen i så många klasser som vi hade önskat, så var både vi och lärarna som testade dem nöjda med problemen.}
\end{abstract}

\newpage

\renewcommand\abstractname{Abstract}
\begin{abstract} 
\noindent \textcolor{WildStrawberry}{
    Mathematics is a core subject in the Swedish school system. Parallel to this, it is a common occurence that the subject is perceived as hard, boring and useless. With this project we want to faciliate the teachers by introducing new mathematical problems targeted for upper secondary school pupils. We conducted a survey for teachers; where 91\% asserted that they are working to include problem-solving in their teachings, while 42\% have issues finding good problems. The main purpose of this project has then been to produce problems designed to help the teachers in question. The problems are open, applicable to real-life, give ground for discussion and, perhaps most importantly, make mathematics feel relevant. Also, since programming is soon going to be part of mathematics, a share of our problems have been constructed to prepare students for this inclusion. Additionally to these problems, a suggested ''complete'' lesson plan is provided and presented on an easily accessible and user-friendly web page. Finally, from all our tested problems, the 26 student answers assert that: 80\% declare they learnt something new and that 75\% state they found the problem fun or meaningful in some way.
    }
    
    %This project aims to change this by introducing \textit{relevant} and easily accessible problem-solving tasks for upper secondary school pupils.  Thus the main purpose
    
    %These problems are designed to spark initiative and allow for creative solutions. Analogous to the real-world, the problems should encourage discussion whether they can be solved in multiple ways or if solutions can be refined to better model, but perhaps most of all: feel useful. Also, since programming is going to be introduced in school mathematics, some problems are made to test and prepare students for the inclusion of programming. }
    
%\textcolor{WildStrawberry}{
    %As a part of the result from the project, 58 teachers took part of our survey.     After testing our problems, some of the pupils answered a questionnaire regarding their experience when approaching the problems. From the 26 answers: 80\% declared that they learnt something new and 75\% stated that they found the problem fun or meaningful in some way.}
    
%\textcolor{WildStrawberry}{
    %Our conclusion is that teachers want to include problems-solving in their teachings, however finding well made problems isn't an easy task and that they lack the time to craft problems themselves. The reach we surmounted to wasn't huge, but the teachers who did try our problems seemed content.}
\end{abstract}

\newpage
 
\tableofcontents

\newpage
\pagenumbering{arabic}

\section*{Övrigt material som ska in någonstans i rapporten}
    \subsection{Bör läggas någonstans i diskussionen}

\textcolor{WildStrawberry}{
    Som grupp har vi alla något gemensamt. Vi har alla relativt nyligen genomgått en gymnasieutbildning och tyckes alla komma ihåg att det existerade problem med motivationen på en mängd elever i våra klasser. Internt inom vår grupp så blir det ett stickprov på 5 gymnasieklasser där alla genomgående känner samma sak. Detta stickprov är ju inte alls egentligen något att komma med, det är alldeles för litet för att kunna härleda någon ordentlig slutsats. Men om motivationen på elever i matematiken inte skulle vara ett vanligt problem i gymnasiet skulle vi troligtvis i alla fall haft en person internt som kunde hävda det. Denna observation är dels underlag för vidare undersökning. Gymnasietskolan i Sverige har nya betygssytem och kriterier som lägger en del vikt på problemlösning i matematiken \ref{sec:Forandringar}. Därav kommer syftet med problemen som utformas i detta arbete. Problem som ska öva elever på förmågan att bemöta problemställningar de inte har svaret på sedan innan och förhoppningsvis medföra större motivation än beräkningar från boken.
}

% exempeltext på syfte...
\textcolor{WildStrawberry}{
    Vi har en bild av att det existerar en svårighet i den svenska matematikundervisningen. Hypotesen lyder att bekvämligheten att hålla fast vid gamla metoder är mer närvarande än föreslagen kursplan vill få ut från undervisningen. Detta projekt har för syfte att testa detta och utforma en mängd problem som skulle uppfylla det syfte som kursplanen vill uppnå. }
\\ \\
    \textcolor{WildStrawberry}{
    Nedan finns uppgifter tagna från  kapitlet \textit{''problemlösning''} i en matematikbok ämnad för matematik 1c \cite{matte5000}. Boken trycktes 2011, alltså är uppgifterna designade utefter förändringarna som ska vara aktuella med GY11. Det krävs inte mycket tid för att hitta all information man behöver för att kunna lösa uppgifterna och när man väl hittat allt man behöver så ställer man upp en ''lös ut x'' uppgift.}

% Det är nog också mer relevant till "matematiken idag". Det existerar en bekvämlighetsfaktor just på grund av tiden är bristande och därför är det najs att använda sig av färdiga problem som inte har mycket tanke bakom sig. - Eleven får övning och "problemlösning (läsförståelse)"

%Här kommer några bra dåliga problemlösningsuppgifter ifrån \cite{matte5000} - MVH Björn

%Följande är en a uppgift, dvs en lätt uppgift:
\begin{displayquote}
\textcolor{turkos}{Marcus läser en bok som innehåller 420 sidor. Mellan kl 19.45 och 20.15 läser han 14 sidor. \\
Hur lång tid tar det att läsa hela boken?}
\end{displayquote}

%Svar

%Detta är en b-uppgift
\begin{displayquote}
\textcolor{turkos}{Jonas kör sin bil samma sträcka varje dag. Sträckan är en mil och Jonas brukar köra med hastigheten 90 $km/h$ en dag kör han sträckan med 100 $km/h$. \\
Hur många sekunder ''tjänar'' Jonas på det?}
\end{displayquote}

%Svar

%Följande två uppgifter är c-uppgifter, dvs de svåraste. 

\begin{displayquote}
\textcolor{turkos}{Vilket tal är x?\\
\( 2*5^x + 3*5^x = 25^{12} \)}
\end{displayquote}

%Svar

\begin{displayquote}
\textcolor{turkos}{En sandstrand är 2km lång, 30 m bred och 3 m djup. \\
Vi antar att ett sandkort ryms inom ett kubiskt område med sidan 0,2 mm.\\
Hur många sandkorn finns på stranden?}
\end{displayquote}

%Svar

%Samtliga fyra uppgifter har tagits från delkapitel 1.4 Problemlösning, som är del av 1 Aritmetik - Om tal. Finns liknande uppgifter i kapitlen om 2.2 Procentuellea förändringar och 3.2 Linjära ekvationer och olikheter. Dock saknas helt uppgifter om problemlösning för Geometri, Sannolikhetslära och statistik, samt Grafer och funktioner.

\section{Matematikundervisning nu och då}
    \textcolor{lila}{Matematik är en av de största delarna i skolan i Sverige idag, vilket bland annat visar sig genom att matematiken är kärnämne i både grundskolan och gymnasiet. Trots det är det ett ämne som många elever blir stressade över, och som ofta framställs som svårt, tråkigt, oanvändbart och abstrakt \cite{Ignacio&Barona}. En vanlig uppfattning verkar också vara att matematik är ett ämne som bara ett fåtal kan bli bra på \cite{Skolverket03} och ett ständigt återkommande inslag i media är det faktum att matematikkunskapen i Sverige har gått ner \cite{CompareOECD}. Men hur ser undervisningen ut idag? Vad kan man ändra för att förbättra dessa resultat?}


    
    \subsection{Traditionell matematikundervisning}
        \label{sec:Traditionellt}
        \textcolor{lila}{Den så kallade traditionella undervisningsmetoden består av två delar: \textsl{genomgång} och \textsl{egen räkning} \cite{traditionellMatte}. Det innebär att lektionen börjar med att läraren står framme vid tavlan och går igenom ny teori varefter varje elev individuellt får träna på detta med hjälp av ett stort antal likartade uppgifter. Därefter kan eleverna kontrollera om de gjort rätt genom att jämföra svaret med facit, och därefter gå vidare. Om man får rätt svar på alla uppgifter anser man sig ha förstått den nya teorin. Därefter upprepas samma procedur med ett nytt begrepp i centrum.} 
    
\textcolor{lila}{Med den här metoden lär sig eleverna olika matematiska begrepp och metoder, men först efter att det specifika begreppet eller metoden just presenterats. På det faktiska provet, när de själva måste ta reda på vilken metod som ska användas i varje uppgift, blir det betydligt svårare \cite{TheElephant}. }
\textcolor{WildStrawberry}{
    Just på grund av detta så tar man ett steg ifrån verklighetskopplingen och användbarheten av matematiken. När applikationen av teorin blir mekanisk istället för modellerande så tappar teorin syftet och blir mer av ett verktyg för att få ut rätt svar från en fråga. Fokus blir att man utnyttjar korrekt formel och snabbt får feedback från facit, eller andra hjälpmedel, om man fått rätt svar istället för att förstå problemet och dess underliggande moment för vad dem faktiskt innebär. }
    
    \subsection{Förändringar i matematikundervisningen}
        \label{sec:Forandringar}
        %Förändringar i matematikundervisningen

\textcolor{lila}{Den traditionella undervisningen var länge den som mer eller mindre uteslutande användes i Sverige, särskilt i högre åldrar. På senare år har man dock börjat att aktivt se över hur undervisningsmetoden skulle kunna förbättras.}

\textcolor{lila}{År 2011 infördes en ny och uppdaterad kursplan för gymnasiet, GY11, som bland annat påverkade matematiken. Jämfört med de tidigare kursplanerna från 2000 fanns det ett antal viktiga ändringar som gällde alla de olika matematikkurserna. %Ej nytt stycke!
Dels lägger de nya kursplanerna mer fokus på att kurserna ska anpassas efter varje program och inriktning. På så sätt plockas begreppet ''verklighetsanknytning'' upp på ett tydligare sätt. Man ska alltså lära sig hur matematik kan användas i vardagssammanhang som t.ex för att betala räkningar, men även i mer specifika sammanhang beroende på vad du kan behöva i yrkeslivet alternativt vidareutbildningen efter gymnasiet. 
En annan viktig ändring tar upp problemlösning. Detta har ingått även tidigare, men då bara som ett mål utöver de övriga. Nu ska det även användas som medel för inlärning av de andra målen. Kursplanen poängterar nu också att undervisningen ska varieras och innehålla undersökande aktiviteter. \cite{GY00-GY11}}

\textcolor{lila}{För att dessa förändringar ska kunna implementeras på ett så bra sätt som möjligt har man också gjort en stor satsning genom att utbilda alla lärare. Detta har gjorts genom \textsl{Matematiklyftet}, som är en kompetensutveckling i didaktik för lärare. Här belyses den kommunicerande, reflekterande och undersökande delen av matematiken. \cite{Nämnaren}}
            
\textcolor{lila}{Problemlösning är alltså ett mycket aktuellt ämne, som man lägger mycket resurser på att införa i matematikundervisningen. Det är dock en mycket stor förändring att genomföra, och det tar därför tid. Många lärare tycker också att det är svårt att hinna med problemlösning vid sidan av det material som redan ska täckas enligt kursplanen \cite{2016Senare}. Även de lärare som arbetar aktivt med att införa mer problemlösning stöter på problem. Kanske är det för att flesta elever är vana vid den traditionella undervisningen. Eleverna förväntar sig då att lärarna ska tala om precis hur man ska göra och vad som är rätt och fel. De gamla vanorna kan alltså sitta djupt hos både elever och lärare, och vara svåra att ändra.}

\textcolor{lila}{I mars i år (2017) beslutades också att skolan ska verka för att stärka elevernas digitala kompetens \cite{regeringen}. För gymnasiematematiken innebär detta att användingen av digitala verktyg ska bli mer central och programmering ska användas för att lösa matematiska problem \cite{itiskolan}. Även här ska det genomföras fortbildning av lärare \cite{prog_utbildning}.}
            
\textcolor{lila}{Vi har pratat med lärare som saknat tillräcklig hjälp i den här övergången till GY11. Trots den stora satsningen Matematiklyftet, där lärare utbildades om nya tankesätt kring matematikundervisning, så är det svårt att införa problemlösning i en klass som inte arbetat med det tidigare. Det finns gott om bra uppgifter, men det saknas hjälp med \emph{hur} man lär ut problemlösning från början. 
Ett rimligt antagande är att den kommande övergången, till att införa mer programmering och andra digitala verktyg i matematiken, också den kommer bli svår och ta lång tid att genomföra.}
%Ev. avsluta med något mer positivt, t.ecx hur vårt arbete kan hjälpa till
        
    \subsection{Verklighetsanknytning i matematiken}
        \label{sec:Verklighetsanknytning}
        \textcolor{lila}{En vanlig uppfattning är att det finns för lite verklighetsanknytning i matematiken som lärs ut \cite{TheElephant}. Mot denna bakgrund är det lätt att förstå att elever kan tolka ämnet som onödigt och irrelevant, vilket förklarar varför det är så viktigt att eleverna upplever uppgifterna som relevanta.}

\textcolor{lila}{Detta är ett faktum som många kursboksförfattare tagit fasta på, men tyvärr uppnår dessa försök inte alltid målet. Ofta känns den så kallade verklighetsanknytningen forcerad, och det blir snarare dåligt förklädda matematikuppgifter än faktiska problem som man kan föreställa sig att någon skulle vilja lösa. Detta riskerar att ge eleverna en känsla av att matematik inte är användbart, eftersom de får se så få exempel från dess verkliga användningsområden.
I vissa fall har man också tänkt för mycket på att den relevanta matematiken ska finnas med i uppgiften, vilket kan leda till att rimligheten blir lidande. Jo Boaler beskriver det som att eleverna inser att det finns ett speciellt ''matteland'', där det vanliga sunda förnuftet inte längre gäller. \cite{TheElephant}}
    
\textcolor{lila}{De textuppgifter som skrivs  i kursböcker med avsikt att införa en verklighetsanknytning kan också enligt vår erfarenhet i många fall brytas ner till standardproblem enbart genom att plocka ut siffrorna ur texten. På så sätt kan man också ofta bortse från den verklighetsanknytning som eventuellt finns i uppgifterna.}

\textcolor{lila}{Det verkar alltså vara viktigt med verkligehtsanknytning i matematiken, så att eleverna kan relatera till uppgiften och få känslan av att matematik är viktigt och användbart.}
        
    \subsection{Matematikundervisningen idag}
        \textcolor{WildStrawberry}{
    Den svenska matematikundervisning idag är relativt lik den traditionella matematikundervisningen, i stora drag gäller det att en lärare lär ut ett eller flera teoretiska begrepp inför sin klass och sedan ska klassen repetera dessa nya begrepp \ref{sec:Traditionellt}. }
    
    %Repetitionen i sin tur, kommer troligtvis innebära att eleven sitter med en lärobok som har en mängd definierade uppgifter där den nya teorin ska appliceras. Detta sker ändå trots att skolverket definierat ny läroplan som ska motverka . Detta moment kommer att hamna i en sluten loop tills det är dags för det stora provet där man testar alla begrepp man tidigare gått igenom. 

% <3 miss you <3

%ny rubrik? problemet?
%\textcolor{WildStrawberry}{
 %   Problemet med denna typen av undervisning är att eleven inte behöver känna igen det underliggande problemet, eleven kommer undan med att memorera hur en formel ser ut – utan att nödvändigtvis behöva förstå vad formeln gör. Inte för att testa förmågan att memorera saker är dåligt, men just på grund av detta så tar man ett steg ifrån verklighetskopplingen och användbarheten av matematiken. När applikationen av teorin blir mekanisk istället för modellerande så tappar teorin syftet och blir mer av ett verktyg för att få ut rätt svar från en fråga. Fokus blir att man utnyttjar korrekt formel och snabbt får feedback från facit, eller andra hjälpmedel, om man fått rätt svar istället för att förstå problemet och dess underliggande moment för vad dem faktiskt innebär. }


%vad matematiken & skola bör lära ut

% från intervju med Gymnasieelev när ställd frågan "blir ni skolade på hur man löser problem eller är problemlösningen ett vanligt matteproblem som är maskerat i text?": 
%" Mestadels det senare, vi gör problemlösningen i matten så man ska försöka hitta den användbara infon och lösa matteproblemet. Jag tar lite svårare matte men samma kurs som andra så min grupp får lite mer roliga problem där man måste använda logik i kombination med algebraisk matte...

% Men generellt så löser man mest maskerade matte problem"

\textcolor{WildStrawberry}{
    Enligt den nya läroplanen för matematik, så ska matematiken beröra problemlösning på så vis att man ska lära sig behärska sunt resonemang och logik \ref{sec:Forandringar}. Eleven ska alltså kunna bemöta uppställda situationer med metodik och kunna modellera lösningar från given information. Trots dessa förändringar så verkar det som applikationen av problemlösning i skolan är bristande. Elever får fortfarande uppgifter som är förklädda i en ''kort historia'' där målet egentligen blir läsförståelse över problemlösning . Givetvis låter det bra att ha mer problemlösning, men om inte mycket har förändrats så är försöket ett misslyckande \todo{har vi en bra källa på detta?}. När verklighetsanknytningen känns forcerad eller löjlig så missar man målgruppen. Sanningen blir att matematiken kommer känns mer oanvändbar \ref{sec:Verklighetsanknytning}. Än behöver alltså fler åtgärder vidtas innan man uppnår den typ av kreativ problemlösning som eftersträvas.
    %Från vår undersökning så får elever uppgifter, precis som i läroboken, maskerade i en kort saga som ska simulera ett problem. Reglerna är ofta tydliga och tanken är att man hittar siffrorna i texten och använder korrekt formel som man fått på undervisningen. 
}

\textcolor{WildStrawberry}{
    Nedan finns uppgifter tagna från en matematikbok, \cite{matte5000}, som trycktes 2011, alltså från en bok som är designad utefter förändringarna som ska vara aktuella med GY11.}

% <3<3<3
%(citat från intervju? hittas i .tex filen ovanför stycket).
% HEEEJ :D ändra precis som du könner är swag! jag bara får ut något på papper just nu :)
% Haha, det är bra att du skriver! :D Tänkte bara hjälpa till när jag såg det och kunde :)
% Super! :D All hjälp är toppen, tror du jag tänker rätt på denna sektion? den är ju mycket lik den om traditionell skola
% Ja... Det är jag lite osäker på... Funderar på om det inte är bättre att du försöker lägga in delar av det du skrivit i det stycket... Det är ju också svårt att påstå saker utan källor, så det måste vi försöka vara noga med. 
% AA exakt! Men på sätt och vis har vi en "intervju" med en duktig matte-student. Som är en källa. Dock en källa 
% - intJeo a,l ol skoor.
% wow :D

% Det är nog också mer relevant till "matematiken idag". Det existerar en bekvämlighetsfaktor just på grund av tiden är bristande och därför är det najs att använda sig av färdiga problem som inte har mycket tanke bakom sig. - Eleven får övning och "problemlösning (läsförståelse)"

% :( 
% okej <3 <3<3<3<3 till synes borta :smirk:
% Haha, förlåt xp
% Tänket bara säga att vi ju kan använda det som att vi vet att det händer, men inte som att det alltid är så. Vi kan också gå in mer på att det är svårt att hinna med, och ta in lite mer från planeringsrapporten. Precis, eller inte från lärarnas håll i alla fall ;)
% Ska vi ta bort detta nu kanske :p
%Fixade! ;) Inte helt borta i alla fall!

%Här kommer några bra dåliga problemlösningsuppgifter ifrån \cite{matte5000} - MVH Björn

%Följande är en a uppgift, dvs en lätt uppgift:
\begin{displayquote}
\textcolor{turkos}{Marcus läser en bok som innehåller 420 sidor. Mellan kl 19.45 och 20.15 läser han 14 sidor. \\
Hur lång tid tar det att läsa hela boken?}
\end{displayquote}

%Svar

%Detta är en b-uppgift
\begin{displayquote}
\textcolor{turkos}{Jonas kör sin bil samma sträcka varje dag. Sträckan är en mil och Jonas brukar köra med hastigheten 90 $km/h$ en dag kör han sträckan med 100 $km/h$. \\
Hur många sekunder ''tjänar'' Jonas på det?}
\end{displayquote}

%Svar

%Följande två uppgifter är c-uppgifter, dvs de svåraste. 

\begin{displayquote}
\textcolor{turkos}{Vilket tal är x?\\
\( 2*5^x + 3*5^x = 25^{12} \)}
\end{displayquote}

%Svar

\begin{displayquote}
\textcolor{turkos}{En sandstrand är 2km lång, 30 m bred och 3 m djup. \\
Vi antar att ett sandkort ryms inom ett kubiskt område med sidan 0,2 mm.\\
Hur många sandkorn finns på stranden?}
\end{displayquote}

%Svar

%Samtliga fyra uppgifter har tagits från delkapitel 1.4 Problemlösning, som är del av 1 Aritmetik - Om tal. Finns liknande uppgifter i kapitlen om 2.2 Procentuellea förändringar och 3.2 Linjära ekvationer och olikheter. Dock saknas helt uppgifter om problemlösning för Geometri, Sannolikhetslära och statistik, samt Grafer och funktioner. 

\textcolor{WildStrawberry}{
    Det krävs inte mycket tid för att hitta all information man behöver för att kunna lösa uppgiften. Författaren av uppgifterna har knappt heller försökt testa läsförståelsen av den som löser uppgiften med att slänga in irrelevant data. \todo{lol, savage stycke är savage} Det är möjligt att eleven hade fått mer ut av en färdiguppställd ekvation och mekaniskt fått öva på att bara lösa ut x, och ha mer tid för fler mekaniskt uträknande i uppgifter.}
        
    \subsection{Gruppindelning}
        \label{sec:Gruppindelning}
        \textcolor{cyan} {Det är vanligt att dela upp elever efter hur snabbt de anses lösa uppgifter. Elever som löser uppgifter snabbt grupperas med andra elever som löser uppgifter ungefär lika snabbt, och det samma gäller elever som anses lösa uppgifter långsamt och en grupp för elever som ligger mellan de två andra grupperna. \cite{Skolverket03}}

\section{Syfte}
    \textcolor{lila}{Med det här projektet vill vi bidra till att införa mer problemlösning i undervisningen, eftersom vi innan detta arbete hade uppfattningen att detta inte inkluderas tillräckligt. Projektets syfte är därför att skapa matematiska problem med förslag på tillhörande lektionsplanering som lärare kan använda i sin undervisning.}
\textcolor{Mahogany}{På så sätt vill vi minimera den tid som lärarna behöver lägga på lektionsplanering och på så sätt göra det lättare att variera kursboksmaterialet med mer problemlösning.} 
\textcolor{lila}{Med dessa uppgifter vill vi visa matematikens många sidor och ge eleverna en känsla för hur relevant matematiken är, och hur den kan användas. Problemen ska uppmuntra eleverna till att diskutera matematik och upptäcka friheten och kreativiteten som finns i matematiken. Flera av problemen bygger också på programmering, några direkt och ett mer indirekt. Med problemens tydliga struktur hoppas vi också kunna inspirera och förenkla för lärare som vill utveckla egna problemlösningslektioner utifrån egna idéer.}

\textcolor{lila}{Vi ville även undersöka om vår ursprungliga hypotes om problemlösning i undervisningen, som nämndes i bakgrunden (\ref{sec:Bakgrund}), stämmer samt ifall det finns några specifika faktorer som motverkar detta. I så fall ville vi anpassa projektet efter de motverkande faktorer som uppdagades, så att vi kan bidra till att minska dessa. För att göra det enkelt för lärare att hitta och använda problemen ska de presenteras på en webbplats, tillsammans med övrig information om hur de kan användas.}

%\textcolor{lila}{Syftet är att förse lärarna med problemlösninguppgifter, som presenteras i form av en hel lektionsplanering. På så sätt hoppas vi kunna hjälpa de lärare som tycker att det är svårt att hitta bra uppgifter alternativt inte anser sig ha tid till att planera lektionen till den grad som behövs om man frångår boken. Planeringen är dock tänkt enbart som en riktlinje, och varje lärare får själv avgöra hur mycket av den dom vill följa, samt lägga till moment som de anser passande.}

\subsection{Varför gymnasiet?}
\textcolor{lila}{
    Vi har valt att rikta in detta arbete specifikt mot gymnasiematematik. Även om vi tror att det är viktigt att utveckla matematiken på likande sätt i alla åldrar, så har vi valt en mindre målgrupp för att kunna genomföra arbetet på ett bra sätt och med ett användbart resultat. }
    
\textcolor{lila}{
    Det finns flera anledningar till varför valet av målgrupp föll på just gymnasiet. Dels finns det forskning som visar att en förändring i matematiskt tänkande, tvärtom vad man kan tro, är bättre att införa i den senare delen av skolan \cite{TheElephant}. Det är lätt att tro att en ändring som införs tidigt fortplantar sig till efterföljande matematikkunskaper, men så är det alltså inte i allmänhet. Gymnasiet är också sista steget innan man eventuellt fortsätter till universitet, högskola eller arbetsliv där man förväntas lösa problem utan att ha alla fakta och metoder givna på förhand. En försmak på detta är något som vi saknade i gymnasiet, och hoppas därför kunna ge till andra.}
    
\textcolor{lila}{        
    På gymnasiet har glädjen för att lära sig matematik, som är vanligt i de yngre åldrarna, för många elever till stor del ersatts utav rena prestationsmål \cite{Skolverket03}. Vi hoppas kunna visa en annan sida utav matematiken som väcker den glädjen igen.}
   
        

%   Borttaget ---
% Anledningen till varför vi inriktat oss mot gymnasiet är dels på grund av anledningen att vi internt känner att det i alla fall har existerat ett problem och vi tror att oavsett hur en kursplan förändras så kommer arbetet vara lika omotiverande så länge uppgifterna inte förändras. Man kan inte klandra lärarna och säga att de inte försöker inspirera och uppmuntra till en god miljö. Med ett förtroende för lärarnas kompetens så anser vi att feletproblemet inte ligger hos dem, utan det befinner sig i naturen av uppgifternas enformighet och fantasilöshet \ref{sec:Verklighetsanknytning}. Lärarna har för lite tid för att skapa intressanta uppgifter till sina elever och måste lägga tiden på allt annat omkring som finns med i kursplanen. Tillsynes verkar det inte heller finnas många som jobbar med att ta fram utmanande problem som passar en bredvid målgrupp av både högt - och -lågt presterande studenter. Målgruppen på sådana problem brukar ha en inriktning mot den ena eller andra gruppen av elever.}
    
    \subsection{Problembank}
        \textcolor{Mahogany}{
    Som nämnts i kapitel \ref{sec:Forandringar} så saknas hjälp med hur man både lär ut samt inkluderar mer problemlösning i undervisningen. Genom att tillhandahålla lärare allmän information kring hur problemen som utformats i detta projekt ska utföras och vad fokus ska ligga på, så underlättas lärare att få in mer problemlösning i sin undervisning. Det kan också om inget annat ge en variation av problem till de som finns i kursboken.
}

% Från "Undervisning i syfte att stödja och utveckla samtliga elevers individuella matematiska förmåga": I dagsläget används som beskrivits ovan läromedlet Matematikboken för grundskolans senare år X (Undvall m.fl. 2001a) i undervisningen i år 7. Detta läromedel innehåller många uppgifter, som också skulle kunna användas i arbetet inom ramen för utökad undervisningstid i matematik. Denna typ av uppgifter finner läsaren oftast under speciella rubriker såsom ”TEMA”, ”Träna problemlösning” respektive ”Lite av varje”, som bland annat innehåller olika typer av fördjupande huvudräknings-, grupp- samt diskussionsuppgifter där eleverna ges möjlighet att reflektera över matematiskt tänkande både enskilt och i grupp. Då dessa uppgifter är lättillgängliga att använda som resursuppgifter i den ordinarie undervisningen i matematik tre lektioner i veckan, speciellt då elevernas kompetens att självständigt ta sig an annorlunda uppgifter i matematik successivt kommer att öka tack vare det utökade matematikarbetet under veckans fjärde matematiktimme, är det önskvärt att i första hand uppgifter från annan litteratur än det ordinarie läromedlet användas för elevernas arbete under veckans fjärde matematiklektion. För undervisande lärare är det dock tidskrävande att åstadkomma variation med stor bredd, varför en gemensam planering med förberedda uppgifter kraftigt underlättar arbetet så att mer kraft kan ägnas åt projektets genomförande.
    
    \subsection{Programmering och matematik}
        \label{sec:ProgrammeringOchMatematik}
        \textcolor{Mahogany}{Att lära sig programmera är inte bara att lära sig ett programmeringsspråks syntax. Framförallt så är det att kunna bryta ner ett problem i mindre delar, i datasammanhang även kallat \textsl{subrutiner}, och definiera tydliga steg för hur man genomför dessa. Att få träning och efter hand färdighet för detta gör att man med större sannolikhet kommer att kunna bemöta ett nytt problem på ett mer systematiskt och rationellt sätt, oavsett om det är relaterat till programmering eller inte.}

\textcolor{Mahogany}{Eftersom en dator behöver exakta instruktioner och inte själv har förmåga att tolka vad som är rätt och fel så är det viktigt att man är tydlig med vad man menar att ett program ska göra. Vad en dator däremot är bra på är att utföra dessa instruktioner på väldigt mycket kortare tid än vad människor kan. Detta ger ett väldigt kraftfullt verktyg för många saker, faktum är att detta används i så stor utsträckning att vårt moderna samhälle still stor del är beroende av det. Det kan handla om att kunna betala sina räkningar på internetbanken på ett säkert sätt till att hitta ett lunchställe i en ny stad med hjälp av sökmotorer.}

\textcolor{Mahogany}{
    Kunskap i programmering är mer efterfrågad i samhället idag än någonsin tidigare. Att lära sig programmera är också en möjlighet för elever att med relativt fria tyglar få syssla med problemlösning. Detta är något som man har väldigt stor nytta av i matematik~\cite{TheElephant}, och som matematik i stor utsträckning även går ut på. Som vi nämnt i~\ref{sec:Forandringar} så kommer programmering framöver även ingå i kursplanen för gymnasiematematik. Detta gör det väldigt aktuellt att inkludera matematikrelaterade programmeringsuppgifter i vårt projekt, och de flesta av de programmeringsuppgifter som vi tagit fram har en tydlig koppling till matematiken.
}
    
    \subsection{Hemsidans roll}
        \textcolor{Mahogany}{Hemsidan är ett sätt att förmedla de problem som vi utformat till fler än de som vi testar problemen med genom att göra det mer lättillgängligt att ta del av problemen. Med hjälp av denna kan vi ge lärare möjligheten att ta del av våra problem när de vill och känner att de har tid över. Det hjälper oss också att få spridning på problemen som vi utformat. Kanske tyckte de att problemen var givande och delar med sig av sidan till en kollega, som kanske i sin tur gör samma sak. I slutändan så hoppas vi helt enkelt att så många som möjligt kan ha hjälp av de problem som vi utformat.}

\textcolor{Mahogany}{Hemsidan är också det som kommer att leva kvar efter projektet, så vi har valt att inkludera en kort beskrivning om oss och vårt arbete, samt en liten informativ text till lärare med vad vi vill uppnå med våra problem och hur vi tänkt att de bör utföras.}

\section{Varför är problemlösning viktigt?}
    \label{sec:Teori}
    \textcolor{lila}{För att förstå detta måste man först fundera på vad problemlösning egentligen är, vad det innebär och hur det kan användas. Detta kapitel förklarar därför begreppet och vilka fördelar som följer med denna typ av undervisning. Eftersom problemlösning ofta leder till intressanta diskussioner och därmed med fördel genomförs i grupp beskrivs även dessa delar mer ingående. Kapitlet avslutas med att förklara hur problemlösning kan gynna alla elever, oavsett nivå.}
    
    \input{Sections/Teori/Teori.tex}
    
    \subsection{Vad är ett problem?}
        \label{sec:problemdef}
        %\textcolor{Mahogany}{Vi definierar ett problem som en uppgift som man på förhand inte vet hur man ska lösa. }

% Vad är inte problemlösning
\textcolor{green}{Begreppet problem är ett brett uttryck och hur det ska tolkas är inte självklart. En studie från Umeå universitet \cite{problemVarierandeDef} försökte man klargöra hur begreppet problem var tänkt att tolkas enligt Skolverkets läroplan och jämföra det med hur lärare faktiskt tolkade begreppet. I studien framkom det att Skolverket i sin läroplan inte uttryckligen definierade vad ett problem var och att lärarnas uppfattningar om vad ett problem var varierade stort. Denna bakgrund belyser alltså tydligt att det inte råder konsensus kring någon specifik definition om vad ett matematiskt problem är. Idag däremot har Skolverket en definition på sin webbplats och eftersom denna rapport handlar om matematiska problem för gymnasiet anser vi det lämpligt att i första hand titta på vad skolverket säger i frågan. De skriver följande~\cite{ProblemDef}:}

\begin{displayquote}
\textcolor{green}{Ett problem är en uppgift som inte är av standardkaraktär och kan lösas på rutin. Det innebär att varje frågeställning där det inte på förhand för eleven finns en känd lösningsmetod kan ses som ett problem.}
\end{displayquote}

\noindent\textcolor{green}{Denna definition belyser svårigheten med att kalla något för ett problem, nämligen att den uppgift som är ett problem för en person, kan vara en enkel standarduppgift för någon annan. Trots detta kommer vi benämna våra framtagna matematiska problem som just problem, eftersom de är gjorda för att vara nya och utmanande för gymnasielever.} 
\textcolor{green}{I detta arbete har vi valt att använda en något mer sammanfattad version av samma definition som Skolverket använder, nämligen att:}

\begin{displayquote}
\textcolor{green}{Ett problem är en uppgift som man inte på förhand vet hur man ska lösa.}
\end{displayquote}

\noindent %\textcolor{green}{Notera att detta betyder att en uppgift kan vara ett problem för någon, medan den för någon annan, med andra förkunskaper, är ett standardproblem.Notera också att detta innebär att vi i vissa sammanhang kommer kalla de uppgifter vi gör för just uppgifter, även om målet är att de ska vara problem för gymnasieeleverna.}

\textcolor{green}{Problemlösning i sin tur beskrivs enligt Skolverket som förmågan att kunna lösa ett problem. Vidare skriver de att~\cite{ProblemDef}:}

% vår definition av problemlösning:
% "En uppgift som man inte på förhand vet hur man ska lösa. Notera att dettabetyder att en uppgift kan vara ett problem för någon, medan den för någonannan, med andra förkunskaper, är ett standardproblem. Notera också att dettainnebär att vi i vissa sammanhang kommer kalla de uppgifter vi gör för justuppgifter, även om målet är att de ska vara problem för gymnasieeleverna."

%\noindent \textcolor{lila}{
 %   Denna definition gör det på sätt och vis mycket svårt att skapa ett problem, eftersom den innebär att en uppgift som är ett problem för en person, kan vara en ren standarduppgift för någon annan. Definitionen innebär också att problemlösning är ett mycket brett område. Nedan presenteras några viktiga delar som, tillsammans eller var och en för sig, kan lyftas fram i ett bra problem.
%}

\begin{displayquote}
\textcolor{green}{Problemlösningsförmåga innebär att kunna analysera och tolka problem vilket inkluderar ett medvetet användande av problemlösningsstrategier som att till exempel förenkla problemet, införa lämpliga beteckningar, ändra förutsättningarna. Att lösa problemet innebär att genomföra ett resonemang där grunderna för resultatets giltighet blir tydligt och resultatet korrekt.}
\end{displayquote}

\noindent \textcolor{green}{
Definitionen visar att problemlösning är ett mycket brett område.}
% Värt att notera är att om problemet anses vara löst så ska det inkludera ett resultat. Alltså påstår denna definition även något om ett problems egenskaper, nämligen att ett problem ska ha ett \textsl{korrekt resultat}. Detta är som sagt Skolverkets egna definition och ingen officiell definition. Lösningen av ett problem kan exempelvis handla om att göra approximationer eller optimeringar och därför vill vi hävda att ett problem nödvändigtvis inte behöver ha \textit{ett} korrekt resultat. I övrigt håller vi med Skolverkets beskrivning om problemlösningsförmåga och problemlösning.}

\textcolor{green}{Nedan presenteras några viktiga egenskaper som hör samman med problem och problemlösningsstrategier. Dessa egenskaper kan tillsammans, eller var och en för sig, lyftas fram i ett bra problem.}
%\textcolor{WildStrawberry}{
%    Komplexiteten med att definiera vad ett problem är ligger i att det existerar olika syften med vad ett problem vill få ut från den som testas. Vissa problem vill att du hittar x - kan du hitta x? Andra problem kanske inte har ett exakt svar - tillvägagångssättet när man försöker lösa problemet är det som är utvecklande. Det vi vill ta ställning för är att en ''lös ut x''-uppgift som är dekorerad i en \textit{saga} skapar inte ett intressant problem och kunde lika väl varit den vanliga ''vad är x''-övningen.
%}

% Vad är problemlösning
%\textcolor{WildStrawberry}{
%    Det finns en mängd olika infallsvinklar man kan ta för att definiera \textit{ett problem}. Den som testas bör behöva fundera på vad som är viktigt i en given situation och skapa egen modellering av verkligheten. Ett problem bör skapa ett behov för teorin som kan appliceras och därav förhoppningsvis härleda för en djupare förståelse till varför teorin fungerar. När man fått fundera på hur man \textit{kan} gå till väga för att lösa problem innan man får underlaget så binder man en starkare koppling till materialet och bör därför komma ihåg det bättre\todo{källa}. Men vi vill definiera att \textit{ett problem} är en form av uppgift där man får arbeta med en situation eller uppställning som man inte kan svaret till på förhand.
%}

\textcolor{lila}{
    Till att börja med kräver problemlösning ett \textsl{undersökande arbetssätt}. Det handlar om att analysera problemet och bryta ner det i mindre delar, och därefter hantera varje del var för sig. Ofta måste man prova sig fram med olika lösningsmetoder innan man hittar en som fungerar.
}
        
\textcolor{lila}{
    En vanlig form av problemlösning är också att använda \textsl{öppna problem}. Det innebär ett  problem till vilket det finns flera olika möjliga lösningsgångar för att hitta ett svar, och detta svar behöver inte heller vara unikt utan kan variera beroende på vilka antaganden som gjorts. Ett exempel på ett öppet problem är att man ska planera en pool med en viss volym, vilket självklart kan göras på en mängd olika sätt.
}

\textcolor{lila}{
    En annan viktig del är att kunna översätta ett problem till \textsl{matematiska modeller}. Detta är en nyttig förmåga att besitta i många olika sammanhang, även i arbetslivet~\cite{TheElephant}. Det är också minst lika viktigt att kunna granska en modell med kritisk blick, och fundera på i vilka sammanhang den gäller och när den leder till orimliga resultat.
}
\textcolor{Mahogany}{Inte minst så är det ett kunskapskrav för gymnasiematematiken att kunna tillämpa matematiska modeller, något som Skolverket beskriver som \textsl{modelleringsförmåga} \cite{ProblemDef} \cite{skolverketMatte}. Denna förmåga beskrivs som följande:}

\begin{displayquote}
    \textcolor{Mahogany}{
        \textsl{Modelleringsförmåga innebär att kunna formulera en matematisk beskrivning – modell – utifrån en realistisk situation.}
    }
\end{displayquote}
\noindent\textcolor{Mahogany}{Vidare beskrivs denna modelleringsförmåga som förmågan att kunna koppla resultatet till en verklig situation och bedöma ifall det är rimligt, för att sedan eventuellt utvärdera modellen på nytt.}
        
    \subsection{Problembaserat lärande}
        \textcolor{green}{Problembaserat lärande är en pedagogik som går ut på att lära sig genom att lösa problem ell


I uppgifterna vi har tagit fram presenteras ingen teori som hör till uppgifterna, utan teorin blir någon som kommer fram under problemlösandets gång [precis som det står på x.x]. matematiska problem vi har gjort så presenteras ingen teori till uppgiften som det står i x.x. Utan eleverna möts av ett problem och får under processens gång finna}

\textcolor{green}{elever ska lösa de problem vi har tagit fram så presenteras}

\textcolor{green}{Den lärande processen som eleven går genom när hen löser problemen som vi har tagit fram kallas "inquiry-based learning". Det är en lärande metod som går ut på [källa].}

%
%Problembaserat lärande är den pedagogik som de flesta av våra uppgifter använder sig av.
%INQUIRY BASED LEARNING...
%Det är den karaktär våra problem har
%Vad är fördelen med det?
%Vi tror att eleverna får djupare förståelse för teori, om de först får se verkliga scenarion som den kan appliceras på
%matten känns inte irrelevant, utan kan lösa riktiga problem
%matematiken får även fler dimensioner än att vara ute efter ett enda rätt svar Detta eftersom uppgifterna är gjorda på ett sätt där de på precis samma sätt som många problem i verkliga inte har "ett rätt svar" utan snarare ska estimeras och optimeras. 
%Dag Wedellin använder i sin kurs x denna metod och har fått denna respons - detta tycker vi talar för att denna metod är bättre än den traditionella
%
%

        
    \subsection{Arbeta i grupp}
        \textcolor{cyan} {
Att låta elever arbeta med problemlösning i små grupper om två till fyra personer leder till att varje elev ges möjlighet att diskutera och reflektera angående problemet. Eleverna får prata om sina idéer till lösningar, lyssna på andra elevers idéer, samt även ges möjlighet att fråga, kritisera och bemöta kritik på ett positivt sätt. När eleverna får förklara sina tankar leder till ökad matematiksförståelse. \cite{RikaProblem}
}

\textcolor{cyan} {
Som tidigare nämnts i \ref{sec:Gruppindelning} är det vanligt att skolor delar in elever i grupper efter deras matematikfärdigheter, dock hävdar både Boaler och Rika Problem att det är bättre med heterogena grupper vad det gäller matematikkunskap \cite{TheElephant}\cite{RikaProblem}. Boaler skriver att klassrum där elever med olika förkunskaper jobbar med varandra med metoder anpassade för att alla elever ska lära av varandra är mer rättvisa\cite{TheElephant}. 
}

\textcolor{cyan} {
Rika problem rekommenderar att läraren blandar medelpresterande elever med högepresterande eller lågpresterande elever när man gör gruppindelningen, men varnar samtidigt för att inte göra grupperna extremt homogena eller heterogena\cite{RikaProblem}.
}

% I blandade grupper hjälper elever varandra vilket leder till fina ord som equality och allt blir bättre - Jo Boaler

% Även Rikaproblem rekommenderar att man blandar medelpresterande elever med högepresterande eller lågpresterande elever när man gör grupp indelningen, men varnar samtidigt för att inte göra grupperna extremt homogena eller heterogena. 


\textcolor{cyan} {
En negativ aspekt med grupparbete är att vissa elever kan bli sittande passivt medan resten av gruppen löser uppgiften åt dem. \cite{RikaProblem} %Ta upp att det blir värre ju fler man blir på någotsätt? 
}

%Bygga en rödtråd till förra kapitlet genom Japanska skolan. 



% Problemlösning i grupp anses vara roligare än vanliga mattelektion - Skolverket

% Saknas något om olika gruppstorlekar, samt nackdelar

% En studie visar att 2 är bättre än 1 förrutom för de allra bästa eleverna där det är ungefär lika bra, men 3-5 är strängt bättre. Den säger även att testpersonerna hade något roligare när det arbetade individuellt.

%Negativt - Om man börjar med grupparbete så kan vissa elever bli sittande passivt medan resten av gruppen löser uppgiften åt dem. - Rikaproblem



% Grupper bör ligga på 2-4 elever, eftersom det i en sådan grupp ger alla elever möjlighet att delta i diskussioner - Rika problem 

% Gruppering efter färdigheter är vanligt, enligt skolverket. 
        
    \subsection{Diskussion och kommunikation inom matematik}
        \textcolor{cyan} {
Givande diskussioner kan få elever att inse att det är tillåtet att ha egna idéer och tankar kring matematik, ett ämne som annars ofta upplevs handla om att följa regler. När elever diskuterar problem med varandra så lär de av varandra, och kan ofta uttrycka sig på ett sätt som kan göra det lättare för dem att förstå än vad ofta läraren kan.  \cite{TheElephant}
}

\textcolor{cyan} {
Faktum är att diskussion är så viktigt för bra problemlösning att Hagland, Hedrén och Taflin skriver att den viktigaste skillnaden mellan ett vanligt problem och vad de kallar ett rikt problem är att det rika problemet leder till diskussion. \cite{RikaProblem}
}

\textcolor{cyan} {
Även Skolverket lyfter att matematikundervisning där elevers egna lösningstratergier diskuteras leder till mycket positiva resultat och ökar elevers lust att lära. \cite{Skolverket03}
}

\textcolor{cyan} {
Ett land som tas upp som ett exempel på god matematikundervisning av både Skolverket och Boaler är Japan \cite{TheElephant}. I Japan läggs stor vikt vid att efter eleverna löst ett problem så ska de dela med sig utav sina lösningar och diskutera dem med varandra. Diskussionen används som en utgångspunkt för att läraren ska kunna lyfta viktiga aspekter ur deras lösningar och tillvägagångsätt. \cite{Skolverket03}
}
%Rika problem är problem som leder till diskussion  

%Ger chans att lyfta sina egna idéer och utveckla dem. 

%Diskussion leder till öka lust att lära. 

%Japanska skolan lägger fokus på diskussion och lyfts som ett föredöme av både skolverket och Jo Boaler 

% Tar upp strukturen med diskussion före och efter elever får lösa problemet. 
        
    \subsection{Göra eleverna delaktiga i undervisningen}
        
\label{sec:delaktighet}
\textcolor{WildStrawberry}{
    Om ett par elever lyckas komma på en lösning, eller del av en lösning, till ett problem så kan det bidra till att göra matematiken mindre främmande. Om en viss del av den teori som lärs ut i matematiken blir i form av egen utforskning, och att man lyckas komma en bit eller hela vägen, så kan denna belöning agera mycket motiverande~\cite{TheElephant}.} \textcolor{lila}{Men genom att utifrån en fråga upptäcka ett matematiskt begrepp ges detta automatiskt ett sammanhang och eleverna inser att det är användbart. På samma sätt är det motiverande för elever att känna att de äger problemet~\cite{TheElephant}, det vill säga att de själva får forma problemet i form av frågeställning och bakgrundsinformation. }
    
\textcolor{WildStrawberry}{
    All teori som en elev utsätts för under sin skolgång har varit ''verkliga'' problem som en gång inte varit färdiga formler eller modeller, som nu går att utnyttja. Vi anser att det kan vara ett kraftfullt verktyg för elevernas delaktighet i sin utbildning. När en individ behärskar förmågan att bemöta situationer objektivt så har hen större kapacitet till att förstå, utforska och utmana existerande begrepp. }
    
    % Att komma till denna insikten kan visa sig vara ett kraftfullt verktyg för studentens delaktighet i sin utbildning. Förmågan att ifrågasätta och kunna bemöta situationer objektivt ger individer en större frihet att förstå, utforska och utmana existerande begrepp.
        
        %Mycket av den teori som elever idag får lära sig under sin skolgång har kommit till under processen att lösa verkliga problem. Det är alltså verkliga problem som har lett till dessa teorier, som sedan har blivit formler. Att komma till denna insikten kan visa sig vara ett kraftfullt verktyg för studentens delaktighet i sin utbildning.
        
% 
        \label{sec:delaktighet}
        
    \subsection{Vem är problemlösning bra för?}
        \textcolor{Mahogany}{Enligt en studie om problemlösning i grupp så har elever ett stort behov av att arbeta med problem som är på deras nivå, då de behöver vara lagom svåra för att behålla motivationen uppe, samtidigt som de ska vara utmanande \cite{undervisningviaproblemlosning}. Detta gäller både elever som har fallenhet för matematik likaväl som de som har inlärningssvårigheter. Vidare så visar studien att traditionella matematikuppgifter sällan erbjuder detta då uppgiftsformuleringarna brukar innehålla ''nyckelord'' som elever lär sig att identifiera. Man blir då som elev fråntagen möjligheten att få övning med att möta och tolka nya problem, vilket kan leda till att man inte lär sig att utveckla någon studieteknik.}

\textcolor{Mahogany}{Studien visar även att risken att välja problem av fel svårighetsgrad minskar vid val av problem vars lösning på förhand inte är uppenbar. Detta eftersom de kan lösas på olika abstraktionsnivåer, och att detta tillåter att elever med olika förutsättningar inte hamnar utanför den ordinarie klassundervisningen.}

%\textcolor{Mahogany}{Problemlösning är att kunna arbeta med problem där framför allt tillvägagångssättet för att lösa problemet inte är uppenbart, samtidigt som det öppnar upp för olika lösningar för olika abstraktionsnivåer. Ytterligare så ska problemen uppmana till diskussion då detta är ett bra tillfälle för elever att motivera och försvara sina lösningar och med det förståelse.}
    
\section{Undersökning av hur matematikundervisngen ser ut idag}
    %\subsection{Enkät till matematiklärare om deras undervisning}
    %Lite text
    %Subsection: Enkät till matematiklärare om deras undervisning
    \textcolor{lila}{Innan vi började med projektet hade vi uppfattningen att problemlösning, trots den ändrade kursplanen, ändå inte inkluderas tillräckligt mycket i matematikundervisningen på gymnasiet. För att undersöka detta gjordes dels en enkät, och dels genomfördes en längre intervju med en lärare.} 

\subsection{Enkät till matematiklärare om deras undervisning}
    \label{sec:Bakgrundsenkat}
    \textcolor{lila}{Enkäten skickades ut via en facebook-sida för matematiklärare. Den gav lärarna en chans att berätta till vilken grad och på vilket sätt de arbetar med problemlösning, samt vilka faktorer de anser hindrar dem i detta arbete. Totalt fick vi in 58 svar från lärare över hela Sverige.}

    \textcolor{lila}{Resultatet består till största del av fritext där lärarna själva har fått uttrycka sin syn på de olika frågorna. Dessa svar har analyserats och sorterats utifrån olika gemensamma nämnare. De svar som presenteras i kursiv text nedan är alltså inte nödvändigtvis svar från enkäten, utan representerar det som vi tolkat som den bakomliggande tanken i de olika svaren. Notera även att ett enskilt svar från en lärare kan falla under flera av dessa kategorier.}

\subsubsection{Hur ser matematikundervisningen ut?}

    \textcolor{lila}{Här löd frågeställningen ''Uppskatta ungefär hur många procent av lektionstiden som spenderas på följande:'' och därefter följde förslag på vad man kan göra på en lektion, samt en punkt för ''Övrigt'' där man även fick specificera vad detta innebar. Resultatet av detta presenteras nedan, samt i figur~\ref{fig:PC}. Här har dock de svar som inte summerade till $100\%$ inte inkluderats, så antalet svarande lärare är här 47.}


\begin{figure}
    \includegraphics{Figures/Barcharts/diskussion.png}
    \includegraphics{Figures/Barcharts/uppfoljning.png}
    \includegraphics{Figures/Barcharts/genomgang.png}
    \includegraphics{Figures/Barcharts/rankning.png}
    \includegraphics{Figures/Barcharts/ovrigt.png}
    \caption{Figurerna visar stapeldiagram som representerar hur stor andel av lektionstiden som olika lärarna anger att de lägger på olika delar av undervisningen. På y-axeln representeras mängden svar i procent, och på x-axeln uppskattad tid lagd på de olika momenten. Notera att y-axeln växlar skala mellan de olika diagrammen.} %I figur (a)-(e) visas cirkeldiagram som visar hur stor andel av lektionstiden som olika lärare anger att de lägger på olika delar av undervisningen. Figur (f) visar det intervall (i procent) som varje färg representerar.}
    \label{fig:PC}
\end{figure}

    \textcolor{lila}{Genom att studera stapeldiagrammen i figur~\ref{fig:PC} kan man notera att den största delen av lektionstiden används till genomgång och egen räkning. Därefter följer elevdiskussion och uppföljning, och utöver detta lägger en liten andel av lärarna även tid på andra saker. Under övrigt faller framförallt laborationer, spel, digitala quiz, redovisningar framförda av eleverna samt problemlösning.}

    \textcolor{lila}{Några av lärarna kommenterade också i samband med den här frågan att de uppmuntrar eleverna att jobba tillsammans med uppgifterna i boken, och att det på så sätt blir mindre individuellt arbete, och mer diskussion mellan eleverna.}

\subsubsection{Vad är problemlösning för dig?}
    \textcolor{lila}{Här bad vi lärarna att skriva en kort förklarande text om hur de definierar problemlösning. I de svar vi fick in kunde vi hitta några olika karatäriserande åsikter, och undersöka hur stor andel av lärarna som nämner de olika delarna.}

    \textcolor{lila}{Många lyfte fram att ett problem är \textsl{en uppgift som man på förhand inte vet hur man ska lösa, och där man får applicera känd kunskap på nya situationer}. Detta är den vanligaste definitionen, och även den vi enligt avsnitt~\ref{sec:problemdef} använder i denna rapport. Hela $79\%$ hade med detta som ett kriterium i sina definitioner av problemlösning.}
    \textcolor{lila}{En annan viktig faktor, som nämndes av ungefär $20\%$, var \textsl{öppna problem}. Dessa definieras i avsnitt~\ref{sec:problemdef}. som problem som går att lösa på flera olika sätt, och i vissa fall även kan ge olika svar. Ett exempel på ett öppet problem är att man ska planera en pool med en viss volym, vilket självklart kan göras på en mängd olika sätt.}

    \textcolor{lila}{Därefter följde två kriterier, som vardera nämndes av cirka $16\%$. Det ena var att problemlösning \textsl{ska utgå från en större uppgift, vilket man måste använda flera olika metoder för att lösa}. Den andra pekar på att ett problem \textsl{innehåller antingen för mycket eller för lite information}. Det innebär att man antingen måste sortera ut det man behöver eller finna ytterligare information Detta kan göras genom att hitta denna via någon källa eller genom egna uppskattningar.}

    \textcolor{lila}{Ungefär $5\%$ av lärarna nämnde att problemlösning bör genomföras \textsl{i par eller grupper} och att det handlar om \textsl{diskussion och reflektion}. Lika många angav att det inkluderar att använda sig av \textsl{modellering}, att man får \textsl{tillämpa matematik} eller att problemlösningsuppgifter har en \textsl{verklighetsanknytning}. En liten andel poängterade att problemlösning ofta innebär att man \textsl{måste prova sig fram för att hitta en korrekt lösningsmetod} och cirka $7\%$ tog upp att problemlösningsuppgifter ofta utgår från en \textsl{textuppgift}. En lärare svarade även enbart med ordet ''Textuppgifter''.}

\subsubsection{Mer problemlösning i matematiken}
\label{sec:MerProblemlosning}

\textcolor{lila}{Hela $91\%$ anger att de arbetar för att inkludera problemlösning i sin undervisning. Detta motiveras till viss del med att det ingår i kursplanen, vilket $19\%$ av lärarna nämner i sin motivering. Men utöver det skriver även hela $81\%$ av lärarna att de gör det för att de tycker att det är viktigt med problemlösning. De poängterar att det är en stor del av matematiken, och att det är bra att kunna angripa problem man inte tidigare stött på i många sammanhang, även inför framtiden på arbetet eller universitetet. De $9\%$ som angav att de inte arbetade med problemlösning angav tidsbrist som en motverkande faktor. Det är mycket som behöver gås igenom, och då man ofta kan klara nationella provet utan problemlösning prioriteras detta ner. Någon sa också att det berodde på att de inte visste hur man skulle göra det på ett bra sätt, och att det är svårt att gå ifrån den undervisningsmetod man är van vid.}

\textcolor{lila}{När vi bad alla lärare att fundera på vad det finns för svårigheter med att införa mer problemlösning, fick vi många intressanta infallsvinklar. Av de som svarade nämnde $42\%$ att de tyckte att det är \textsl{svårt att hitta bra problem}. Ett bra problem ska ju vara utmanande för hela klassen, och då det ofta finns en mycket stor spridning i matematikkunskaperna hos en klass är detta inte helt lätt. Det tar också \textsl{tid}, och det nämns också som en stor bidragande faktor till varför man inte har mer problemlösning än vad man har, och nämns av $38\%$ av de svarande lärarna. Att elevernas tidigare skolår präglats till stor del av traditionell undervisning är också del av problematiken. $29\%$ av lärarna tar upp att eleverna ofta har för dåliga förkunskaper för att kunna ta sig an större problemuppgifter, samt att de också ofta är skeptiska till annan form av matematik än den de  är vana vid. Detta anges gälla speciellt för de högpresterande eleverna. En liten andel anger också att det som lärare kan vara svårt att ändra sin undervisning, och lätt att ''falla tillbaka i gamla hjulspår''.}

\textcolor{lila}{Trots dessa svårigheter är det ändå som sagt $91\%$ av lärarna som arbetar med att införa mer problemlösning i sin undervisning. Varför jobbar man så hårt med detta? Lärarna fick frågan om vad de ser för möjligheter med problemlösning. Många framhäver här att det är väldigt nyttigt att lära sig att \textsl{tänka på nya sätt}, och att även \textsl{lyfta styrkorna i olika tankesätt}. Det är också viktigt att kunna \textsl{hitta relevant information} vid ett givet problem, och detta är färdigheter som kan appliceras i många fler sammanhang än bara vid skolbänken. Mer problemlösning nämns också som ett bra sätt att få \textsl{tillämpa} sin kunskap, och därmed få se hur matematikens många delar kan användas i verkligheten, vilket ger en \textsl{relevans} till ämnet. Det är också ett bra sätt för eleverna att träna på att \textsl{samarbeta} samt att få \textsl{diskutera} matematik, samt att det \textsl{går att anpassa till olika nivåer}. Rätt uppgift kan vara ett problem för en hel klass, och ger alla en chans att \textsl{utgå från liknande villkor}.}
    
    \subsection{Intervju med Niklas Grip}
    %Vem är Niklas
\textcolor{turkos}{
För att få ett perspektiv ifrån någon som faktiskt undervisar i matematik och höra den personens åsikter om problemlösning kontaktade vi Niklas Grip. Niklas är ämneslärare i matematik på Mikael Elias teoretiska gymnasium i Göteborg och har bland annat kursen Matematik – specialisering som han inriktat mot just problemlösning.
}

% Hur genomfördes intervjun, samt våra frågor till honom. 
\textcolor{turkos}{
Intervjun genomfördes som ett djupgående samtal mellan oss och Niklas, där Niklas gavs stort utrymme att svara fritt. Samtalet var uppstyrt kring följande fyra frågor, samt följdfrågor på dessa: 
}
\begin{itemize}
  \item \textcolor{turkos}{Arbetar du med problemlösning i din undervisning?}
  \item \textcolor{turkos}{Hur definierar du problemlösning?}
  \item \textcolor{turkos}{Kan du ge några exempel på problem du använt i din undervisning?}
  \item \textcolor{turkos}{Hur arbetar du med teknik i din matematikundervisning?}
\end{itemize}

\noindent \textcolor{turkos}{
Följande är sammanfattning utav intervjun där vi försöker komprimera det det viktigaste som Niklas sa. De citat som tas upp kommer beskriva det som vi tyckte var av särskilt intresse eller vikt under intervjun. 
}

\subsubsection{Niklas syn på problemlösning}

% Hur undervisar han problemlösning? och %Exempel på problem han har använt. 
\textcolor{turkos}{
Niklas arbetar med problemlösning på flera olika sätt. Dels så använder han problemlösning som en metod för att introducera nya begrepp för sina elever, men han har även lektioner helt inriktade på problemlösning och öppna frågeställningar. Han berättar att han inte får in lika mycket öppna problem som han skulle önska. Som ett exempel på vad ett öppet problem är tar Niklas upp att han bett sina elever räkna ut hur stor sannolikheten är att bli träffad utav en fågelskit under ett liv. Andra problem han jobbar med är klassiska optimeringsproblem, så som att dra en kabel över en flod. Till sist så har han hand om gymnasiearbeten som inriktar sig mot problemlösning, några av hans elever räknade ut vilken ''starter pokémon'' som var bäst i ett pokémonspel. Själv tycker han att han lyckas få in hela skalan av problemlösning i sin undervisning.
}

%Problem med problemlösning?
\textcolor{turkos}{
Niklas ser tre svårigheter med sitt arbete med problemlösning. Den första är att han har många högpresterande elever som har en väldigt klar bild utav vad matematik är och som ofta blir negativt inställda till lektioner som går utanför deras bild av vad en matematiklektion ska innehålla. Den andra är att de nationella proven i framför allt matematik 2 till 4 inte testar problemlösning, vilket leder till att både Niklas och hans elever tappar lite av motivationen att jobba med problemlösning.}

\textcolor{turkos}{Den tredje svårigheten som Niklas nämner är det rent pedagogiska i hur man ska undervisa om problemlösning. Niklas upplever att överallt så talas det gott om problemlösning, men att det finns lite hjälp att få ifrån andra lärare eller andra personer när det gäller saker som hur man ska göra när en elev fastnar i ett problem. Han efterfrågar en undervisningskultur runt problemlösning där han kan diskutera sina erfarenheter och svårigheter med andra lärare: 
}

% Indrag på marginalerna, mindre text. Inget citattecken, beskriv i texten. Skriv efteråt hur vi tolkar det. 
\begin{displayquote}
\textcolor{turkos}{Det är väl kanske de här sakerna som jag sa förut att det finns en lite väldigt svag kultur och med det också goda exempel på hur man undervisar om just själva problemlösande. Det finns både forskning och litteratur om det, och det har funnits länge.}
\end{displayquote}

\noindent\textcolor{turkos}{
Niklas upplever alltså att trots tillgång till mycket resurser så känner han att hans arbete med problemlösning bedrivs mycket på egen hand. När det gäller undervisning utav andra delar av matematiken så har han hjälp utav böcker, andra lärare och YouTube-kanaler som visar hur man löser uppgifter. Han upplever däremot att det är väldigt få som inriktar sig på den generella förmågan att lösa problem. Niklas tar upp att han försökt lära sina elever att arbeta enligt Pólyas metoder (se \ref{sec:polya}). Han känner dock att han misslyckas få dem att förstå dem och inse vad som är relevant med dem, där hade andra lärares tankar kunnat vara till stor hjälp för honom.
}

%Hur definerar han problemlösning?
\textcolor{turkos}{
Niklas definierar själv problemlösning väldigt brett, för honom kan allt vara problemlösning. Huruvida en uppgift är ett problem eller ej beror på kunskapen hos personen som försöker lösa det:
}

\begin{displayquote}
\textcolor{turkos}{
Kommer man till problem som man inte riktigt vet hur man ska lösa så upplever jag att det är då man använder sin problemlösningsförmåga.
}
\end{displayquote}

\textcolor{turkos}{
Det är viktigt att man förstår var elevernas kunskap ligger. Niklas berättade att han ibland har haft lärarpraktikanter som varit med på hans lektioner där han använt problemlösning som en metod för att lära sina elever hur de ska använda enhetscirkeln. Praktikanterna förstod inte hur det kunde vara problemlösning när Niklas hade haft genomgång om enhetscirkeln dagen innan, men i och med att eleven själv fick ta reda på hur de skulle använda sig av den så blev det ändå problemlösning i slutändan. 
%Det är den idéen som Niklas använder som grund när han använder problemlösning för att introducera nya begrepp. Kan en elev inte lösa andragradsekvationer så det utmärkt tillfälle att både lära sig ett nytt begrepp och öva upp sin problemlösningsförmåga. 
}

%Hur jobbar han med teknik och tekniska hjälpmedel
\textcolor{turkos}{
Niklas har introducerat sina elever till Geogebra, kalkylark och programmering, och han ser hur verktygen hjälper eleverna att förstå vad olika uppgifter handlar om. Han förklarar att elever som använder sig utav Geogebra kommer kunna kontrollera sina lösningar X sätt än vad elever som löser uppgiften med enbart algebra kommer att kunna göra. Niklas trycker på att det han vill är att eleverna själva ska använda sig av verktygen utan att han behöver visa dem hur de ska göra. Det är när eleverna själva får sitta med verktygen som de verkligen börjar lära sig teknikerna.
} 

%\subsubsection{Niklas kommentarer på våra problem}
    
\section{Skapande av problembank med tillhörande problem}
    \subsection{Konstruktion av matematiska problem}
    \label{sec:Skapandetavproblem}
        % Ta ställning till omständigheter
\textcolor{Mahogany}{Något som är viktigt att ta ställning till när man utformar problem för gymnasiematematik är att det måste vara förenligt med övriga undervisningen. Man ska övertyga både lärare och elever att det för det första är relevant, men också att det kommer att vara mer givande än att räkna i boken. Det är också viktigt att ha som en röd tråd vad man egentligen vill att elever ska få ut av ett problem, och försöker förmedla detta på ett naturligt sätt. Är problemlösning bara något som tar tid från räkning i boken eller kan man få elever att förstå att de kan gynnas av denna typ av matematikundervisning? Ett ''bra'' problem bör rimligtvis lyfta fram den nyttan. Framför allt så bör de även vara roliga att lösa.}

% Hur bör det utföras?
\textcolor{Mahogany}{Vi har lagt ner en del tid på att informera lärare hur vi tänker att våra problem ska utföras. Detta är för att vi vill att man ska få ut så mycket som möjligt från problemen framförallt via diskussion och reflektion. Diskussionsdelen är också ett sätt för elever att våga försöka lösa ett problem som till synes kanske skulle vara avskräckande. Oavsett svårighetsgraden på problemet så anser vi att diskussionen är en viktig del i lärandet, då man får möjligheten att uttrycka sin nivå av förstående och ens kunskap testas genom att man behöver förklara det för någon annan. Samtidigt så gynnas de elever som inte kommit lika långt att få höra hur andra tänkt, och därifrån kunna arbeta vidare själva.}

% Rimlig nivå
\textcolor{Mahogany}{En svårighet med att utveckla problem är att sätta en rimlig svårighetsgrad, speciellt med tanke på att vi under utvärderingen fick jobba med framför allt väldigt specifik tidsram. Ett sätt att konfrontera detta är att ha en tydlig grund som problemet bygger på, för att sedan möjliggöra vidareutveckling i största mån. Denna vidareutveckling handlar ofta om fördjupningsfrågor som uppmanar till diskussion, vilket kan underbygga elevernas möjlighet att få en djupare förståelse för ämnet.}

    \subsection{Hemsidan}
        \textcolor{green}{Hemsidan utvecklades i programmeringsspråket Javascript. Fördelarna med att använda sig av ett programmeringsspråk som Javascript istället för att exempelvis bara använda HTML är att man kan designa sidan på många fler sätt och får total kontroll över sidans innehåll. Med total kontroll menas det att man skulle kunna ha annan material på webbplatsen än bara text- och bildmaterial. Det skulle kunna vara någon form av digitalt verktyg, t ex ett matematiskt problem som använder sig av en simulator. Detta kan göras med hjälp av Javascripts många olika bibliotek.}

\textcolor{green}{Biblioteket som användes för att utveckla webbplatsen var React JS. React JS är ett av de mest populära biblioteken att bygga webbplatser med i Javascript [källa]. I Reacts bibliotek finns främst verktyg för att jobba med vy-delen av en hemsida. Vill man integrera tyngre applikationer i webbsidan så finns möjligheter för det. Men om vyn är det största fokuset så är React ett mycket lämpligt val då det är väldokumenterat och gör det enkelt att bygga vidare på hemsidans design.}

        
\section{Den färdiga problembanken}
    %"Resultat": Hemsidan och alla problemen
        \textcolor{lila}{Det färdiga resultatet presenteras i form av en problembank i form av en hemsida med de skapade problemen samt tillhörande information.} 
    
    \subsection{De problem som vi skapat}
        \textcolor{lila}{Här presenteras huvuddragen av varje problem samt den tillhörande informationen. Notera att allt runtomkring själva problemet enbart är förslag, och att varje lärare kan anpassa genomförandet efter hur hen tror att det blir bäst i en specifik klass. Problemen i avsnitt \ref{sec:Fermi} till och med \ref{sec:Sortera} är allmänna problem som inte kräver några speciella förkunskaper från läraren, förutom det som skickas med som kompletterande information. Övriga problem baseras på programmering, och det är då lättare för läraren att hjälpa till med koden om denne kan programmera. Notera dock att det är problemlösningen som står i centrum, och inte kodens specifika syntax. Lärarna får även själva välja vilket programmeringsspråk som ska användas, men som vägledning skickas ett lösningsförslag skrivet i Java. Alla problemen finns även i sin fullständiga form i appendix~\ref{appendix:Problem}.}

\subsubsection{Fermiproblem}
    \label{sec:Fermi}
 
    \textcolor{lila}{Målet med detta problem är att eleverna ska få träna på att göra uppskattningar, samt att bryta ner ett problem i mindre delar.}
    
    \textcolor{lila}{Som inledning presenteras vad ett \textsl{fermiproblem} är. Det innebär att man, i fall där ett specifikt värde är svårt alternativt omöjligt att mäta, bryter ner problemet i många små delar och uppskattar varje del för sig. På så sätt kan man uppskatta lösningen på frågor som vid första anblick kan verka omöjliga. För att illustrera detta visas även ett exempel på ett fermiproblem, samt exempel på hur man kan dela upp det i minde delar.}

    \textcolor{lila}{Eleverna får en lista med olika fermiproblem att välja mellan, och ska arbeta i grupper om 2. Först ska de gissa på svaret, och sedan beräkna det genom att dela upp i delar som man kan uppskatta. Därefter får de i mindre grupper presentera och diskutera sitt arbete. Slutligen diskuteras i helklass om resultaten kändes rimliga, hur genomförandet gick, ifall lösningsmetoden är användbar samt varför den fungerar så bra som den gör. För den sista frågan ger vi även lärararen svaret, det vill säga att det fungerar eftersom man ibland överskattar och ibland underskattar de mindre delarna, vilket gör att det slutgiltiga resultatet ofta blir en mycket bra uppskattning.}

\subsubsection{Flygplan}
    \label{sec:Flygplan}
    
    \textcolor{lila}{Detta  problem syftar till att träna eleverna på ett undersökande arbetssätt där inte alla påverkande faktorer är givna, utan måste resoneras fram av eleverna. Den leder också fram till ett ekvationssystem, vilket ger ett exempel på när dessa är användbara.}
    
    \textcolor{lila}{En Sverigekarta med utmarkerade flygplatser och flygrutter presenteras för klassen, se figur~\ref{fig:Flygplan}. Detta görs bitvis, med plats för en kort diskussion om vad frågeställningen skulle kunna vara, givet den dittills givna informationen. Med all information given får eleverna i grupper om två arbeta med en specifik sträcka, Visby-Karlstad. På vilka olika sätt kan man ta sig mellan dessa två städer? Vilken sträcka är bäst och vilka faktorer påverkar detta? Därefter specificeras uppgiften ytterligare, genom att de får reda på hur mycket det kostar att åka mellan två olika direkta flygsträckor. De ska nu hitta den \textsl{billigaste} vägen mellan Karlstad och Visby, samt vilken väg som blir billigast om man ska från Karlstad till Stockholm, men sträckan Stockholm-Göteborg är fullbokad. Den avslutande helklassdiskussionen tar bland annat upp vilka faktorer som påverkar bränslekostnaden, vilka faktorer som påverkar biljettpris för en specifik sträcka och ifall det är det totala avståndet eller antalet mellanlandningar som avgör priset för en resa.}
    
    \textcolor{lila}{Under punkten ''Ytterligare information'' diskuteras tankar bakom uppgift och diskussion. Eleverna får själva upptäcka att priset beror på sträckan, men att det också tillkommer ett fast pris för start och landning, vilket speciellt visar sig i att det för sträckan Karlstad-Göteborg blir billigare att flyga en längre sträcka, men med färre mellanlandningar. De får också själva reflektera över ytterligare faktorer som skulle kunna påverka priset, till exempel löner och vinstmarginal.}
    
\subsubsection{Fritt fall}
    \label{sec:FrittFall}
    
    \textcolor{lila}{Tanken med problemet är att eleverna ska få använda derivata utifrån en verklig situation istället för utifrån en färdig formel, samt få en djupare förståelse för vad derivata egentligen är.} 
        
\begin{figure}
    %\centering
    \hspace{0.4cm}
    \begin{subfigure}[b]{0.45\textwidth}
        \centering
        %\hspace{-40pt}
        \includegraphics[width=0.9\textwidth]{Figures/Flygplan_rapport.png}
        \caption{\textsl{Sverigekarta med utmarkerade flygplatser och rutter.}}
        \label{fig:Flygplan}
    \end{subfigure}
    \hfill
    \begin{subfigure}[b]{0.45\textwidth}
        \centering
        %\hspace{-40pt}
        \includegraphics[width=0.7\textwidth]{Figures/FrittFall.PNG}
        \caption{\textsl{Fallande boll med angivna fallsträckor för varje sekund.}}
        \label{fig:FrittFall}
    \end{subfigure}
    \hspace{0.4cm}
    \caption{Figurer tillhörande två av problemen, ''Flygplan'' till vänster och ''Fritt fall'' till höger.}
    \label{fig:three graphs}
\end{figure}
    
    \textcolor{lila}{Som en introduktion till problemet diskuterar man tillsammans i klassen vad en derivata är och hur man kan beräkna den. Därefter presenteras en bild av en fallande boll, med utmarkerade fallsträckor för olika tider, se figur~\ref{fig:FrittFall}. Frågan är nu vad man utifrån detta kan komma fram till om bollens hastighet och acceleration, vilket klassen får arbeta med i grupper om två. Tanken är att man ska konstruera en graf över sträcka som en funktion av tid, och utifrån denna använda en linjal för att rita tangenter längs med grafen och därmed skapa en graf över hastighet som en funktion av tiden. Därefter kan samma procedur genomföras för att rita accelerationen som funktion av tiden. Denna bör, med avvikningar på grund av felkällor, bli konstant och ungefär lika med tyngdaccelerationen, det vill säga ungefär~10. Avslutningsvis diskuteras denna metod med avseende på resultat och noggrannhet.}
    
\subsubsection{Försvåring av en ekvation}
    \label{sec:ekvation}

    \textcolor{lila}{Detta är en mycket fri uppgift med målet att ge eleverna en djupare förståelse för ekvationer och ekvationslösning.}

    \textcolor{lila}{Inledningsvis diskuteras begreppet ekvation, samt hur man löser en ekvation, i helklass. Det viktiga här är att komma fram till att så länge man gör samma sak på båda sidorna om likhetstecknet, och använder prioriteringsreglerna rätt, så får man göra vad som helst. Eleverna får därför två och två arbeta med att ''försvåra'' ekvationen x=2, genom att i flera steg göra en bestämd operation i både vänster- och högerledet. Följande enkla exempel på hur man kan börja presenteras för eleverna som tips på hur man kan börja}
    
        \begin{equation*}
            x=2
        \end{equation*}
        \begin{equation*}
            2x=4 \quad (\cdot2=\cdot2)
        \end{equation*}
        \begin{equation*}
            2x+5=7+2 \quad (+5=+3+2)
        \end{equation*}
    
    \noindent\textcolor{lila}{Varje steg ska även skrivas upp och motiveras enligt ovan. Därefter diskuteras lösningsgången, samt potentiella utvecklingar. Eleverna får också diskutera ifall det finns någon lösning på ekvationen, och hur den i så fall ser ut. De får på så sätt reflektera över att de nu har ''löst en ekvation baklänges'' och att varje ekvation, som kan tyckas se jobbig ut, kan brytas ner i små, enkla steg.}

\subsubsection{Matematisk modell för bil och löpare}
    \label{sec:lopare}
    
    \textcolor{lila}{Detta problem är ursprungligen förklätt som en enklare standarduppgift, men låter därefter eleverna fundera över den matematiska modell de har använt, samt ifall den är rimlig. Målet är att visa fördelar och nackdelar med matematiska modeller, samt införa ett sunt kritiskt tänkande hos eleverna.}
    
    \textcolor{lila}{Inledningsvis får eleverna i par lösa två till synes likartade uppgifter av standardkaraktär. I den ena får de givet hur snabbt det tar att köra en viss sträcka med bil, och ska beräkna hur lång tid ett antal andra sträckor tar att köra. I den andra uppgiften är bilen utbytt mot en löpare, men för övrigt är det samma frågeställning. Vi upplyser här läraren om att vi antar att de flesta elever kommer använda en linjär modell, dvs anta att både bilen och löparen håller samma fart oavsett sträcka. Därefter diskuterar men resultaten i helklass, samt vilka antaganden man har gjort, om de är rimliga och om det skiljer sig mellan bilen och löparen. Som jämförelse får man reda på att den tid som med den linjära modellen fås för den längsta sträckan för löparen är betydligt snabbare än världsrekordet, och man får även verkliga tider för de efterfrågade sträckorna\footnote{Tiderna för löparen som använts är tagna från hur fort Axel, medförfattare i detta arbete, springer dessa sträckor.}. Utifrån detta får eleverna diskutera vilka faktorer som avgör, till exempel att man orkar springa snabbare på kortare sträckor, men också att startsträckan tar upp större delen av tiden.}
    
\subsubsection{Sortera en kortlek}
    \label{sec:Sortera}
    
    \textcolor{WildStrawberry}{
        Kortleken är ett enkelt sätt för en elev att få en god introduktion till hur enkla sorteringsalgoritmer fungerar samt hur man nyttjar sig av dem.}
        
    \textcolor{WildStrawberry}{
        Lektionen kommer innefatta att eleven får en introducerande beskrivning på vad en algoritm är och hur den kan användas. Sedan kommer eleverna få fundera en stund på hur man kan utnyttja algoritmer till att bryta ner större problem till enkla upprepande arbetsflöden. Till uppgiften så kommer kortlekar att tillhandahållas till eleverna som får instruktioner på hur man får interagera med sina kort när man försöker sortera dem. Dessa instruktioner efterliknar sättet en dator jämför och byter position på element i en lista. När eleverna fått experimentera och kommit fram till sina algoritmförslag så skriver de ner sina instruktioner. De byter instruktioner med en annan grupp och ska kunna sortera sina lekar genom att enbart följa instruktionerna, precis som en dator hade gjort. }
        
    \textcolor{WildStrawberry}{
        Den stora lärdomen i denna uppgift är att eleverna i fråga troligtvis lyckas skapa algoritmer som har givna namn och som faktiskt används i ''den riktiga världen''.  Eleverna kommer få reflektera över vad som gått bra, eller dåligt, och hur man kan göra optimeringar eller fixa problemen (''buggarna'') som uppstod. Ungefär som en riktig programmerare får göra. \todo{:)}}
        
\subsubsection{Approximera ett irrationellt tal}
    \label{sec:approx}

\subsubsection{Binär till decimal}
    \label{sec:binar}
    
\subsubsection{Fibonaccis talsekvens}
    \label{sec:Fibonacci}
    
    \textcolor{WildStrawberry}{
        Fibonaccis talsekvens är mycket klassiskt problem när det kommer till programmering. Det är ofta en startpunkt för rutinerade programmerare som försöker lära sig syntaxen i nya programmeringsspråk (förutsatt att man känner till lösningen). }
        
    \textcolor{WildStrawberry}{
        Undervisningen börjar med att läraren introducerar Fibonaccis talsekvens och hur talföljden utvecklar sig. Efter detta kommer eleverna försöka skapa egna stycken kod som ska skriva ut talföljden. Mer erfarna programmerare uppmuntras till att prova lösa uppgiften med hjälp av rekursion. När väl eleverna provat att lösa uppgiften så kommer en diskussionstund där man bör reflektera på hur det har gått, hur väl sin kod fungerar och vilka optimeringar som man kan göra. Slutligen kan man diskutera och reflektera över hur man kan använda sig av loopar för att lösa problem.}
        
    \textcolor{WildStrawberry}{
        I denna uppgift får får eleven träning i iterativa och rekursiva loopar. Båda vilka är verktyg som programmerare behöver kunna använda sig utav. Lär man sig behärska dessa tekniker så kommer man i sin följd kunna lösa andra problem som har iterativ, eller rekursiv, struktur.}
        
\subsubsection{Identifiera primtalsfaktorer}
    \label{sec:primtal}

\subsubsection{Personnummer}
    \label{sec:Pnr}
    
    \textcolor{WildStrawberry}{
        slack}
        
\subsubsection{Primtalsfaktorer}
    \label{sec:Primtal}
    
    \textcolor{WildStrawberry}{
        Att kunna faktorera tal in till primtal är något man kan råka stöta på i sitt liv. Kanske speciellt om man någon gång behöver nyttja sig av eulers phi-sats..... \todo{this is bajs!} }
        
    \textcolor{WildStrawberry}{
        För att introducera detta problem så krävs ingen lång bakgrund, så exempelvis kan läraren skriva upp en mängd siffror som elevernas program ska kunna köra. Sen kommer eleverna, i små grupper om 2 eller 3, försöka skapa program som hanterar alla fallen. Slutligen kommer klassen diskutera hur man hanterat fallen där talen haft fler än 2 primtalsfaktorer, upprepande faktorer och vilka ''fel'' man kan fånga med sina program. }
        
    \textcolor{WildStrawberry}{
        Under detta tillfälle så kommer eleven få öva på enkla booleanska uttryck, matematiska operatorer, optimering av kod och eventuella fält eller listor. Det blir en salig blandning av olika moment i programmering och man utsätts för många olika syntax i en och samma uppgift.}

\subsubsection{Skapa ett chiffer}
    \label{sec:chiffer}
    
\subsubsection{Sorteringsalgoritmer}
    \label{sec:sorteringsalgoritmer}
    
    \subsection{Hemsidan}
        \textcolor{Mahogany}{Hemsidans design har gjorts enkel där känslan ska vara att man hela tiden är på ''en'' sida. Den består av olika problem som listas i ett rutnät på ''Problem''-sidan, innehållandes titel och bild. Klickar man på ett av problemen i rutnätet så kommer man vidare till en mer detaljerad vy över problemet med en kort beskrivande text följt av en länk till det fullständiga problemet. I många fall bifogas även en tillhörande PowerPoint, som är tänkt att underlätta utförandet. För programmeringsuppgifterna så bifogas även ett lösningsförslag skrivet i Java.}

\textcolor{Mahogany}{Hemsidan har även en startsida som designats för att kännas inbjudande och engagerande, där besökaren ges en introducerande text om vad vi arbetar med samt en knapp som tar personen direkt till problemen. Dessa kommer man givetvis också åt via menyn, skillnaden är att vi vill fånga uppmärksamheten på ett mer välkomnande sätt för den som besöker hemsidan för första gången genom att ha en knapp över en bild som illustrerar några av våra problem.}

\textcolor{Mahogany}{Till sist så inkluderas sidan ''Om oss'', som beskriver projektets arbete, samt en sida med ''Information till läraren'', vars innehåll beskrivs närmare i \ref{sec:Skapandetavproblem}.}
        
    
\section{Test av problemen}
    
    \subsection{''Metod''}
        \textcolor{lila}{Problemen har tillsammans med tillhörande material som presenterades ovan skickats ut till olika svenska gymnasielärare. Varje lärare fick ange vilka kurser de undervisade i för tillfället, och fick därmed ett problem tilldelat som skulle passa till någon av de angivna kurserna.}

\textcolor{lila}{Lärarna hade som tidigare nämnt möjlighet att följa det givna upplägget till den grad de själva ville, och därefter fick de en utvärderingsenkät. Den innehåller frågor om hur de tyckte att problemet preseterats från vårt håll, om de valt att ändra något eller om det var något annat de saknade samt om hur de tyckte att problemet togs emot av eleverna. Eleverna fick en liknande utvärderingsenkät där de fick skriva sin definition av problemlösning, vad de tyckte om problemet, vad de tyckte om nivån på problemet samt om de upplevde att de lärt sig något.}

        
    \subsection{''Resultat''}
    \label{sec:slutenkat}

\section{Diskussion}
    \textcolor{Mahogany}{
    I det här avsnittet så lyfts bland annat frågan kring vad vi anser vara ett bra problem och vad man ska få ut av det. Vidare diskuteras projektets process, framför allt sett till hur vi utformat problemen och hur vi kunde utfört projektet annorlunda. Där läggs vikt vid vilka begränsningar vi fått göra, brister kring testningen samt urvalet av testpersoner.
    Ytterligare så försöker vi svara på huruvida vi lyckats underlätta för lärare att få in mer problemlösning i deras undervisning, även efter projektets slut.
}
    
    \subsection{Vad är ett bra problem?}
        % Vi testade bara problem som utfördes under 1h
\textcolor{Mahogany}{
    %Målet med de problem som vi utformat i projektet är att framhäva vikten av problemlösning i matematiken. Samtidigt så är också ett mål att visa hur den kan vara användbar, och att den användbarheten inte enbart existerar i det ''matteland'', som beskrivs i \ref{sec:Verklighetsanknytning}.  Man vill inte heller, som nämndes i avsnitt~\ref{sec:problemdef}, riskera att verklighetsanknytning blir en alltför stor faktor i problemskapandet. Ibland kan ett abstrakt problem leda till ett större engagemang och nyfikenhet som ett med en tydlig verklighetsanknytning. En utmaning med att utforma verkliga problem är därför att ta ställning till detta samtidigt som man framhäver den mångsidiga nyttan av matematiken.
    Ett mål med de problem som vi har utformat är att vi vill visa hur användbar matematik kan vara, och att den användbarheten inte enbart existerar i det ''matteland'', som beskrivs i \ref{sec:Verklighetsanknytning}.  Man vill inte heller, som nämndes i avsnitt~\ref{sec:problemdef}, riskera att verklighetsanknytning blir en alltför stor faktor i problemskapandet. Ibland kan ett abstrakt problem leda till ett större engagemang och nyfikenhet som ett med en tydlig verklighetsanknytning. En utmaning med att utforma verkliga problem är därför att ta ställning till detta samtidigt som man framhäver den mångsidiga nyttan av matematiken.
}

%\textcolor{Mahogany}{
    % Detta kan vara betydligt lättare att uppnå med programmeringsproblem, då man, utöver programmeringsspråkets syntax, inte är bunden till hur man kan lösa något av problemen. Man ges också i många fall möjligheten att själv avgränsa hur utförligt problemet ska utföras. Som exempel så har vi utformat ett problem där man ska verifiera ett personnummer, där man företrädesvis kan använda sig av den så kallade Luhn-algoritmen\cite{Luhn}. Dock så kanske det inte räcker med att enbart kontrollera med hjälp av den algoritmen. Om man kontrollerar personnummer på de i sin klass så kanske man vill kontrollera att personen inte är född på 1800-talet exempelvis!
%}
 
% Röd tråd samt övertyga elever att det är relevant
\textcolor{Mahogany}{
    Med våra uppgifter vill vi övertyga både lärare och elever att de är relevanta, men också att de kommer vara mer givande än att enbart räkna i boken. Det är också viktigt att ha som en röd tråd vad man egentligen vill att elever ska få ut av ett problem, och försöka förmedla detta på ett naturligt sätt. Är problemlösning bara något som tar tid från räkning i boken eller kan man få elever att förstå att de kan gynnas av denna typ av matematikundervisning? Ett bra problem bör därför rimligtvis lyfta fram den nyttan. Framför allt så bör problemen även vara roliga att lösa, och ge eleverna en chans att utforska och utmanas av matematiken.
}

% Hur bör det utföras?
\textcolor{Mahogany}{
    Vi har lagt ner en del tid på att informera lärare hur vi tänker att våra problem ska utföras. Detta framförallt för att vi anser att diskussion och reflektion är mycket viktiga för inlärningen, vilket även diskuterades i avsnitt~\ref{sec:Diskussion}. Diskussionsdelen är också ett sätt för elever att våga försöka lösa ett problem som till synes kanske skulle vara avskräckande. Oavsett svårighetsgraden på problemet så anser vi att diskussionen är en viktig del i lärandet, då man får möjligheten att uttrycka sin nivå av förstående och ens kunskap testas genom att man behöver förklara det för någon annan. Samtidigt så gynnas de elever som inte kommit lika långt av att få höra hur andra tänkt, och därifrån kunna arbeta vidare själva.
}

% Rimlig nivå
\textcolor{Mahogany}{
    En svårighet med att utveckla problem är att sätta en rimlig svårighetsgrad, speciellt med tanke på att vi under utvärderingen fick jobba med en väldigt specifik tidsram. Ett sätt att konfrontera detta är att ha en tydlig grund som problemet bygger på, för att sedan möjliggöra vidareutveckling i största mån. Denna vidareutveckling handlar ofta om fördjupningsfrågor som uppmanar till diskussion, vilket kan underbygga elevernas möjlighet att få en djupare förståelse för ämnet.
}
        
    %\subsection{Begreppet problemlösning}
    %    % Om det finns något här som ni vill ha med i rapporten så är det bara att plocka, men personligen så kände jag att det här inte längre tillförde något.

\textcolor{Mahogany}{
    Från svaren på en av våra undersökningsenkäter så märkte vi att begreppet problemlösning skiljde sig från lärare till lärare, där vissa kände att räkning i boken var problemlösning. Det är dock inte ett helt självbeskrivande begrepp som med fördel kan förtydligas.
}

\textcolor{Mahogany}{
    Även om problemlösning är en del av kursplanen i matematik så är det tydligt att det skiljer sig i vilken utsträckning och form det utövas. Som nämnts i \ref{sec:Forandringar} så ska problemlösning vara både ett mål och medel i undervisningen enligt kursplanen för gymnasiematematik, dessvärre så förlitar sig lärare i stor utsträckning enbart på de problem som kursboken tillhandahåller. Därtill blir den typen av problem ofta enbart extrauppgifter för de som hinner \cite{2010UndervisningenGymnasieskolan}. Det är alltså rimligt att anta att problemlösning i gymnasiematematik är bristfällig samt att lärare behöver fler verktyg och resurser för att uppnå dessa mål.
}

\textcolor{Mahogany}{
    Vi tror att en stor del handlar om att låta elever få diskutera med varandra, men utformningen av rätt typ av problem är givetvis också viktigt. Det ska inte mynna ut till att handla om att ta ur värden och sedan applicera dessa på aktuell teori som nyss gåtts igenom. Men det gäller också att ge lärare resurserna och samtidigt utmana deras uppfattning om vad problemlösning är, för att på så sätt kunna öka andelen av vad vi ser som problemlösning.
    Kanske är en stor anledning att det inte är mer problemlösning i undervisningen att lärare i olika utsträckning är osäkra på hur de ska genomföra en sådan undervisning, med en kombination av att inte ha rätt resurser. Frågan är om man uteslutande kan förlita sig på att enbart kursböckerna ska täcka denna del av kursplanen, men samtidigt krävs tid och resurser om man vill förbättra detta, något som många lärare känner att de inte har.
}
    
    \subsection{Processen av att ta fram problem samt dess begränsningar}
        \textcolor{Mahogany}{Att ta fram problem har inte varit lätt. Till en början så hade vi ett angreppssätt där vi inte alltid tog hänsyn till nivå och nödvändig teori, utan tog för givet att eleven med hjälp av nyfikenhet skulle vilja ta till sig den nya teori som skulle vara nödvändig för att lösa uppgiften. 
Ju mer kontakt vi fick med lärare ju mer insåg vi att vi var tvungna att i någon mån anpassa våra problem för att de skulle passa in i deras undervisning. Vi bestämde oss därför också för att avgränsa oss till kortare problem. Detta dels för att underlätta testning och dels för att de skulle vara lättare för lärare att inkludera i sin undervisning även efter projektets slut. Vi har dock även diskuterat om att utforma längre problem också. I vissa fall kan det nämligen vara bra att få gott om tid för att fundera, och även ha tid till att testa olika metoder för att se vilken som fungerar bäst. Denna tanke ligger bakom till exempel ''Fermiproblem'', som är planerad för att kunna utföras under en enda lektion, men även kan utföras som ett längre projekt som kan avslutas med att eleverna får presentera sina olika resultat och metoder för klassen.}

\textcolor{Mahogany}{Med det sagt så har vi alltså begränsat oss till kortare problem om cirka 50 minuter. Hade vi haft mer tid och möjlighet hade vi troligtvis även velat testa längre, mer projektliknande problem. Eftersom nästan hela projektgruppen tidigare läst kursen \textsl{Mathematical modelling and problem solving}\cite{matmod} där man fick veckovisa problem, samt en som sträckte sig över merparten av kursen, så är vi också medvetna om vikten av att reflektera kring problem över en längre tid. Detta gör det möjligt att få en djupare förståelse för något än om man löser ett problem under en lektion.}

\textcolor{lila}{
I enkäten uttrycktes att det är ett problem att hitta uppgifter, samt att hinna planera problemlösningslektioner. Niklas uttryckte även i sin intervju \ref{sec:intervju} att han saknade mer hjälp på \textsl{hur} problemlösning ska läras ut. }
\textcolor{Mahogany}{Utifrån detta känns det som ett utmärkt komplement till matematikboken att kunna ta del av kompletta lektionsplaneringar med problemlösning. I samband med problemen har vi därför valt att vara så utförliga som möjligt med instruktioner och förslag på utförande samt innehåll. Samtidigt har vi försökt understryka att de ska känna att de får styra innehållet utefter vad de anser passar bäst för sina egna klasser.}
\textcolor{lila}{Problemet ''Försvåra en ekvation'' skapades också för att kännas mest som en lek, och därmed kunna användas i en klass som inte sedan tidigare är vana vid att arbetamed problemlösning.}


        
    \subsection{Alternativa sätt att testa problem}
        \textcolor{green}{Metoden för att testa de problem som tagits fram i detta arbete var att låta matematiklärare testa uppgifterna på sina elever. Tanken med detta var att uppgifterna skulle testas i stor skala på många gymnasieskolor samtidigt. På så sätt samlades det in stora kvantiteter av data på ett enkelt sätt. Dock så hade denna metod några större nackdelar.}

\textcolor{green}{När lärarna testade uppgifterna var tanken att de skulle utgå ifrån uppgifternas beskrivningar, men att de sedan hade relativt fria tyglar att förmedla uppgifterna på det sätt de fann lämpliga. Ett dilemma med denna metod var att det inte fanns någon möjlighet för oss att direkt se hur lärarna genomförde lektionen. Den data vi samlade in via enkäterna gav oss elevernas och lärarnas individuella bedömningar. Trots stor kvantitet blev det svårt att verifiera hur kvalitativ datan var.}

\textcolor{green}{Ett alternativ till att be en lärare att prova uppgifterna på sin klass hade varit att vi själva hade utfört testerna på gymnasielever. En stor skillnad hade varit att vi hade kunnat testa uppgifterna på precis samma sätt med många olika klasser. Datan vi sedan hade samlat in hade varit mer kvalitativ, eftersom vi hade varit konsekventa med tillvägagångssättet vi testade uppgifterna på.}
    \textcolor{WildStrawberry}{
    Om vi tagit en mer aktiv ställning till testandet kunde detta varit möjligt, men i brist på tid gav vi lärarna friheten att testa problemen på det vis de anser bäst själva.} \textcolor{lila}{Fördelen med detta är dock att det bättre speglar hur våra problem ska användas, det vill säga som en guide för att hjälpa lärare  planera och genomföra lektioner med problemlösning.}


%
% * Vi använde denna metoden
% * Den data som sedan samlades in på hur testerna hade gått samlades in via enkäter. Hur lärarna gjorde för att ta reda på vad eleverna tyckte kan ha varierat. Vissa lärare kan ha frågat eleverna medan andra kan ha skrivit ner vad de själva tror att eleverna tyckte
% 
% det fanns inget sätt för oss att se hur läraren genomförda lektionen
%
% 
%
        
    \subsection{Hemsida}
        % Just att ha en plats där vem som helst kan ha åtkomst till problemen anser vi är viktigt. När möjligheten finns att kunna dela information på ett lättillgängligt sätt, då når man potentiellt ut till fler lärare som skulle vara intresserade utav att testa problemen.

% Motargumentet till att bygga en hemsida är att vi skulle kunna skicka ut PDF-filer per epost till lärare - vilket var fallet när vi testade problemen. Men potentialen för spridning blir mycket enklare med möjligheten att klistra in en länk i ett mail, chattrutor eller sociala nätverk än att först ladda ner en fil och sedan bifoga den. Tanken är att ju fler handlingar som finns i en interaktion, ju större chans är det att man missar engagemangen att dela med sig till vänner och kollegor som potentiellt skulle vilja testa.
        
    \subsection{Hur vi kan påverka}
        % \textcolor{Mahogany}{
%     Som nämndes i avsnitt~\ref{sec:slutenkat} så var det under utvärderingen inte ovanligt att lärare kände att de inte hade tid att testa våra problem på grund av annat i kursplanen så som nationella prov. De var dock intresserade, och de flestakommenterade att problemen och uppläggen verkade bra. Vi hoppas därför att även om många lärare just nu inte kan utvärdera våra problem att vi ändå lyckats förmedla nyttan med denna typ av arbetssätt och att matematik bör handla mer om den undersökande delen som till större del bygger på diskussion och rimlighetsanalyser.
% }

\textcolor{Mahogany}{
    I \ref{sec:MerProblemlosning} så framgår det att många lärare påpekar att de anser problemlösning vara viktigt. Samtidigt känner de att de inte kan arbeta med det till den grad de skulle vilja på grund av tidsbrist, både på och inför lektionerna, och för att de inte vet hur man skulle göra det på ett bra sätt. De tyckte bland annat att det var svårt att hitta bra problem som är utmanande för hela klassen. %I utformandet av våra problem så försökte vi i någon mån att anpassa problemen efter vissa nivåer, kanske främst för testningssyften då vissa lärare uttryckte intresse av att testa problem som specifikt skulle passa in på en viss matematikkurs. Vi hade egentligen som utgångspunkt att utforma \textsl{bra} problem, utan hänsyn till vilken nivå den hamnar på. Samtidigt kan det vara bra att ta försöka få in vissa element som gör att man kan fånga upp relevant teori, eller efterfrågan av den, även om fokus fortfarande ligger på problemlösning.
}

    \textcolor{lila}{Med den fullständiga lektionsplanering som följer med hoppas vi kunna avlasta lärarna från en del av det förberedande arbete som krävs för att genomföra en problemlösningslektion, och därmed förbättra deras arbetssituation. Med den brist på lärare som råder idag är det viktigt ur ett samhällsperspektiv att spara på de resurser vi har, för att kunna utnyttja deras fulla potential genom att de får finnas till hands och handleda problemlösningen. Bristen av ren lektionstid är svårt för oss att påverka, men dels har vi även tryckt på den bevisade nyttan av problemlösning och dels hoppas vi som sagt att detta ska hjälpa med en del av de nuvarande faktorer som motverkar mer problemlösning i undervisningen.}
    
    \textcolor{lila}{Även om man arbetar hårt för att ändra matematikundervisningen, och utvecklingen enligt oss är på väg åt rätt håll, så tror vi att det finns mer som kan göras för att hjälpa den på traven. Vi vill därför arbeta för att hjälpa lärarna att få eleverna att gå från att beskriva matematiken som ''tråkig'' och ''onödig'' och istället börja använda ord som ''rolig'', ''spännande'' och ''användar''. Kanske till och med ''kreativ''.}
    
% \textcolor{Mahogany}{
%     Vi hoppas att även om det är en småskalig inverkan så kan vi fortfarande hjälpa till att förändra bilden kring matematik så att den inte ska förknippas med ord som och att elever ska känna att det är något kreativt och undersökande snarare än att handla om memorering och upprepning.}
    %\textcolor{lila}{Och även om detta bara är en liten del av den utveckling vi hoppas följer, så hoppas vi att matematikundervisningen utvecklas till en mer levande och kreativ process. I bästa fall leder detta till en ny generation av skickliga problemlösare, som kan säkra Sveriges roll i framtidens tekniska samhälle. 
%}

\section{Slutsats}

\newpage
\begin{thebibliography}{3}
        %Skriv källa:
    %1
    \bibitem{Ignacio&Barona}
    N.G. Ignacio, L.J.B.N.a.E.G. Barona, "The Affective Domain in Mathematics Learning," i \textsl{IEJME-Mathematics Education}. [Online]. Tillgänglig: \url{http://iejme.com/makale/70}. Hämtad: 5 apr 2017.
    
    %2
    \bibitem{Skolverket03}
    Skolverket, ''Lusten att l{\"a}ra : med fokus p{\aa} matematik : nationella kvalitetsgranskningar 2001-2002'',  Skolverket, Stockholm, Skolverkets rapport, 1103-2421 ; 221, 2003
    
    %Rapportserie och nummer, volym, år. [Online]. Tillgänglig: \url{http://www.mah.se/pages/45519/lustattlara.pdf}. Hämtad: datum.
   
    % Author(s). Book title. Location: Publishing company, year, pp. 
    % http://libris.kb.se/bib/8904038?vw=full
    
    %3
    \bibitem{CompareOECD}
    “Compare your country - PISA 2015,” Compare your country by OECD. [Online]. Tillgänglig: http://www.compareyourcountry.org/pisa/country/SWE. Hämtad: Feb. 10, 2017.
    
    %4
     \bibitem{traditionellMatte}
    E. Berggren, "Traditionell skolmatematik: En studie av undervisning och lärande under en matematiklektion," examensarbete, Institutionen för datavetenskap, fysik och matematik, Linnéuniversitetet, Växjö, Sverige, 2010. [Online]. Tillgänglig: \url{http://www.diva-portal.org/smash/get/diva2:337889/FULLTEXT01.pdf}. Hämtad: 7 mars, 2017.
    
    %5
    \bibitem{TheElephant}
    J. Boaler, ''The Elephant in the Classroom: Helping Children Learn and Love Maths,'' 
    London,
    England: Souvenir Press Ltd, 
    2010. 
    
    %6
    \bibitem{Nämnaren}
    E. Silver, M. Smith, "Samtalsmiljöer," Nämnaren, nr. 1, ss. 55-59, feb. 2015. [Online]. Tillgänglig: \url{http://ncm.gu.se/pdf/namnaren/5559_15_1.pdf}. Hämtad: Apr. 4, 2017
    
    %7
    \bibitem{2016Senare}
    ''Senare matematik i gymnasieskolan (matematik 3c),'' Startsidan - Skolinspektionen. [Online]. Tillgänglig: \url{https://www.skolinspektionen.se/sv/Beslut-och-rapporter/Publikationer/Granskningsrapport/Kvalitetsgranskning/senare-matematik-i-gymnasie-skolan-matematik-3c/}. Hämtad: Feb. 10, 2017.
    
    %8
    \bibitem{2010UndervisningenGymnasieskolan}
    “Undervisningen i matematik i gymnasieskolan,” Startsidan - Skolinspektionen. [Online]. Tillgänglig: \url{https://www.skolinspektionen.se/sv/Beslut-och-rapporter/Publikationer/Granskningsrapport/Kvalitetsgranskning/------Undervisningen-i-matematik-i-gymnasieskolan/}. Hämtad: Feb. 10, 2017.
    
    %9
    \bibitem{lockhart}
    P. Lockhart, \textit{A mathematician's lament}. New York: Bellevue Literary Press, 2009.
    
    %10
    \bibitem{GY00-GY11}
    Skolverket, ''Jämförelse med kursplan 2000,'' 2011. [Online]. Tillgänglig: \url{https://www.skolverket.se/laroplaner-amnen-och-kurser/gymnasieutbildning/gymnasieskola/mat}. Hämtad: Apr. 11, 2017.
    
    %11
    \bibitem{regeringen}
    Regeringskansliet, ''Stärkt digital kompetens i läroplaner och kursplaner,'' \textsl{regeringen.se} 2017. [Online]. Tillgänglig: \url{http://www.regeringen.se/pressmeddelanden/2017/03/starkt-digital-kompetens-i-laroplaner-och-kursplaner/}. Hämtad: Apr. 11, 2017.
    
    %12
    \bibitem{itiskolan}
    ?
    M. Halápi, K.L. Rüter, ''Redovisning av uppdraget om att föreslå nationella IT-strategier för skolväsendet'', Skolverket,
    
    \url{https://www.skolverket.se/om-skolverket/publikationer/visa-enskild-publikation?_xurl_=http\%3A\%2F\%2Fwww5.skolverket.se\%2Fwtpub\%2Fws\%2Fskolbok\%2Fwpubext\%2Ftrycksak\%2FBlob\%2Fpdf3647.pdf\%3Fk\%3D3647}
    
   %13
   \bibitem{prog_utbildning}
   Skolverket, ''Tydligare om digital kompetens i läroplaner, kursplaner och ämnesplaner'', 2017. [Online]. Tillgänglig: \url{https://www.skolverket.se/skolutveckling/resurser-for-larande/itiskolan/styrdokument}. Hämtad: Apr. 11, 2017.
   
   %14
   \bibitem{RikaProblem}
    Hagland, K., Hedrén, R. and Taflin, E., \textit{Rika matematiska problem - inspiration till variation}, Malmö, Sverige, Elanders Berlings AB, 2005.
\end{thebibliography}
\end{document}