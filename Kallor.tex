    %Skriv källa:
    %1
    \bibitem{Ignacio&Barona}
    N.G. Ignacio, L.J.B.N.a.E.G. Barona, "The Affective Domain in Mathematics Learning," i \textsl{IEJME-Mathematics Education}. [Online]. Tillgänglig: \url{http://iejme.com/makale/70}. Hämtad: 5 apr 2017.
    
    %2
    \bibitem{Skolverket03}
    Skolverket, ''Lusten att l{\"a}ra : med fokus p{\aa} matematik : nationella kvalitetsgranskningar 2001-2002'',  Skolverket, Stockholm, Skolverkets rapport, 1103-2421 ; 221, 2003
    
    %Rapportserie och nummer, volym, år. [Online]. Tillgänglig: \url{http://www.mah.se/pages/45519/lustattlara.pdf}. Hämtad: datum.
   
    % Author(s). Book title. Location: Publishing company, year, pp. 
    % http://libris.kb.se/bib/8904038?vw=full
    
    %3
    \bibitem{CompareOECD}
    “Compare your country - PISA 2015,” Compare your country by OECD. [Online]. Tillgänglig: http://www.compareyourcountry.org/pisa/country/SWE. Hämtad: Feb. 10, 2017.
    
    %4
     \bibitem{traditionellMatte}
    E. Berggren, "Traditionell skolmatematik: En studie av undervisning och lärande under en matematiklektion," examensarbete, Institutionen för datavetenskap, fysik och matematik, Linnéuniversitetet, Växjö, Sverige, 2010. [Online]. Tillgänglig: \url{http://www.diva-portal.org/smash/get/diva2:337889/FULLTEXT01.pdf}. Hämtad: 7 mars, 2017.
    
    %5
    \bibitem{TheElephant}
    J. Boaler, ''The Elephant in the Classroom: Helping Children Learn and Love Maths,'' 
    London,
    England: Souvenir Press Ltd, 
    2010. 
    
    %6
    \bibitem{Namnaren}
    E. Silver, M. Smith, ''Samtalsmiljöer,'' Nämnaren, nr. 1, ss. 55-59, feb. 2015. [Online]. Tillgänglig: \url{http://ncm.gu.se/pdf/namnaren/5559_15_1.pdf}. Hämtad: 4 Apr., 2017
    
    \bibitem{80-talet}
    K. Johansson, ''Problemlösning i matematik på gymnasiet,'' examensarbete, Pedagogutbildningarna, Luleå Tekniska Universitet, Luleå, Sverige, 2003. [Online]. Tillgänglig: \url{http://www.diva-portal.org/smash/get/diva2:1016004/FULLTEXT01.pdf}. Hämtad 5 Apr., 2017.
    
    \bibitem{mattelyftet}
    Skolverket,''Matematiklyftet,'' \textsl{Skolverket.se}, 2017. [Online]. Tillgänglig: \url{https://www.skolverket.se/kompetens-och-fortbildning/larare/matematiklyftet}. Hämtad: 1 maj, 2017.
    
    %7
    \bibitem{2016Senare}
    ''Senare matematik i gymnasieskolan (matematik 3c),'' Startsidan - Skolinspektionen. [Online]. Tillgänglig: \url{https://www.skolinspektionen.se/sv/Beslut-och-rapporter/Publikationer/Granskningsrapport/Kvalitetsgranskning/senare-matematik-i-gymnasie-skolan-matematik-3c/}. Hämtad: 10 Feb, 2017.
    
    %8
    \bibitem{2010UndervisningenGymnasieskolan}
    “Undervisningen i matematik i gymnasieskolan,” Startsidan - Skolinspektionen. [Online]. Tillgänglig: \url{https://www.skolinspektionen.se/sv/Beslut-och-rapporter/Publikationer/Granskningsrapport/Kvalitetsgranskning/------Undervisningen-i-matematik-i-gymnasieskolan/}. Hämtad: 10 Feb, 2017.
    
    %9
    \bibitem{lockhart}
    P. Lockhart, ''A mathematician's lament''. New York: Bellevue Literary Press, 2009.
    
    %10
    \bibitem{GY00-GY11}
    Skolverket, ''Jämförelse med kursplan 2000,'' 2011. [Online]. Tillgänglig: \url{https://www.skolverket.se/laroplaner-amnen-och-kurser/gymnasieutbildning/gymnasieskola/mat}. Hämtad: Apr. 11, 2017.
    
    %11
    \bibitem{regeringen}
    Regeringskansliet, ''Stärkt digital kompetens i läroplaner och kursplaner,'' \textsl{regeringen.se} 2017. [Online]. Tillgänglig: \url{http://www.regeringen.se/pressmeddelanden/2017/03/starkt-digital-kompetens-i-laroplaner-och-kursplaner/}. Hämtad: Apr. 11, 2017.
    
    %12
    \bibitem{itiskolan}
    ?
    M. Halápi, K.L. Rüter, ''Redovisning av uppdraget om att föreslå nationella IT-strategier för skolväsendet'', Skolverket,
    
    \url{https://www.skolverket.se/om-skolverket/publikationer/visa-enskild-publikation?_xurl_=http\%3A\%2F\%2Fwww5.skolverket.se\%2Fwtpub\%2Fws\%2Fskolbok\%2Fwpubext\%2Ftrycksak\%2FBlob\%2Fpdf3647.pdf\%3Fk\%3D3647}
    
   %13
   \bibitem{prog_utbildning}
   Skolverket, ''Tydligare om digital kompetens i läroplaner, kursplaner och ämnesplaner'', 2017. [Online]. Tillgänglig: \url{https://www.skolverket.se/skolutveckling/resurser-for-larande/itiskolan/styrdokument}. Hämtad: Apr. 11, 2017.
   
   %14
   \bibitem{RikaProblem}
    Hagland, K., Hedrén, R. and Taflin, E., \textit{Rika matematiska problem - inspiration till variation}, Malmö, Sverige, Elanders Berlings AB, 2005.
    
    %15
    \bibitem{Luhn}
    “modulus 10 - Uppslagsverk - NE.” [Online]. Tillgänglig: \url{http://www.ne.se/uppslagsverk/encyklopedi/lång/modulus-10}. 
    Hämtad: Apr. 25, 2017.
    
    %16
    \bibitem{matmod}
    “Course PM.” [Online]. Tillgänglig: \url{http://www.cse.chalmers.se/edu/year/2010/course/DAT026/CoursePM/index.html}. 
    Hämtad: Apr. 27, 2017.
    
    %17
    \bibitem{matte5000}
    Alfredsson L, Bråting, K, Erixon P, Heikne H,, \textit{Matematik 5000 - Kurs 1c Blå lärobok}, Stockholm, Sverige, Natur \& Kultur, 2011.
     
    %18
    \bibitem{sarskildautbildningsbehov}
    K. Eriksson, “Elever i särskilda utbildningsbehov och problemlösning. : En studie av elevers upplevelse och deltagande vid problemlösning i grupp.,” 2016.
     
    %20
    \bibitem{djupareKunskapPBL}
     H. S. Barrows, “Problem-Based, Self-directed Learning,” JAMA J. Am. Med. Assoc., vol. 250, no. 22, pp. 3077–3080, 1983.
      
    %21
    \bibitem{PBLdefinition}
    W. Hung, D. H. Jonassen, and R. Liu, “Problem-Based Learning,” Handb. Res. Educ. Commun. Technol., pp. 485–506, 2008.
    
    %23
    \bibitem{deduktivInlärning}
    D. Wedelin and T. Adawi, “Teaching Mathematical Modelling and Problem Solving - A Cognitive Apprenticeship Approach to Mathematics and Engineering Education,” International Journal of Engineering Pedagogy (iJEP), vol. 4, no. 5, p. 49, Mar. 2014.
    
    %24
    \bibitem{undervisningviaproblemlosning}
    K. Fredriksson, “Matematikundervisning via problemlösning : En litteraturstudie om lärandefaktorer som kan påverkas av matematikundervisning via problemlösning,” 2014.
    
    %25
    \bibitem{Polya}
    G. Polya, ''How to Solve It : a New Aspect of Mathematical Method. Princeton University Press'', 1948.
    
    %26
    \bibitem{ProblemDef}
    Skolverket, ''Om ämnet Matematik'' 2011. [Online]. Tillgänglig: \url{https://webcache.googleusercontent.com/search?q=cache:u6xMQy4vk8kJ:https://www.skolverket.se/laroplaner-amnen-och-kurser/gymnasieutbildning/gymnasieskola/mat/comment.pdf\%3FsubjectCode\%3DMAT\%26commentCode\%3DALL\%26lang\%3Dsv+&cd=3&hl=sv&ct=clnk&gl=se}. Hämtad: Maj 04, 2017.
    
    %27
    \bibitem{problemVarierandeDef} %OBS! behöver hjälp att fixa till korrekt format
    http://www.mai.liu.se/SMDF/madif8/Bergqvist\%26Bergqvist.pdf
    
    %28
    \bibitem{olikaDefinitioner} %OBS! behöver hjälp att fixa till korrekt format
    http://www.artandscience.se/Matematisk\%20probleml\%F6sning.pdf
    
    %29
    \bibitem{SverigeKarta}
    Pixabay, Clker-Free-Vector-Images, Sverige karta land Europa. 2012 [Online]. Tillgänglig: \url{https://pixabay.com/sv/sverige-karta-land-europa-23576/:https://pixabay.com/sv/sverige-karta-land-europa-23576/} Hämtad: Maj 08, 2017.
    
    %30
    \bibitem{Carlsson}
    H. Carlsson, “Problembaserat lärande i de tidiga skolåren,” 2007.
    
    %31
    \bibitem{Johansson}
    Johansson, “Problembaserat lärande – en elevaktiv arbetsmodell för grundskolans tidigare år?,” 2009.
    
    %32
    \bibitem{WhyDontStudents}
    D. Willingham, ''Why Don't Students Like School? - A cognitive scientist answers questions about how the mind works and what it means for your classroom'' 
    San Fransisco,
    Jossey-Bass, 
    2009.
    
    %33
    \bibitem{pi}
    “pi - Uppslagsverk - NE.” [Online]. Tillgänglig: http://www.ne.se/uppslagsverk/encyklopedi/lång/pi. Hämtad: Maj 08, 2017.
    
    \bibitem{pisaImedia}
    E. Leijnse, “Så har det gått för Sverige i Pisa”, december 2016. [Online], \url{http://www.sydsvenskan.se/2016-12-04/sa-har-det-gatt-for-sverige-i-pisa}. 
    [Hämtad: Maj 10, 2017]
    
    \bibitem{spriddKunskap}
    H. Thunberg, et al. ”Gymnasiets mål och högskolans förväntningar”, Nämnaren, no. 2, pp. 10-15, 2006.