\subsection{Bör läggas någonstans i diskussionen}

\textcolor{WildStrawberry}{
    Som grupp har vi alla något gemensamt. Vi har alla relativt nyligen genomgått en gymnasieutbildning och tyckes alla komma ihåg att det existerade problem med motivationen på en mängd elever i våra klasser. Internt inom vår grupp så blir det ett stickprov på 5 gymnasieklasser där alla genomgående känner samma sak. Detta stickprov är ju inte alls egentligen något att komma med, det är alldeles för litet för att kunna härleda någon ordentlig slutsats. Men om motivationen på elever i matematiken inte skulle vara ett vanligt problem i gymnasiet skulle vi troligtvis i alla fall haft en person internt som kunde hävda det. Denna observation är dels underlag för vidare undersökning. Gymnasietskolan i Sverige har nya betygssytem och kriterier som lägger en del vikt på problemlösning i matematiken \ref{sec:Forandringar}. Därav kommer syftet med problemen som utformas i detta arbete. Problem som ska öva elever på förmågan att bemöta uppställningar dem inte har svaret på sedan innan och förhoppningsvis medföra större motivation än beräkningar från boken.
}

% exempeltext på syfte...
\textcolor{WildStrawberry}{
    Vi har en bild av att det existerar en svårighet i den svenska matematikundervisningen. Hypotesen lyder att bekvämligheten att hålla fast vid gamla metoder är mer närvarande än föreslagen kursplan vill få ut från undervisningen. Detta projekt har för syfte att testa detta och utforma en mängd problem som skulle uppfylla det syfte som kursplanen vill uppnå. }