\subsection{Bör läggas någonstans i diskussionen}

\textcolor{WildStrawberry}{
    Som grupp har vi alla något gemensamt. Vi har alla relativt nyligen genomgått en gymnasieutbildning och tyckes alla komma ihåg att det existerade problem med motivationen på en mängd elever i våra klasser. Internt inom vår grupp så blir det ett stickprov på 5 gymnasieklasser där alla genomgående känner samma sak. Detta stickprov är ju inte alls egentligen något att komma med, det är alldeles för litet för att kunna härleda någon ordentlig slutsats. Men om motivationen på elever i matematiken inte skulle vara ett vanligt problem i gymnasiet skulle vi troligtvis i alla fall haft en person internt som kunde hävda det. Denna observation är dels underlag för vidare undersökning. Gymnasietskolan i Sverige har nya betygssytem och kriterier som lägger en del vikt på problemlösning i matematiken \ref{sec:Forandringar}. Därav kommer syftet med problemen som utformas i detta arbete. Problem som ska öva elever på förmågan att bemöta problemställningar de inte har svaret på sedan innan och förhoppningsvis medföra större motivation än beräkningar från boken.
}

% exempeltext på syfte...
\textcolor{WildStrawberry}{
    Vi har en bild av att det existerar en svårighet i den svenska matematikundervisningen. Hypotesen lyder att bekvämligheten att hålla fast vid gamla metoder är mer närvarande än föreslagen kursplan vill få ut från undervisningen. Detta projekt har för syfte att testa detta och utforma en mängd problem som skulle uppfylla det syfte som kursplanen vill uppnå. }
\\ \\
    \textcolor{WildStrawberry}{
    Nedan finns uppgifter tagna från  kapitlet \textit{''problemlösning''} i en matematikbok ämnad för matematik 1c \cite{matte5000}. Boken trycktes 2011, alltså är uppgifterna designade utefter förändringarna som ska vara aktuella med GY11. Det krävs inte mycket tid för att hitta all information man behöver för att kunna lösa uppgifterna och när man väl hittat allt man behöver så ställer man upp en ''lös ut x'' uppgift.}

% Det är nog också mer relevant till "matematiken idag". Det existerar en bekvämlighetsfaktor just på grund av tiden är bristande och därför är det najs att använda sig av färdiga problem som inte har mycket tanke bakom sig. - Eleven får övning och "problemlösning (läsförståelse)"

%Här kommer några bra dåliga problemlösningsuppgifter ifrån \cite{matte5000} - MVH Björn

%Följande är en a uppgift, dvs en lätt uppgift:
\begin{displayquote}
\textcolor{turkos}{Marcus läser en bok som innehåller 420 sidor. Mellan kl 19.45 och 20.15 läser han 14 sidor. \\
Hur lång tid tar det att läsa hela boken?}
\end{displayquote}

%Svar

%Detta är en b-uppgift
\begin{displayquote}
\textcolor{turkos}{Jonas kör sin bil samma sträcka varje dag. Sträckan är en mil och Jonas brukar köra med hastigheten 90 $km/h$ en dag kör han sträckan med 100 $km/h$. \\
Hur många sekunder ''tjänar'' Jonas på det?}
\end{displayquote}

%Svar

%Följande två uppgifter är c-uppgifter, dvs de svåraste. 

\begin{displayquote}
\textcolor{turkos}{Vilket tal är x?\\
\( 2*5^x + 3*5^x = 25^{12} \)}
\end{displayquote}

%Svar

\begin{displayquote}
\textcolor{turkos}{En sandstrand är 2km lång, 30 m bred och 3 m djup. \\
Vi antar att ett sandkort ryms inom ett kubiskt område med sidan 0,2 mm.\\
Hur många sandkorn finns på stranden?}
\end{displayquote}

%Svar

%Samtliga fyra uppgifter har tagits från delkapitel 1.4 Problemlösning, som är del av 1 Aritmetik - Om tal. Finns liknande uppgifter i kapitlen om 2.2 Procentuellea förändringar och 3.2 Linjära ekvationer och olikheter. Dock saknas helt uppgifter om problemlösning för Geometri, Sannolikhetslära och statistik, samt Grafer och funktioner.