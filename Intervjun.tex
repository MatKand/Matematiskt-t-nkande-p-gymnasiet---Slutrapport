%Vem är Niklas
\textcolor{turkos}{
För att få ett perspektivet ifrån någon som faktiskt undervisar matematik och höra den personens åsikter om problemlösning kontaktade vi Niklas Grip. Niklas är ämneslärare i matematik på Mikael Elias teoretiska gymnasium i Göteborg och har bland annat kursen Matematik – specialisering som han inriktat mot just problemlösning.
}

% Hur genomfördes intervjun, samt våra frågor till honom. 
\textcolor{turkos}{
Intervjun genomfördes som ett djupgående samtal mellan oss och Niklas, där Niklas gavs stort utrymme att svara fritt. Samtalet var uppstyr kring följande fyra frågor, samt följdfrågor på dessa: 
}
\begin{itemize}
  \item \textcolor{turkos}{Arbetar du med problemlösning i din undervisning?}
  \item \textcolor{turkos}{Hur definierar du problemlösning?}
  \item \textcolor{turkos}{Kan du ge några exempel på problem du använt i din undervisning?}
  \item \textcolor{turkos}{Hur arbetar du med teknik i din matematikundervisning?}
\end{itemize}

\noindent \textcolor{turkos}{
Följande är sammanfattning utav intervjun där vi försöker komprimera det det viktigaste som Niklas sa. De citat som tas upp kommer beskriva det som vi tyckte var av särskilt intresse eller vikt under intervju. 
}

% Hur undervisar han problemlösning? 

%Hur definerar han problemlösning?
% ''...man får något sorts problem som metoden att lösa problemet är inte på förhand givet på något sätt.'' - Niklas

%Problem med problemlösning?

%Hur jobbar han med teknik och tekniska hjälpmedel

