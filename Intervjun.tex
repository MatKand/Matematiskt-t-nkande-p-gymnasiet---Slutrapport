%Vem är Niklas
\textcolor{turkos}{
För att få ett perspektivet ifrån någon som faktiskt undervisar matematik och höra den personens åsikter om problemlösning kontaktade vi Niklas Grip. Niklas är ämneslärare i matematik på Mikael Elias teoretiska gymnasium i Göteborg och har bland annat kursen Matematik – specialisering som han inriktat mot just problemlösning.
}

% Hur genomfördes intervjun, samt våra frågor till honom. 
\textcolor{turkos}{
Intervjun genomfördes som ett djupgående samtal mellan oss och Niklas, där Niklas gavs stort utrymme att svara fritt. Samtalet var uppstyr kring följande fyra frågor, samt följdfrågor på dessa: 
}
\begin{itemize}
  \item \textcolor{turkos}{Arbetar du med problemlösning i din undervisning?}
  \item \textcolor{turkos}{Hur definierar du problemlösning?}
  \item \textcolor{turkos}{Kan du ge några exempel på problem du använt i din undervisning?}
  \item \textcolor{turkos}{Hur arbetar du med teknik i din matematikundervisning?}
\end{itemize}

\noindent \textcolor{turkos}{
Följande är sammanfattning utav intervjun där vi försöker komprimera det det viktigaste som Niklas sa. De citat som tas upp kommer beskriva det som vi tyckte var av särskilt intresse eller vikt under intervju. 
}

\subsubsection{Niklas syn på problemlösning}

% Hur undervisar han problemlösning? 
\textcolor{turkos}{
Niklas arbetar med problemlösning på flera olika sätt. Dels så använder han problemlösning som en metod för att introducera nya begrepp för sina elever, men han har även lektioner helt inriktade på problemlösning och öppna frågeställningar. 
}

%Problem med problemlösning?
\textcolor{turkos}{
Han ser tre svårigheter med att sitt arbete med problemlösning. Det första är att han har många högpresterande elever som har en väldigt klar bild utav vad matematik är och som ofta blir negativt inställda till lektioner som går utanför deras bild av vad en matematiklektion ska innehålla. Det andra är att de nationella proven i framför allt matematik 2 till 4 inte testar problemlösning, vilket leder till att både Niklas och hans elever tappar lite av motivationen att jobba med problemlösning.}

\textcolor{turkos}{Det sista är det rent pedagogiska i hur man ska undervisa om problemlösning. Niklas upplever att överallt så talas det gott om problemlösning, men att finns lite hjälp att få ifrån andra lärare eller andra personer när det gäller saker som hur man ska göra när en elev fastnar i ett problem. Han efterfrågar en undervisningskultur runt problemlösning där han kan diskutera sina erfarenheter och svårigheter med andra lärare: 
}

% Indrag på marginalerna, mindre text. Inget citattecken, beskriv i texten. Skriv efteråt hur vi tolkar det. 
\begin{displayquote}
\textcolor{turkos}{Det är väl kanske de här sakerna som jag sa förut att det finns en lite väldigt svag kultur och med det också goda exempel på hur man undervisar om just själva problemlösande. Det finns både forskning och litteratur om det, och det har funnits länge.}
\end{displayquote}

\textcolor{turkos}{
Niklas upplever alltså att trots tillgång till mycket resurser så känner han hans arbete med problemlösning bedrivs mycket på egen hand. När det gäller undervisning utav andra delar av matematiken så har han hjälp utav ifrån böcker, andra lärare och youtube-kanaler som visar hur man löser uppgifter. Hand upplever däremot att det är väldigt få som inriktar sig på den generella förmågan att lösa problem. 
}

%Hur definerar han problemlösning?
\textcolor{turkos}{
Niklas definierar själv problemlösning väldigt brett, för honom kan allt vara problemlösning. Huruvida en uppgift är ett problem eller ej beror på kunskapen hos personen som försöker lösa det:
}

\begin{displayquote}
\textcolor{turkos}{
...man får något sorts problem som metoden att lösa problemet är inte på förhand givet på något sätt.
}
\end{displayquote}

\textcolor{turkos}{
Det är den idéen som Niklas använder som grund när han använder problemlösning för att introducera nya begrepp. Kan en elev inte lösa andragradsekvationer så det utmärkt tillfälle att både lära sig ett nytt begrepp och öva upp sin problemlösningsförmåga. 
}

%Exempel på problem han har använt. 

%Hur jobbar han med teknik och tekniska hjälpmedel


\subsubsection{Niklas kommentarer på våra problem}

