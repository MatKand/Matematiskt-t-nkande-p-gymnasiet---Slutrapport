\noindent \textcolor{WildStrawberry}{
    Mathematics is a core subject in the Swedish school system. Parallel to this, it is a common occurence that the subject is perceived as hard, boring and useless. With this project we want to faciliate the teachers by introducing new mathematical problems targeted for upper secondary school pupils. We conducted a survey for teachers; where 91\% asserted that they are working to include problem-solving in their teachings, while 42\% have issues finding good problems. The main purpose of this project has then been to produce problems designed to help the teachers in question. The problems are open, applicable to real-life, give ground for discussion and, perhaps most importantly, make mathematics feel relevant. Also, since programming is soon going to be part of mathematics, a share of our problems have been constructed to prepare students for this inclusion. Additionally to these problems, a suggested ''complete'' lesson plan is provided and presented on an easily accessible and user-friendly web page. Finally, from all our tested problems, the 26 student answers assert that: 80\% declare they learnt something new and that 75\% state they found the problem fun or meaningful in some way.
    }
    
    %This project aims to change this by introducing \textit{relevant} and easily accessible problem-solving tasks for upper secondary school pupils.  Thus the main purpose
    
    %These problems are designed to spark initiative and allow for creative solutions. Analogous to the real-world, the problems should encourage discussion whether they can be solved in multiple ways or if solutions can be refined to better model, but perhaps most of all: feel useful. Also, since programming is going to be introduced in school mathematics, some problems are made to test and prepare students for the inclusion of programming. }
    
%\textcolor{WildStrawberry}{
    %As a part of the result from the project, 58 teachers took part of our survey.     After testing our problems, some of the pupils answered a questionnaire regarding their experience when approaching the problems. From the 26 answers: 80\% declared that they learnt something new and 75\% stated that they found the problem fun or meaningful in some way.}
    
%\textcolor{WildStrawberry}{
    %Our conclusion is that teachers want to include problems-solving in their teachings, however finding well made problems isn't an easy task and that they lack the time to craft problems themselves. The reach we surmounted to wasn't huge, but the teachers who did try our problems seemed content.}