%Vem är Niklas
\textcolor{turkos}{
För att få ett perspektiv ifrån någon som faktiskt undervisar i matematik och höra den personens åsikter om problemlösning kontaktade vi Niklas Grip. Niklas är ämneslärare i matematik på Mikael Elias teoretiska gymnasium i Göteborg och har bland annat kursen Matematik – specialisering som han inriktat mot just problemlösning.
}

% Hur genomfördes intervjun, samt våra frågor till honom. 
\textcolor{turkos}{
Intervjun genomfördes som ett djupgående samtal mellan oss och Niklas, där Niklas gavs stort utrymme att svara fritt. Samtalet var uppstyrt kring följande fyra frågor, samt följdfrågor på dessa: 
}
\begin{itemize}
  \item \textcolor{turkos}{Arbetar du med problemlösning i din undervisning?}
  \item \textcolor{turkos}{Hur definierar du problemlösning?}
  \item \textcolor{turkos}{Kan du ge några exempel på problem du använt i din undervisning?}
  \item \textcolor{turkos}{Hur arbetar du med teknik i din matematikundervisning?}
\end{itemize}

\noindent \textcolor{turkos}{
Följande är sammanfattning utav intervjun där vi försöker komprimera det det viktigaste som Niklas sa. De citat som tas upp kommer beskriva det som vi tyckte var av särskilt intresse eller vikt under intervjun. 
}

\subsubsection{Niklas syn på problemlösning}
\label{sec:Niklassyn}

% Hur undervisar han problemlösning? och %Exempel på problem han har använt. 
\textcolor{turkos}{
Niklas arbetar med problemlösning på flera olika sätt. Dels så använder han problemlösning som en metod för att introducera nya begrepp för sina elever, men han har även lektioner helt inriktade på problemlösning och öppna frågeställningar. Han berättar att han inte får in lika mycket öppna problem som han skulle önska. Som ett exempel på vad ett öppet problem är tar Niklas upp att han bett sina elever räkna ut hur stor sannolikheten är att bli träffad utav en fågelskit under ett liv. Andra problem han jobbar med är klassiska optimeringsproblem, så som att dra en kabel över en flod. Till sist så har han hand om gymnasiearbeten som inriktar sig mot problemlösning, några av hans elever räknade ut vilken ''starter pokémon'' som var bäst i ett pokémonspel. Själv tycker han att han lyckas få in hela skalan av problemlösning i sin undervisning.
}

%Problem med problemlösning?
\textcolor{turkos}{
Niklas ser tre svårigheter med sitt arbete med problemlösning. Den första är att han har många högpresterande elever som har en väldigt klar bild utav vad matematik är och som ofta blir negativt inställda till lektioner som går utanför deras bild av vad en matematiklektion ska innehålla. Den andra är att de nationella proven i framför allt matematik 2 till 4 inte testar problemlösning, vilket leder till att både Niklas och hans elever tappar lite av motivationen att jobba med problemlösning.}

\textcolor{turkos}{Den tredje svårigheten som Niklas nämner är det rent pedagogiska i hur man ska undervisa om problemlösning. Niklas upplever att överallt så talas det gott om problemlösning, men att det finns lite hjälp att få ifrån andra lärare eller andra personer när det gäller saker som hur man ska göra när en elev fastnar i ett problem. Han efterfrågar en undervisningskultur runt problemlösning där han kan diskutera sina erfarenheter och svårigheter med andra lärare: 
}

% Indrag på marginalerna, mindre text. Inget citattecken, beskriv i texten. Skriv efteråt hur vi tolkar det. 
\begin{displayquote}
\textcolor{turkos}{Det är väl kanske de här sakerna som jag sa förut att det finns en lite väldigt svag kultur och med det också goda exempel på hur man undervisar om just själva problemlösande. Det finns både forskning och litteratur om det, och det har funnits länge.}
\end{displayquote}

\noindent\textcolor{turkos}{
Niklas upplever alltså att trots tillgång till mycket resurser så känner han att hans arbete med problemlösning bedrivs mycket på egen hand. När det gäller undervisning utav andra delar av matematiken så har han hjälp utav böcker, andra lärare och YouTube-kanaler som visar hur man löser uppgifter. Han upplever däremot att det är väldigt få som inriktar sig på den generella förmågan att lösa problem. Niklas tar upp att han försökt lära sina elever att arbeta enligt Pólyas metoder (se \ref{sec:polya}). Han känner dock att han misslyckas få dem att förstå dem och inse vad som är relevant med dem, där hade andra lärares tankar kunnat vara till stor hjälp för honom.
}

%Hur definerar han problemlösning?
\textcolor{turkos}{
Niklas definierar själv problemlösning väldigt brett, för honom kan allt vara problemlösning. Huruvida en uppgift är ett problem eller ej beror på kunskapen hos personen som försöker lösa det:
}

\begin{displayquote}
\textcolor{turkos}{
Kommer man till problem som man inte riktigt vet hur man ska lösa så upplever jag att det är då man använder sin problemlösningsförmåga.
}
\end{displayquote}

\textcolor{turkos}{
Det är viktigt att man förstår var elevernas kunskap ligger. Niklas berättade att han ibland har haft lärarpraktikanter som varit med på hans lektioner där han använt problemlösning som en metod för att lära sina elever hur de ska använda enhetscirkeln. Praktikanterna förstod inte hur det kunde vara problemlösning när Niklas hade haft genomgång om enhetscirkeln dagen innan, men i och med att eleven själv fick ta reda på hur de skulle använda sig av den så blev det ändå problemlösning i slutändan. 
%Det är den idéen som Niklas använder som grund när han använder problemlösning för att introducera nya begrepp. Kan en elev inte lösa andragradsekvationer så det utmärkt tillfälle att både lära sig ett nytt begrepp och öva upp sin problemlösningsförmåga. 
}

%Hur jobbar han med teknik och tekniska hjälpmedel
\textcolor{turkos}{
Niklas har introducerat sina elever till Geogebra (ett interaktivt verktyg för matematik), kalkylark och programmering, och han ser hur verktygen hjälper eleverna att förstå vad olika uppgifter handlar om. Han förklarar att elever som använder sig utav Geogebra kommer få en större överblick över problemet som de försöker lösa, och där igenom ha lättare att förstå det än vad en elev som endast använder sig utav algebra skulle göra. Niklas trycker på att det han vill är att eleverna själva ska använda sig av verktygen utan att han behöver visa dem hur de ska göra. Det är när eleverna själva får sitta med verktygen som de verkligen börjar lära sig teknikerna.
} 

%\subsubsection{Niklas kommentarer på våra problem}