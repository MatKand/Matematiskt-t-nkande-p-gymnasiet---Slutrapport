\textcolor{lila}{Nedan presenteras svaren på de utvärderingsenkäter vi fick in. De uppgifter som testades var ''Fermiproblem'', ''Försvåring av ekvation'' och ''Matematisk modell för bil och löpare'', vilka förklaras närmare i avsnitt~\ref{sec:Fermi}, \ref{sec:ekvation} respektive \ref{sec:lopare}. Niklas grip, som presenterades i avsnitt~\ref{sec:intervju} i samband med den intervju han medverkade i, testade också några av programmeringsproblemen dvs problem~\ref{sec:approx} till och med \ref{sec:sorteringsalgoritmer}. Sist följer en sammanfattning av några av frågorna, samt elevernas svar på frågan ''Vad är problemlösning för dig?''.}

\subsubsection{Fermiproblem}
    \label{resultat:Fermi}

\subsubsection{Försvåring av ekvation}
    \label{resultat:Ekvation}

    \subsubsection{Matematisk modell med bil och löpare}
        \label{resultat:Lopare}
    
        \textcolor{lila}{Problemet testades av två lärare. Den ena, som vi kallar lärare A, testade problemet i kursen matematik 1a, och den andra, lärare B, testade det i 1b. Båda lärarna har även angett att de arbetat med problemlösning i klasserna förut.}
    
        \textcolor{lila}{Båda två tyckte att informationen om problemet var tydlig och bra. Lärare A upplevde inget större intresse för uppgiften från eleverna, medan lärare B förklarar att eleverna ''körde igång med full fart''. Båda klasserna löste utelutande uppgiften genom att anta en linjär modell. Som extra intressant del har de kommenterat på den linjära modellen i uppgiften. Lärare A säger att det gav en bra diskussion om begränsningar hos den matematiska modellen, medan lärare B kommenterade att eleverna kunde formulera antagandet om konstant hastighet.}
    
        \textcolor{lila}{Lärare A uppskattar att alla elever deltog i problemet, och att de flesta hade nytta av diskussionen om rimligheten. Däremot uppskattar hen att bara $20\%$ hade nytta av beräkningsdelen av uppgiften. Lärare B tror att både hög- och lågpresterande elever hade nytta av problemet. Båda utvecklade också problemet genom att skissa en graf. Lärare B skissade den linjära modellen medan lärare A gemensamt med klassen försökte skissa hur modellen borde se ut för löparen. Lärare A fick inga särskillda frågor under lektionen, medan lärare B fick frågor om hur man omvandlade mellan m, km och mil.}
    
        \textcolor{lila}{Båda lärarna tror att eleverna lärde sig av problemet, framförallt om att man måste granska och värdera rimligheten i ett svar samt i matematiska modeller. Dock tyckte lärare B att uppgiften var lite väl lätt för att kallas problemlösning, och hade velat ha ett mer öppet problem. Lärare B var däremot väldigt nöjd, och tyckte att det var ''väl använd tid'', eftersom de annars ''är ganska låsta vid kursboken''.}
    
        \textcolor{lila}{Av de åtta elever som fyllde i utvärderingsenkäten var det en elev som inte alla tyckte att problemet kändes meningsfullt. Tre stycken tyckte att det var intressant, användbart eller båda delarna, och en kommenterade att det var bra att ''jämföra med verkligheten och tänka på rimligheten och relevansen''. Två elever tyckte att det var bra för att det var relativ lätt matematik och en säger enbart att vissa problem är meningsfulla. 
        Vid frågan om svårighetsgrad tyckte 2 styckan att detvar för svårt, en att det var för lätt och övriga 6 att det var lagom utmanande.}
    
        \textcolor{lila}{Sex av eleverna upplevde att de hade lärt sig något av uppgiften, exempelvis ''att tänka efter mer och inte bara skriva den uppenbara lösningen''. En av de som inte ansåg sig ha lärt någonting kommenterade att det var för att hen genomförde uppgiften på enklast möjliga sätt, och att det hade blivit svårare om hen hade funderat på relevansen i modellen.}
    
        \textcolor{lila}{Eleverna fick också frågan om ifall de vill göra fler uppgifter av den här typen eller hellre arbetar i boken. Två personer har svarat att det inte spelar någon roll, tre har angett att de vill ha en kombination av båda och lika många vill hellre göra den här typen av uppgifter. Två av de senare har dock angett att detta berodde på att de tyckte att uppgiften var lätt.}
    
    \subsubsection{Försvåring av en ekvation}
        \textcolor{lila}{En lärare testade problemet ''Försvåring av en ekvation'' i kursen matematik 1a. Klassen hade enstaka gånger arbetat med problemlösning tidigare, på ungefär samma sätt med grupper och därefter diskussion i helklass.}
        
        \textcolor{lila}{Läraren var nöjd med den information hörde till problemet, och upplevde att eleverna blev ganska engagerade och intresserade av det. De hade tyckt att det var extra roligt att få visa upp sina jobbiga ekvationer, och läraren tyckte att problemet passade de allra flesta. Läraren lyfter dock fram att många hade problem med prioriteringsreglerna, och inte hanterade instruktionerna att ''man får göra vilka matematiska operationer som helst med en ekvation, bara man gör det på båda sidorna'' på riktigt rätt sätt. Några hade också haft lite svårt för att skriva om ekvationen stegvis, men de flesta hade förstått det momentet.}
        
        \textcolor{lila}{Läraren upplevde att eleverna lärde sig av problemet, framförallt påmindes de om vikten av att göra samma sak i båda led, och läraren hoppas att de fick en djupare förståelse för hur en ekvation byggs upp. Problemet kändes relevant, en läraren kommenterar att man borde förtydliga så att det inte blir så stora problem med prioriteringsreglerna, samt föreslår att man skulle kunna börja med en lite mindre  uppgift så att eleverna kommer in i uppgiften först.}
        
        \textcolor{lila}{Här var det fyra elever som svarade på utvärderingsenkäten. Två av dem tyckte att problemet var lagom svårt, medan en tyckte det var för lätt och en att det var för svårt.  Tre av dem tyckte att problemet var roligt och lika många angav att de lärt sig något om ekvationer eller om att lösa svåra uppgifter. En elev nämnde också att lektionen blev väldigt rörig, och att det därför var svårt att lära sig något. Hälften skulle vilja göra mer av den här typen av uppgifter, medan hälften hellre arbetar själva i boken.}
        
    \subsubsection{Fermiproblem}
        Fermiproblemen testades i kursen matematik 4, med en klass som enligt läraren arbetade en del med problemlösning under det första året på gymnasiet, men inte så mycket i år två.

    
\subsection{Sammanfattning}

\subsection{Elevsvar: Vad är problemlösning?}


