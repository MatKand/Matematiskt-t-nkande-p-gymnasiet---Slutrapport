\textcolor{lila}{Nedan presenteras svaren på de utvärderingsenkäter vi fick in. De uppgifter som testades var ''Fermiproblem'', ''Försvåring av ekvation'' och ''Matematisk modell för bil och löpare'', vilka förklaras närmare i avsnitt~\ref{sec:Fermi}, \ref{sec:ekvation} respektive \ref{sec:lopare}. Niklas grip, som presenterades i avsnitt~\ref{sec:intervju} i samband med den intervju han medverkade i, testade också några av programmeringsproblemen dvs problem~\ref{sec:approx} till och med \ref{sec:sorteringsalgoritmer}. Vissa svar presenteras under respektive problem, men andra, mer generella svar, presenteras för sig.}

\subsection{Generella åsikter från lärarna}

Hjälpte informationen/bakgrunden given i uppgiftsbeskrivningen? På vilket sätt?
A: Ja, bra med världsrekordstiden.
B: Det var bra med en tydlig uppgift som jag som lärare med lätthet kunde ta till mig och förstå meningen med.

Var det någon information som du saknade? Utveckla.
A: Nej
B: Inget som jag tyckte saknades.


\subsection{Generella åsikter från eleverna}

\subsubsection{Fermiproblem}
    \label{resultat:Fermi}

\subsubsection{Försvåring av ekvation}
    \label{resultat:Ekvation}

\subsubsection{Matematisk modell med bil och löpare}
    \label{resultat:Lopare}
    
    \textcolor{lila}{Problemet testades av två lärare. Den ena, som vi kallar lärare A, testade problemet i kursen matematik 1a, och den andra, lärare B, testade det i 1b. Båda lärarna har även angett att de arbetat med problemlösning i klasserna förut.}
    
    \textcolor{lila}{Lärare A upplevde inget större intresse för uppgiften från eleverna, medan lärare B förklarar att eleverna ''körde igång med full fart''. Båda klasserna löste utelutande uppgiften genom att anta en linjär modell. Som extra intressant del har de kommenterat på den linjära modellen i uppgiften. Lärare A säger att det gav en bra diskussion om begränsningar hos den matematiska modellen, medan lärare B kommenterade att eleverna kunde formulera antagandet om konstant hastighet.}
    
    \textcolor{lila}{Lärare A uppskattar att alla elever deltog i problemet, och att de flesta hade nytta av diskussionen om rimligheten. Däremot uppskattar hen att bara $20\%$ hade nytta av beräkningsdelen av uppgiften. Lärare B }

    


