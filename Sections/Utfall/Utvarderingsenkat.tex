\textcolor{lila}{Nedan presenteras svaren på de utvärderingsenkäter vi fick in. I slutändan var det fyra av de tjugo anmälda som genomförde ett problem, samt Niklas Grip, som presenterades i avsnitt~\ref{sec:intervju} i samband med den intervju han medverkade i. De lärare som hörde av sig och sa att de inte skulle kunna genomföra problemen förklarade att det berodde på tidsbrist inför de nationella proven. De flesta skrev dock att de tyckte att det verkade vara bra problem med ett genomarbetat och tydligt upplägg.}

\textcolor{lila}{De problem som testades var ''Fermiproblem'', ''Försvåring av ekvation'' och ''Matematisk modell för bil och löpare'', vilka förklaras närmare i avsnitt~\ref{sec:Fermi}, \ref{sec:ekvation} respektive \ref{sec:lopare}. Vidare testade Niklas programmeringsproblemen ''Binär till decimal'', ''Fibonaccis talsekvens'', ''Identifiera primtalsfaktorer'' samt ''Personnummer'', vilka förklaras närmare i avsnitt \ref{sec:binar}, \ref{sec:Fibonacci}, \ref{sec:primtal}, respektive \ref{sec:Personnummer}.
%Niklas testade några av programmeringsproblemen, det vill säga problem~\ref{sec:sorteringsalgoritmer} till och med \ref{sec:chiffer}. 
Först presenteras de viktigaste resultaten av undersökningen, och därefter följer varje problem för sig, med svar från lärare och i de flesta fall elever. Sist följer elevernas svar på frågan ''Vad är problemlösning för dig?''.}

\subsubsection{Sammanfattning av några av de viktigaste resultaten}
    \textcolor{lila}{Sammanfattningsvis var alla lärare till största delen nöjda med problemen. En lärare uttryckte visserligen att den rent matematiska delen av ''Matematisk modell för bil och löpare'' var för lätt, men var mycket nöjd med den diskussion med rimlightesananlys som följde. Alla lärarna var också väldigt nöjda med hur vi presenterade problemen, och alla hade upplevt att eleverna lärt sig något av problemet.}
    
    \textcolor{lila}{Av alla elever tyckte ungefär $80\%$ att de lärde sig av problemet och $75\%$ uttryckte på något sätt att de tyckte att problemet var roligt eller meningsfullt. Ungefär $58\%$ av eleverna tyckte att problemet var lagom svårt, medan cirka $19\%$ ansåg att det var för svårt, och $23\%$ tyckte att det var för enkelt. Slutligen ville $38,5\%$ hellre jobba med den här typen av problem än uppgifter i boken, medan $25\%$ föredrog boken. Resterande $38,5\%$ ville att undervisningen skulle bestå av en blandning av båda.}
    
    
    \subsubsection{Fermiproblem}
        \label{resutat:Fermi}
        \textcolor{lila}{Fermiproblemen testades i kursen matematik 4, med en klass som enligt läraren arbetade en del med problemlösning under det första året på gymnasiet, men inte så mycket i år två. Den problemlösning de gjort tidigare har varit i form av ''praktiska öppna problem och problemlösning i bok och på genomgångar''.}
        
        \textcolor{lila}{Problemet angavs vara tydligt beskrivet, och eleverna ''gick igång'' när läraren presenterade de olika fermiproblemen. Läraren anger att alla elever deltagit aktivt under hela lektionen, och ställde frågor som ''Är det här rimligt?'' och ''Kan det vara sant att...''. Som extra information berättade läraren även om Fermis liv, vilket väckte elevernas intresse. De googlade vidare på egen hand och upptäckte bland annat fermiparadoxen\footnote{Motsägelsen i att universum är oändligt stort och det därmed är mycket stor sannolikhet för utomjordiskt liv, men att inga tecken på detta ännu har hittats.}, som det blev en diskussion om.}
        
        \textcolor{lila}{Läraren ansåg att eleverna lärde sig av problemet, framförallt att de fick en koppling till verkligheten och att de insåg att ''alla svar inte går att hitta i matteboken och att man kommer långt med att tänka realistiskt''. Dock anmärkte läraren att det var någon grupp som direkt kunde hitta svaret på någon om frågorna genom att googla, vilket gjorde dem mindre motiverade att uppskatta detta själva. Läraren planerade också in så att problemet genomfördes under en ''klämlektion'' precis innan lovet, istället för att då introducera ett nytt område, vilket hen ansåg fungerade väldigt bra.}
        
        \textcolor{lila}{Av de tretton elever som svarade på enkäten tyckte nio stycken att problemet var roligt, intressant eller meningsfullt. Fyra eleverna var istället negativt inställda, och tyckte att det var tråkigt eller irrelevant. En av dessa tyckte även att det var dåligt att det inte var tillräckligt med data givet till problemet. En elev lämnade blankt på denna fråga. Problemet var lagom utmanande för fem av eleverna, medan sex stycken tyckte att det var för lätt och två att det var för svårt. Även här angav en av eleverna bristen på data som anledning. Den andra tyckte att det blev svårt eftersom det blev många uträkningar att genomföra. Däremot har hela elva av dessa tretton angivit att de lärt sig något av uppgiften. En del av lärdomarna har varit rena faktasvar, t.ex hur många slag ett hjärta slår under en livstid, medan andra har angett en mer principiell insikt, som till exempel att ''det går att räkna ut saker som kan verka omöjliga'' eller ''hur enkel matte kan användas för att lösa problem och hur uppdelning kan förenkla''. De som angett att de inte lärde sig något säger att det berodde på att de inte var intresserade.}
        
        \textcolor{lila}{Gällande vilka typer av uppgifter de helst vill jobba med anger fyra av eleverna att de hellre arbetar i boken, men lika många tyckte att uppgiften var rolig och lärorik och vill gärna göra fler liknande uppgifter. Fyra elever poängterar att de gärna vill ha en blandning av båda, och en elev har valt att svara blankt på denna fråga.}
        
    \subsubsection{Försvåring av en ekvation}
        \label{resultat:ekvation}
        \textcolor{lila}{En lärare testade problemet ''Försvåring av en ekvation'' i kursen matematik 1a. Klassen hade enstaka gånger arbetat med problemlösning tidigare, på ungefär samma sätt med grupper och därefter diskussion i helklass.}
        
        \textcolor{lila}{Läraren var nöjd med den information som hörde till problemet, och upplevde att eleverna blev ganska engagerade och intresserade av det. De hade tyckt att det var extra roligt att få visa upp sina jobbiga ekvationer, och läraren tyckte att problemet passade de allra flesta. Läraren lyfter dock fram att många hade problem med prioriteringsreglerna, och inte hanterade instruktionerna att ''man får göra vilka matematiska operationer som helst med en ekvation, bara man gör det på båda sidorna'' på riktigt rätt sätt. Några hade också haft lite svårt för att skriva om ekvationen stegvis, men de flesta hade förstått det momentet.}
        
        \textcolor{lila}{Läraren upplevde att eleverna lärde sig av problemet, framförallt påmindes de om vikten av att göra samma sak i båda led, och läraren hoppas att de fick en djupare förståelse för hur en ekvation byggs upp. Problemet kändes relevant, läraren kommenterar att man borde förtydliga så att det inte blir så stora problem med prioriteringsreglerna, samt föreslår att man skulle kunna börja med en lite mindre  uppgift så att eleverna kommer in i uppgiften först.}
        
        \textcolor{lila}{Här var det fyra elever som svarade på utvärderingsenkäten. Två av dem tyckte att problemet var lagom svårt, medan en tyckte det var för lätt och en att det var för svårt. Tre av dem tyckte att problemet var roligt och lika många angav att de lärt sig något om ekvationer eller om att lösa svåra uppgifter. En elev nämnde också att lektionen blev väldigt rörig, och att det därför var svårt att lära sig något. Hälften skulle vilja göra mer av den här typen av uppgifter, medan hälften hellre arbetar själva i boken.}

    \subsubsection{Matematisk modell för bil och löpare}
        \label{resultat:Lopare}
    
        \textcolor{lila}{Problemet testades av två lärare. Den ena, som vi kallar lärare A, testade problemet i kursen matematik 1a, och den andra, lärare B, testade det i 1b. Båda lärarna har även angett att de arbetat med problemlösning i klasserna förut.}
    
        \textcolor{lila}{Båda två tyckte att informationen om problemet var tydlig och bra. Lärare A upplevde inget större intresse för uppgiften från eleverna, medan lärare B förklarar att eleverna ''körde igång med full fart''. Båda klasserna löste utelutande uppgiften genom att anta en linjär modell. Som extra intressant del har de kommenterat på den linjära modellen i uppgiften. Lärare A säger att det gav en bra diskussion om begränsningar hos den matematiska modellen, medan lärare B kommenterade att eleverna även kunde formulera antagandet om konstant hastighet.}
    
        \textcolor{lila}{Lärare A uppskattar att alla elever deltog i problemet, och att de flesta hade nytta av diskussionen om rimligheten. Däremot uppskattar hen att bara $20\%$ hade nytta av beräkningsdelen av uppgiften. Lärare B tror att både hög- och lågpresterande elever hade nytta av problemet. Båda utvecklade också problemet genom att skissa en graf. Lärare B skissade den linjära modellen medan lärare A gemensamt med klassen försökte skissa hur modellen borde se ut för löparen. Lärare A fick inga särskillda frågor under lektionen, medan lärare B fick frågor om hur man omvandlade mellan m, km och mil.}
    
        \textcolor{lila}{Båda lärarna tror att eleverna lärde sig av problemet, framförallt om att man måste granska och värdera rimligheten i ett svar samt i matematiska modeller. Dock tyckte lärare B att uppgiften var lite väl lätt för att kallas problemlösning, och hade velat ha ett mer öppet problem. Lärare B var däremot väldigt nöjd, och tyckte att det var ''väl använd tid'', eftersom de annars ''är ganska låsta vid kursboken''.}
    
        \textcolor{lila}{Av de åtta elever som fyllde i utvärderingsenkäten var det en elev som inte alls tyckte att problemet kändes meningsfullt. Tre stycken tyckte att det var intressant, användbart eller båda delarna, och en kommenterade att det var bra att ''jämföra med verkligheten och tänka på rimligheten och relevansen''. Två elever tyckte att det var bra för att det var relativ lätt matematik och en elev gav den svårtolkade kommentaren att ''vissa problem är meningsfulla''. 
        Vid frågan om svårighetsgrad tyckte två stycken att det var för svårt, en att det var för lätt och övriga 6 att det var lagom utmanande.}
    
        \textcolor{lila}{Sex av eleverna, det vill säga $75\%$, upplevde att de hade lärt sig något av uppgiften, exempelvis ''att tänka efter mer och inte bara skriva den uppenbara lösningen''. En av de som inte ansåg sig ha lärt någonting kommenterade att det var för att hen genomförde uppgiften på enklast möjliga sätt, och att det hade blivit svårare om hen hade funderat på relevansen i modellen.}
    
        \textcolor{lila}{Eleverna fick också frågan om ifall de vill göra fler uppgifter av den här typen eller hellre arbetar i boken. Två personer har svarat att det inte spelar någon roll, tre har angett att de vill ha en kombination av båda och lika många vill hellre göra den här typen av uppgifter. Två av de senare har dock angett att detta berodde på att de tyckte att uppgiften var lätt.}
        
    \subsubsection{Programmeringsproblem}
        \label{resultat:Programmering}
        
        \textcolor{lila}{Matematikläraren Niklas Grip från intervjun i avsnitt~\ref{sec:intervju} testade några av våra programmeringsproblem i sin kurs matematik specialicering. Mer specifikt var det problemen ''Binär till decimal'' (\ref{sec:binar}), ''Fibonaccis talsekvens'' (\ref{sec:Fibonacci}), ''Identifiera primtalsfaktorer'' (\ref{sec:primtal}) samt ''Personnummer'' (\ref{sec:Personnummer}). Klassen har både under år ett och två arbetat ganska mycket med problemlösning, och specifikt i samband med programmering.}
        
        \textcolor{lila}{Niklas tyckte att instruktionerna till problemet var tydliga och hjälpsamma, och att eleverna fångades av programmeringsproblemen. Speciellt av de med en tydlig verklighetsanknytning, som till exempel problemet om kontroll av personnummer. Han hade valt ett upplägg som gjorde att eleverna själva fick välja om de ville genomföra uppgifterna. Han tror att de som valde att inte göra uppgifterna inte heller hade klarat av att genomföra dem själva, och att det då hade varit ett krav att eleverna fick arbeta i grupp.}
        
        \textcolor{lila}{När det gällde de frågor som Niklas fick av eleverna var det en blandning av problemlösningsfrågor i form av till exempel ''Hur ska man tänka här'' och rena syntaxfrågor. Han upplevde att eleverna lärde sig olika sätt som man kan använda programmering på, samt att ''Det var ganska olika problem vilket ställer högre krav på problemlösningsförmåga och kreativitet.'' Han ansåg att svårighetsgraden var perfekt för de som ville ha lite mer avancerade uppgifter, men nämnde som sagt att han tror att de något svagare eleverna hade behövt arbeta tillsammans för att klara av att lösa problemen.}
        
\subsubsection{Elevernas åsikt om vad problemlösning är}
    \textcolor{lila}{Det vanligaste svaret på frågan ''Vad är problemlösning för dig?'' var svar liknande antingen ''matematik'' eller ''Att lösa problem''. Denna tolkning gavs som svar av ungefär $44\%$ av eleverna. Därutöver svarade några att det var svåra problem medan någon tvärtom sa att det var lätt. Cirka $12\%$ sa att det var problem som ''kräver ett friare tänkande'' eller att det går ut på att ''kunna förstå frågan och bearbeta den''. Ungefär $19\%$ nämner verklighetsanknytning som en viktig faktor, eller mer specifik ''problem som man kan stöta på i vardagen''.}


