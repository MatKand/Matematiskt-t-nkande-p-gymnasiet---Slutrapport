\textcolor{green}{Metoden för att testa de problem som tagits fram i detta arbete var att låta matematiklärare testa uppgifterna på sina elever. Tanken med detta var att uppgifterna skulle kunna testat i stor skala på många gymnasieskolor samtidigt. På så sätt samlades det in stora kvantiteter av data på ett relativt enkelt sätt. Dock så har denna metod några större nackdelar.}

\textcolor{green}{När lärarna testade uppgifterna var tanken att de skulle utgå ifrån uppgifternas beskrivningar, men att de sedan hade relativt fria tyglar att förmedla uppgifterna på det sätt de fann lämpliga. Ett dilemma med denna metod var att det inte fanns någon möjlighet för oss att se hur lärarna faktiskt genomförde lektionen. Den data vi samlade in via enkäterna gav oss elevernas och lärarnas individuella bedömningar.}

%
% * Vi använde denna metoden
% * Den data som sedan samlades in på hur testerna hade gått samlades in via enkäter. Hur lärarna gjorde för att ta reda på vad eleverna tyckte kan ha varierat. Vissa lärare kan ha frågat eleverna medan andra kan ha skrivit ner vad de själva tror att eleverna tyckte
% 
% det fanns inget sätt för oss att se hur läraren genomförda lektionen
%
% MATTIAS REKKAR MIG I FIFA
%