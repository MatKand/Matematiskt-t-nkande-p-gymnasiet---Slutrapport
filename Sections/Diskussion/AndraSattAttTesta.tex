\textcolor{green}{Metoden för att testa de problem som tagits fram i detta arbete var att låta matematiklärare testa uppgifterna på sina elever. Tanken med detta var att uppgifterna skulle testas i stor skala på många gymnasieskolor samtidigt. På så sätt samlades det in stora kvantiteter av data på ett enkelt sätt. Dock så hade denna metod några större nackdelar.}

\textcolor{green}{När lärarna testade uppgifterna var tanken att de skulle utgå ifrån uppgifternas beskrivningar, men att de sedan hade relativt fria tyglar att förmedla uppgifterna på det sätt de fann lämpliga. Ett dilemma med denna metod var att det inte fanns någon möjlighet för oss att direkt se hur lärarna genomförde lektionen. Den data vi samlade in via enkäterna gav oss elevernas och lärarnas individuella bedömningar. Trots stor kvantitet blev det svårt att verifiera hur kvalitativ datan var.}

\textcolor{green}{Ett alternativ till att be en lärare att prova uppgifterna på sin klass hade varit att vi själva hade utfört testerna på gymnasielever. En stor skillnad hade varit att vi hade kunnat testa uppgifterna på precis samma sätt med många olika klasser. Datan vi sedan hade samlat in hade varit mer kvalitativ, eftersom vi hade varit konsekventa med tillvägagångssättet vi testade uppgifterna på.}
    \textcolor{WildStrawberry}{
    Om vi tagit en mer aktiv ställning till testandet kunde detta varit möjligt, men i brist på tid gav vi lärarna friheten att testa problemen på det vis de anser bäst själva.} \textcolor{lila}{Fördelen med detta är dock att det bättre speglar hur våra problem ska användas, det vill säga som en guide för att hjälpa lärare  planera och genomföra lektioner med problemlösning.}


%
% * Vi använde denna metoden
% * Den data som sedan samlades in på hur testerna hade gått samlades in via enkäter. Hur lärarna gjorde för att ta reda på vad eleverna tyckte kan ha varierat. Vissa lärare kan ha frågat eleverna medan andra kan ha skrivit ner vad de själva tror att eleverna tyckte
% 
% det fanns inget sätt för oss att se hur läraren genomförda lektionen
%
% 
%