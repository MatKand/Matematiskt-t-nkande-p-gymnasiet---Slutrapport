\textcolor{lila}{I avsnitt~\ref{sec:Skapandetavproblem} nämndes de viktigaste faktorerna som vi har försökt inkludera i problemen. Som tidigare nämnts appliceras dock inte alla dessa faktorer på samtliga problem, men ett eller flera ska ändå finnas som grundtanke bakom varje skapat problem. Den kanske viktigaste av dessa faktorer är att eleverna använder ett \textsl{undersökande arbetssätt}, vilket vi anser att vi har inkluderat i samtliga av våra problem. I de flesta problemen lägger detta grunden för att lösa de givna frågeställningarna medan det i problemet med den matematiska modellen, som presenteras i avsnitt~\ref{sec:lopare}, snarare dyker upp i den avslutande diskussionen. I samma problem tas även begreppet \textsl{matematisk modell} in. Eleverna får här upptäcka hur matematiska modeller kan användas och även reflektera över dess begränsningar.} 

\textcolor{lila}{Om man tittar på ''Fermiproblem'' (\ref{sec:Fermi}), ''Flygplan'' (\ref{sec:Flygplan}), ''Fritt fall'' (\ref{sec:FrittFall}), ''Matematisk modell för bil och löpare'' (\ref{sec:lopare}) och ''Personnummer'' (\ref{sec:Personnummer}) finns där också en mycket tydlig \textsl{verklighetsanknytning}. Problemen vid namn ''Sortering av kortlek'', (\ref{sec:Sortera}) ''Sorteringsalgoritmer'' (\ref{sec:sorteringsalgoritmer}) och ''Skapa ett chiffer'' (\ref{sec:chiffer}) har också en förankring i verkligheten även om denna är någon mindre tydligt än för ovanstående exempel. I dessa uppgifter har vi dock inkluderat information om hur de används i verkligheten, och uppmuntrar även läraren till att berätta om detta. Övriga uppgifter är något mer abstrakta.}

\textcolor{green}{Problemen ~\ref{sec:sorteringsalgoritmer} till och med~\ref{sec:chiffer} är uppgifter som är tänkta att framför allt förbereda elever för grundläggande koncept inom \textsl{programmering}. Vi skriver i avsnitt~\ref{sec:ProgrammeringOchMatematik} om att programmering framöver kommer att ingå i kursplanen för gymnasiematematik, vilket gör det väldigt relevant att utforma problem som även ska testa programmeringskunskaper, samtidigt som många av dem även har en tydlig matematisk koppling.} \textcolor{lila}{I problemet ''Sortera en kortlek'' (\ref{sec:Sortera}) finns programmeringen också med, om än något i bakgrunden. Här får eleverna testa på att tänka som en dator, och även ge varandra instruktioner på liknande form som man skulle göra i en programkod. På så sätt är detta tänkt som en bra introduktion innan man börjar med programmeringssyntax i ett specifikt språk.}

% \textcolor{green}{Vissa programmeringsproblem har även, som beskrivs i avsnitt \ref{sec:tankarbakomprob}, verklighetsanknytning. Exempel på dessa är problem som behandlar väldigt enkel kryptering och att kunna verifiera personnummer, med hjälp av algoritmer som används i verkligheten.}

% \textcolor{green}{De flesta andra problemen har en tydligare matematisk koppling, där förhoppningen är att man både får träning att programmera samtidigt som man får en fördjupning inom olika matematiska begrepp.}

%Dessa var framtagna efter specifikt önskemål av Niklas, som det står mer om i~\ref{sec:intervju}. Eftersom han redan i sin undervisning lär ut programmering till sina elever så kände han att det vore relevant att få programmeringsproblem att använda i sin undervisning.