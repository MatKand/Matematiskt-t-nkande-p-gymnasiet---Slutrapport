% Jag kommer under denna kommentar att diskutera programmeringsproblemen, de andra problemen bör diskuteras ovanför den här.
\textcolor{green}{Problemen ~\ref{sec:sorteringsalgoritmer} till och med~\ref{sec:chiffer} är uppgifter som är tänkta att framför allt förbereda elever för grundläggande koncept inom programmering. Dessa var framtagna efter specifikt önskemål av Niklas, som det står mer om i~\ref{sec:intervju}. Eftersom han redan i sin undervisning lär ut programmering till sina elever så kände han att det vore väldigt relevant att få programmeringsproblem att använda i sin undervisning. Det tillsammans med det som vi skriver om i~\ref{sec:ProgrammeringOchMatematik}, att programmering framöver kommer att ingå i kursplanen för gymnasiematematik, gör det väldigt relevant att utforma problem som även ska testa ren syntax.}

\textcolor{green}{Ett av problemen som togs fram har en tydlig koppling till ett av våra andra problem, nämligen~\ref{sec:Sortera} och~\ref{sec:sorteringsalgoritmer}. Medan det förstnämnda problemet mer är en introduktion att förstå hur datorer ''tänker'', så är det andra att också kunna implementera en riktig algoritm. Vad för algoritm det är väljer förstås eleverna själva, men i lösningsförslaget presenteras två av de enklare algoritmerna.}

\textcolor{green}{En del av problemen har även, som beskrivs i avsnitt \ref{sec:tankarbakomprob}, verklighetsanknytning. Framför allt kanske det tidigare nämnda problemet om sorteringsalgoritmer, men även problem som behandlar väldigt enkel kryptering och att kunna verifiera personnummer med hjälp av algoritmer som används i verkligheten.}

\textcolor{green}{De flesta andra problemen har en tydligare matematisk koppling, där förhoppningen är att man både får träning att programmera samtidigt som man får en fördjupning inom olika matematiska begrepp.}