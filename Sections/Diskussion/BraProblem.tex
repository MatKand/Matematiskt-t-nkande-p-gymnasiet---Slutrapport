% Vi testade bara problem som utfördes under 1h
\textcolor{Mahogany}{Målet med de problem som vi utformat i projektet är att framhäva vikten av problemlösning i matematiken. Samtidigt så är också ett mål att visa hur den kan vara användbar, och att den användbarheten inte enbart existerar i ''mattelandet'', som beskrivs i \ref{sec:Verklighetsanknytning}. Ett \textsl{bra} problem bör alltså inte bara framhäva att matematik är användbart i det verkliga livet, utan att även få eleven att känna att det är användbart. Detta kan vara betydligt lättare att uppnå med programmeringsproblem, då man, utöver programmeringsspråkets syntax, inte är bunden till hur man kan lösa något av problemen. Man ges också i många fall möjligheten att själv avgränsa hur utförligt problemet ska utföras. Som exempel så har vi utformat ett problem där man ska verifiera ett personnummer, där man företrädesvis kan använda sig av den så kallade Luhn-algoritmen\cite{Luhn}. Dock så kanske det inte räcker med att enbart kontrollera med hjälp av den algoritmen. Om man kontrollerar personnummer på de i sin klass så kanske man vill kontrollera att personen inte är född på 1800-talet exempelvis!}
 
% Ta ställning till omständigheter
\textcolor{Mahogany}{Något som är viktigt att ta ställning till när man utformar problem för gymnasiematematik är att det måste vara förenligt med övriga undervisningen. Man ska övertyga både lärare och elever att det för det första är relevant, men också att det kommer att vara mer givande än att räkna i boken. Det är också viktigt att ha som en röd tråd vad man egentligen vill att elever ska få ut av ett problem, och försöker förmedla detta på ett naturligt sätt. Är problemlösning bara något som tar tid från räkning i boken eller kan man få elever att förstå att de kan gynnas av denna typ av matematikundervisning? Ett ''bra'' problem bör rimligtvis lyfta fram den nyttan. Framför allt så bör de även vara roliga att lösa.}

% Hur bör det utföras?
\textcolor{Mahogany}{Vi har lagt ner en del tid på att informera lärare hur vi tänker att våra problem ska utföras. Detta är för att vi vill att man ska få ut så mycket som möjligt från problemen framförallt via diskussion och reflektion. Diskussionsdelen är också ett sätt för elever att våga försöka lösa ett problem som till synes kanske skulle vara avskräckande. Oavsett svårighetsgraden på problemet så anser vi att diskussionen är en viktig del i lärandet, då man får möjligheten att uttrycka sin nivå av förstående och ens kunskap testas genom att man behöver förklara det för någon annan. Samtidigt så gynnas de elever som inte kommit lika långt av att få höra hur andra tänkt, och därifrån kunna arbeta vidare själva.}

% Rimlig nivå
\textcolor{Mahogany}{En svårighet med att utveckla problem är att sätta en rimlig svårighetsgrad, speciellt med tanke på att vi under utvärderingen fick jobba med framför allt väldigt specifik tidsram. Ett sätt att konfrontera detta är att ha en tydlig grund som problemet bygger på, för att sedan möjliggöra vidareutveckling i största mån. Denna vidareutveckling handlar ofta om fördjupningsfrågor som uppmanar till diskussion, vilket kan underbygga elevernas möjlighet att få en djupare förståelse för ämnet.}
\newline

% Risken med verkliga problem
\textcolor{Mahogany}{En utmaning med att försöka anknyta matematik till något verkligt är att det finns en risk att bli krystat och samtidigt tar ifrån eleven rätten att utforska hur något kan användas i verkligheten. I ''A Mathematician's Lament'' \cite{lockhart} så framhäver Lockhart behovet av att elever ska få utforska matematiken och att man ska försöka underbygga deras fantasi istället. En utmaning med att utforma ''verkliga'' problem är att ta ställning till detta samtidigt som man vill framhäva den mångsidiga nyttan med matematiken.}
