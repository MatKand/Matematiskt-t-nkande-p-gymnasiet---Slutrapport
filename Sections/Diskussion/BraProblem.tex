% Vi testade bara problem som utfördes under 1h
\textcolor{Mahogany}{
    %Målet med de problem som vi utformat i projektet är att framhäva vikten av problemlösning i matematiken. Samtidigt så är också ett mål att visa hur den kan vara användbar, och att den användbarheten inte enbart existerar i det ''matteland'', som beskrivs i \ref{sec:Verklighetsanknytning}.  Man vill inte heller, som nämndes i avsnitt~\ref{sec:problemdef}, riskera att verklighetsanknytning blir en alltför stor faktor i problemskapandet. Ibland kan ett abstrakt problem leda till ett större engagemang och nyfikenhet som ett med en tydlig verklighetsanknytning. En utmaning med att utforma verkliga problem är därför att ta ställning till detta samtidigt som man framhäver den mångsidiga nyttan av matematiken.
    Ett mål med de problem som vi har utformat är att vi vill visa hur användbar matematik kan vara, och att den användbarheten inte enbart existerar i det ''matteland'', som beskrivs i \ref{sec:Verklighetsanknytning}.  Man vill inte heller, som nämndes i avsnitt~\ref{sec:problemdef}, riskera att verklighetsanknytning blir en alltför stor faktor i problemskapandet. Ibland kan ett abstrakt problem leda till ett större engagemang och nyfikenhet som ett med en tydlig verklighetsanknytning. En utmaning med att utforma verkliga problem är därför att ta ställning till detta samtidigt som man framhäver den mångsidiga nyttan av matematiken.
}

%\textcolor{Mahogany}{
    % Detta kan vara betydligt lättare att uppnå med programmeringsproblem, då man, utöver programmeringsspråkets syntax, inte är bunden till hur man kan lösa något av problemen. Man ges också i många fall möjligheten att själv avgränsa hur utförligt problemet ska utföras. Som exempel så har vi utformat ett problem där man ska verifiera ett personnummer, där man företrädesvis kan använda sig av den så kallade Luhn-algoritmen\cite{Luhn}. Dock så kanske det inte räcker med att enbart kontrollera med hjälp av den algoritmen. Om man kontrollerar personnummer på de i sin klass så kanske man vill kontrollera att personen inte är född på 1800-talet exempelvis!
%}
 
% Röd tråd samt övertyga elever att det är relevant
\textcolor{Mahogany}{
    Med våra uppgifter vill vi övertyga både lärare och elever att de är relevanta, men också att de kommer vara mer givande än att enbart räkna i boken. Det är också viktigt att ha som en röd tråd vad man egentligen vill att elever ska få ut av ett problem, och försöka förmedla detta på ett naturligt sätt. Är problemlösning bara något som tar tid från räkning i boken eller kan man få elever att förstå att de kan gynnas av denna typ av matematikundervisning? Ett bra problem bör därför rimligtvis lyfta fram den nyttan. Framför allt så bör problemen även vara roliga att lösa, och ge eleverna en chans att utforska och utmanas av matematiken.
}

% Hur bör det utföras?
\textcolor{Mahogany}{
    Vi har lagt ner en del tid på att informera lärare hur vi tänker att våra problem ska utföras. Detta framförallt för att vi anser att diskussion och reflektion är mycket viktiga för inlärningen, vilket även diskuterades i avsnitt~\ref{sec:Diskussion}. Diskussionsdelen är också ett sätt för elever att våga försöka lösa ett problem som till synes kanske skulle vara avskräckande. Oavsett svårighetsgraden på problemet så anser vi att diskussionen är en viktig del i lärandet, då man får möjligheten att uttrycka sin nivå av förstående och ens kunskap testas genom att man behöver förklara det för någon annan. Samtidigt så gynnas de elever som inte kommit lika långt av att få höra hur andra tänkt, och därifrån kunna arbeta vidare själva.
}

% Rimlig nivå
\textcolor{Mahogany}{
    En svårighet med att utveckla problem är att sätta en rimlig svårighetsgrad, speciellt med tanke på att vi under utvärderingen fick jobba med en väldigt specifik tidsram. Ett sätt att konfrontera detta är att ha en tydlig grund som problemet bygger på, för att sedan möjliggöra vidareutveckling i största mån. Denna vidareutveckling handlar ofta om fördjupningsfrågor som uppmanar till diskussion, vilket kan underbygga elevernas möjlighet att få en djupare förståelse för ämnet.
}