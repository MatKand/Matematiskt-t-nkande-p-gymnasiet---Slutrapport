\textcolor{lila}{Det största hindret mot mer problemlösning i undervisningen som framkom från bakgrundsenkäten var olika typer av tidsbrist. Dels framkom åsikter om att problemlösningsbaser undervisning tar mer tid från själva lektionerna och dels att de tar längre tid att planera än att låta eleverna räkna färdiga uppgifter från boken. Det här anges som tid man inte har efter att allt övrigt material har gåtts igenom, för att eleverna ska kunna prestera på nationella provet. Och självklart är matematikens grundläggande teori viktig, men vad betyder den om man aldrig får träna på att använda den? Vi anser att vissa delar av kraven i de olika kurserna borde nedprioriteras, och att problemlösning bör få en tydligare roll i undervisningen av matematik.} 

\textcolor{lila}{Idag är problemlösning något som till alltför stor del lämnas som extrauppgifter till de elever som har fallenhet för matematik, men vi anser att det är andra delar av matematiken som kan hanteras på detta sätt. Att man bör utforska matematiken med problemlösning som främsta verktyg, och låta eleverna själva upptäcka de begrepp och metoder som då dyker upp. Och de delar av matematiken som inte går att hitta på ett rimligt problem om, antingen hörande till nutida eller dåtida behov eller intressen, kanske helt enkelt inte kan vara så centrala i alla fall.}