
\textcolor{Mahogany}{Något som vi inte har undersökt i det här projektet är att uppskatta \textsl{hur stor del} av undervisningen som bör läggas på problemlösning.} \textcolor{turkos}{Som vi nämnde i avsnitt \ref{sec:pbl} så krävs det grundläggande kunskap för att kunna tänka som en expert, och det enda sättet att få den grundkunskapen är genom repetition. }\textcolor{Mahogany}{Detta belyser att det även finns ett behov av vanlig mekanisk räkning. Vi vill därför förtydliga att vi inte förespråkar att man fullständigt bör ersätta dagens matematik med problemlösning. Det vi vill få fram av den här rapporten är att problemlösning kan användas för att ge eleverna en djupare förståelse för de matematiska teorier som de lär sig. Detta har visat sig att det i många fall inte appliceras så mycket som Skolverket och lärarna önskar.}

\textcolor{lila}{Enligt bakgrundsenkäten som finns under avsnitt~\ref{sec:Bakgrundsenkat}, så är det största hindret mot att införa mer problemlösning i undervisningen olika typer av tidsbrist. Det blev även märkbart i praktiken när sexton av tjugo lärare som hade anmält intresse att genomföra problemen avböjde på grund av tidsbrist i samband med de nationella proven. Detta stärkte våra farhågor att alltför mycket vikt läggs vid de nationella proven och det som testas där.}

%Det största hindret mot att införa mer problemlösning i undervisningen var enligt bakgrundsenkäten var olika typer av tidsbrist. Dels framkom detta i bakgrundsenkäten som finns i avsnitt~\ref{sec:bakgrundsenkat}, och dels så blev det märkbart i praktiken när sexton av tjugo lärare som hade anmält intresse att genomföra problemen avböjde på grund av tidsbrist i samband med nationella proven. Detta stärkte våra farhågor att alltför mycket vikt läggs vid de nationella proven och det som testas  där. Detta diskuteras även mer ingående i avsnitt~\ref{sec:VadArViktigt}.}

\textcolor{lila}{I enkäten framkom bland annat åsikter om problemlösningsbaserad undervisning. De flesta av lärarna vet om att det är bra och viktigt för inlärningen, men tycker att det tar mer tid från själva lektionerna. Dessutom angav de att det tar längre tid att planera än att låta eleverna räkna färdiga uppgifter från boken. Det här anges som tid man inte har efter att allt övrigt material har gåtts igenom, för att eleverna ska kunna prestera på nationella provet. Och självklart är matematikens grundläggande teori viktig, men vad betyder den om man aldrig får träna på att använda den? Vi anser att vissa delar av kraven i de olika kurserna borde nedprioriteras, och att problemlösning bör få en tydligare roll i undervisningen av matematik.} 

\textcolor{lila}{Idag är problemlösning något som till alltför stor del lämnas som extrauppgifter till de elever som har fallenhet för matematik, men vi anser att det är andra delar av matematiken som kan hanteras på detta sätt. Att man bör utforska matematiken med problemlösning som främsta verktyg, och låta eleverna själva upptäcka de begrepp och metoder som då dyker upp. Och de delar av matematiken som inte går att hitta på ett rimligt problem om, antingen hörande till nutida eller dåtida behov eller intressen, kanske helt enkelt inte kan vara så centrala i alla fall.}
\textcolor{Mahogany}{Vårt mål är därför att underlätta för lärare att inkludera mer problemlösning i sin undervisning, så att problemlösning på sikt kan ta sin roll som huvudsakliga verktyg för att låta eleverna undersöka och upptäcka matematiken, för att i samband med undervisningen själva skapa sig en känsla för hur viktig och användbar matematiken verkligen är.}

%\textcolor{lila}{Bakgrundsenkäten i avsnitt~\ref{sec:bakgrundsenkat} visade att de flesta lärarna vet att problemlösning är väldigt bra och viktigt för inlärningen, men det nedprioriteras ändå mot andra metoder på grund av tidsbrist. Detta stärker våra farhågor att alltför mycket vikt läggs vid de nationella proven och det som testas  där. Detta diskuteras även mer ingående i avsnitt~\ref{sec:VadArViktigt}}.