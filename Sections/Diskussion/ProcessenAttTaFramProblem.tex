\noindent \textcolor{Mahogany}{Att ta fram problem har inte varit en helt lätt uppgift. Till en början så hade vi ett angreppssätt där vi inte alltid tog hänsyn till nivå och nödvändig teori, utan tog för givet att eleven med hjälp av nyfikenhet skulle vilja ta till sig den nya teori som skulle vara nödvändig för att lösa uppgiften. Vi fick inspiration från TED Talks och liknande där talare som forskar inom området gick igenom hur de ser på problemlösning och vilka aspekter de anser vara värdefulla. Ju mer kontakt vi fick med lärare ju mer insåg vi att vi dessvärre var tvungna att i någon mån anpassa våra problem för att de skulle passa in i deras undervisning, då de annars inte skulle ha möjlighet att testa dem. Vi bestämde oss därför för att avgränsa oss till kortare uppgifter. Detta dels för att underlätta testning och dels för att de skulle vara lättare för lärare att inkludera i sin undervisning även efter projektets slut. Vi har dock även diskuterat om att även utforma längre problem. I vissa fall kan det nämligen vara bra att få gott om tid för att fundera, och även ha tid till att testa olika metoder för att se vilken som fungerar bäst. Denna tanke ligger bakom till exempel ''Fermiproblem'', som är planerad för att kunna utföras under en enda lektion, men även kan utföras som ett lite längre projekt som kan avslutas med att eleverna får presentera sina olika resultat och metoder för klassen.}

\textcolor{Mahogany}{Med det sagt så har vi alltså begränsat oss till kortare problem om ca 50 minuter. Hade vi haft mer tid och möjlighet hade vi troligtvis även velat testa längre, mer projektliknande problem. Eftersom nästan hela projektgruppen tidigare läst kursen \textsl{Mathematical modelling and problem solving}\cite{matmod} där man fick veckovisa uppgifter, samt en som sträckte sig över merparten av kursen, så är vi också medvetna om vikten av att reflektera kring problem över en längre tid. Detta gör det möjligt att få en djupare förståelse för något än om man löser ett problem under en lektion.}