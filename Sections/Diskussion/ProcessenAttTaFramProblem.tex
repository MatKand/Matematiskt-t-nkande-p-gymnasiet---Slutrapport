\textcolor{Mahogany}{Att ta fram problem har inte varit lätt. Till en början så hade vi ett angreppssätt där vi inte alltid tog hänsyn till nivå och nödvändig teori, utan tog för givet att eleven med hjälp av nyfikenhet skulle vilja ta till sig den nya teori som skulle vara nödvändig för att lösa uppgiften. 
Ju mer kontakt vi fick med lärare ju mer insåg vi att vi var tvungna att i någon mån anpassa våra problem för att de skulle passa in i deras undervisning. Vi bestämde oss därför också för att avgränsa oss till kortare problem. Detta dels för att underlätta testning och dels för att de skulle vara lättare för lärare att inkludera i sin undervisning även efter projektets slut. Vi har dock även diskuterat om att utforma längre problem också. I vissa fall kan det nämligen vara bra att få gott om tid för att fundera, och även ha tid till att testa olika metoder för att se vilken som fungerar bäst. Denna tanke ligger bakom till exempel ''Fermiproblem'', som är planerad för att kunna utföras under en enda lektion, men även kan utföras som ett längre projekt som kan avslutas med att eleverna får presentera sina olika resultat och metoder för klassen.}

\textcolor{Mahogany}{Med det sagt så har vi alltså begränsat oss till kortare problem om cirka 50 minuter. Hade vi haft mer tid och möjlighet hade vi troligtvis även velat testa längre, mer projektliknande problem. Eftersom nästan hela projektgruppen tidigare läst kursen \textsl{Mathematical modelling and problem solving}\cite{matmod} där man fick veckovisa problem, samt en som sträckte sig över merparten av kursen, så är vi också medvetna om vikten av att reflektera kring problem över en längre tid. Detta gör det möjligt att få en djupare förståelse för något än om man löser ett problem under en lektion.}

\textcolor{lila}{
I enkäten uttrycktes att det är ett problem att hitta uppgifter, samt att hinna planera problemlösningslektioner. Niklas uttryckte även i sin intervju \ref{sec:intervju} att han saknade mer hjälp på \textsl{hur} problemlösning ska läras ut. }
\textcolor{Mahogany}{Utifrån detta känns det som ett utmärkt komplement till matematikboken att kunna ta del av kompletta lektionsplaneringar med problemlösning. I samband med problemen har vi därför valt att vara så utförliga som möjligt med instruktioner och förslag på utförande samt innehåll. Samtidigt har vi försökt understryka att de ska känna att de får styra innehållet utefter vad de anser passar bäst för sina egna klasser.}
\textcolor{lila}{Problemet ''Försvåra en ekvation'' skapades också för att kännas mest som en lek, och därmed kunna användas i en klass som inte sedan tidigare är vana vid att arbetamed problemlösning.}

