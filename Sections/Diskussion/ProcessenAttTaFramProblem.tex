\noindent \textcolor{Mahogany}{Att ta fram problem har inte varit en helt lätt uppgift. Till en början så hade vi ett angreppssätt där vi inte alltid tog hänsyn till nivå och nödvändig teori, utan tog för givet att eleven med hjälp av nyfikenhet skulle vilja ta till sig den nya teori som skulle vara nödvändig för att lösa uppgiften. Vi fick inspiration från TED Talks och liknande där talare som forskar inom området gick igenom hur de ser på problemlösning och vilka aspekter de anser vara värdefulla. Ju mer kontakt vi fick med lärare ju mer insåg vi att vi dessvärre var tvungna att i någon mån anpassa våra problem för att de skulle passa in i deras undervisning, annars skulle de inte ha möjlighet att testa dem, och de skulle i förlängningen inte heller ha varit särskilt hjälpsamma för lärare över huvud taget. Vi bestämde oss därför för att avgränsa oss till kortare uppgifter för att underlätta testning, medan vi egentligen var ute efter att utforma både kortare och längre problem. I vissa fall kan det nämligen vara bra att få gott om tid för att fundera, och även ha tid till att testa olika metoder för att se vilken som fungerar bäst.}

\textcolor{Mahogany}{Med det sagt så har vi alltså begränsat oss till kortare problem om ca 50 minuter. Hade vi haft mer tid och möjlighet hade vi troligtvis även velat testa längre, mer projektliknande problem. Eftersom nästan hela projektgruppen tidigare läst kursen \textsl{Mathematical modelling and problem solving}\cite{matmod} där man fick veckovisa uppgifter, samt en som sträckte sig över merparten av kursen, så är vi också medvetna om vikten av att reflektera kring problem över en längre tid. Detta gör det möjligt att få en djupare förståelse för något än om man löser ett problem under en lektion.}

\textcolor{lila}{I avsnitt~\ref{sec:Skapandetavproblem} nämndes de viktigaste faktorerna som vi har försökt inkludera i problemen. Som tidigare nämnts appliceras dock inte alla dessa faktorer på samtliga problem, men ett eller flera ska ändå finnas som grundtanke bakom varje skapat problem. Den kanske viktigaste av dessa faktorer är att eleverna använder ett \textsl{undersökande arbetssätt}, vilket vi anser att vi har inkluderat i samtliga av våra problem. I de flesta problemen lägger detta grunden för att lösa de givna frågeställningarna medan det i problemet med den matematiska modellen, som presenteras i avsnitt~\ref{sec:lopare}, snarare dyker upp i den avslutande diskussionen. I samma problem tas även begreppet \textsl{matematisk modell} in. Eleverna får här upptäcka hur matematiska modeller kan användas och även reflektera över dess begränsningar.} 

\textcolor{lila}{Om man tittar på ''Fermiproblem'' (\ref{sec:Fermi}), ''Flygplan'' (\ref{sec:Flygplan}), ''Fritt fall'' (\ref{sec:FrittFall}), ''Matematisk modell för bil och löpare'' (\ref{sec:lopare}) och ''Personnummer'' (\ref{sec:Personnummer}) finns där också en mycket tydlig \textsl{verklighetsanknytning}. Problemen vid namn ''Sortering av kortlek'', (\ref{sec:Sortera}) ''Sorteringsalgoritmer'' (\ref{sec:sorteringsalgoritmer}) och ''Skapa ett chiffer'' (\ref{sec:chiffer}) har också en förankring i verkligheten även om denna är någon mindre tydligt än för ovanstående exempel. I dessa uppgifter har vi dock inkluderat information om hur de används i verkligheten, och uppmuntrar även läraren till att berätta om detta. Övriga uppgifter är lite mer abstrakta. }

\textcolor{lila}{Vi har också arbetat för att inkludera programmering i problemen. Detta syns tydligt i programmeringsproblemen, det vill säga problem~\ref{sec:sorteringsalgoritmer} till och med~\ref{sec:chiffer}. I problemet ''Sortera en kortlek'' (\ref{sec:Sortera}) finns programmeringen också med, om än något i bakgrunden. Här får eleverna testa på att tänka som en dator, och även ge varandra instruktioner på liknande form som man skulle göra i en programkod. På så sätt är detta tänkt som en bra introduktion innan man börjar med programmeringssyntax i ett specifikt språk.}