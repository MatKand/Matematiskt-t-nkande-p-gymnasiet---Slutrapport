\textcolor{Mahogany}{Att ta fram problem har inte varit en helt villkorslös uppgift. Till en början så hade vi ett angreppssätt där vi inte alltid tog hänsyn till nivå och nödvändig teori, utan tog för givet att eleven med hjälp av nyfikenhet skulle vilja ta till sig den nya teori som skulle vara nödvändig för att lösa uppgiften. Vi fick inspiration från TED Talks och liknande där talare som forskar inom området går igenom hur de ser på problemlösning och vilka aspekter de anser vara värdefulla. Ju mer kontakt vi fick med lärare ju mer insåg vi att vi dessvärre var tvungna att i någon mån anpassa våra problem att passa deras undervisning. Vi bestämde oss för att avgränsa oss till kortare uppgifter för att underlätta testning, medan vi egentligen var ute efter att både utforma kortare och längre uppgifter, då vissa uppgifter helt enkelt skulle kräva mer tid.}

\textcolor{Mahogany}{Med det sagt så har vi alltså begränsat oss till kortare problem med små grupper. Hade vi haft mer tid och möjlighet att testa problem som förslagsvis hade kunnat vara veckovisa. Eftersom nästan hela projektgruppen tidigare läst kursen \textsl{Mathematical modelling and problem solving}\cite{matmod} där man fick veckovisa uppgifter, samt en som sträckte sig över merparten av kursen, så är vi också medvetna om vikten av att reflektera kring problem över en längre tid. Detta gör det möjligt att få en djupare förståelse för något än om man löser ett problem under en lektion.}