% Urval och svarsfrekvens vi fick för enkäten, och hur vi kunde ha gått tillväga för att få fler svar. Det sistnämna kanske kan vara en egen punkt.

\textcolor{Mahogany}{När vi under projektets tidigare stadie sökte lärare som vi kunde testa problemen genom så hade vi inte i åtanke någon speciell metod att hitta dessa samt hur datan skulle påverkas. Detta gjorde att vi när det började bli dags att testa problemen istället försökte nå ut till så många som möjligt, vilket blev via en facebook-sida, vilket vi nämner i \ref{sec:Bakgrunsenkat}. Denna metod, eller brist på metod, har en del nackdelar. Till att börja med så har vi ingen större möjlighet att undersöka svarsfrekvensen, och vi vet egentligen inte så mycket om spridningen på lärare. Det kan likaväl vara enbart de mest engagerade lärarna som valde att anmäla intresse.}

% Osäker på mycket som jag skriver här under, utgår från vad jag minns, eller tror mig minnas.
\textcolor{Mahogany}{Det vi hoppades på i projektets början var att direkt kunna medverka under en lektion och vara med i utförandet, och kunna delta i diskussionen. Vi insåg dock att detta skulle begränsa vår testning i den mån att det skulle bli svårt att få spridning på vår data.}