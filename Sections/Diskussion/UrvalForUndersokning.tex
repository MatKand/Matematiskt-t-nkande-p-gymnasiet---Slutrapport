% Urval och svarsfrekvens vi fick för enkäten, och hur vi kunde ha gått tillväga för att få fler svar. Det sistnämna kanske kan vara en egen punkt.

\textcolor{Mahogany}{
    I början av projektet så var en tanke att direkt kunna medverka under en lektion och vara med i utförandet av ett problem. Vi ville bland annat kunna delta i diskussionen och i någon mån styra utförandet. Vi insåg dock att detta skulle begränsa vår testning, att det skulle bli svårt att få spridning på vår data. 
    Detta gjorde att siktade på att få en så stor spridning som möjligt, och detta via en Facebook-sida, som nämns i \ref{sec:Bakgrunsenkat}. Nackdelarna med detta blev
    %När vi under projektets tidigare stadie sökte lärare som vi kunde testa problemen genom så hade vi inte i åtanke någon speciell metod att hitta dessa. Vi tänkte inte heller så mycket på hur stor spridning vi skulle få på datan. Detta gjorde att vi när det började bli dags att testa problemen istället försökte nå ut till så många som möjligt, vilket blev via en facebook-sida, som vi nämner i \ref{sec:Bakgrunsenkat}. Denna metod, eller brist på metod, har en del nackdelar.
    att vi inte hade någon möjlighet att undersöka svarsfrekvensen. Vi vet egentligen heller inte så mycket om spridningen på både typ av lärare och klass. Det kan likaväl varit enbart de mest engagerade lärarna som valde att anmäla intresse. Deras klasser i sin tur kan bestå av elever som nödvändigtvis inte behöver extra pedagogiskt stöd. På så sätt blir det svårt att undersöka hur exempelvis gruppdynamiken fungerade samt ifall denna arbetsmetod även fungerar för de med extra behov.
}