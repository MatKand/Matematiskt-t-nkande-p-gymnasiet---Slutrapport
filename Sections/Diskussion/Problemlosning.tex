% Om det finns något här som ni vill ha med i rapporten så är det bara att plocka, men personligen så kände jag att det här inte längre tillförde något.

\textcolor{Mahogany}{
    Från svaren på en av våra undersökningsenkäter så märkte vi att begreppet problemlösning skiljde sig från lärare till lärare, där vissa kände att räkning i boken var problemlösning. Det är dock inte ett helt självbeskrivande begrepp som med fördel kan förtydligas.
}

\textcolor{Mahogany}{
    Även om problemlösning är en del av kursplanen i matematik så är det tydligt att det skiljer sig i vilken utsträckning och form det utövas. Som nämnts i \ref{sec:Forandringar} så ska problemlösning vara både ett mål och medel i undervisningen enligt kursplanen för gymnasiematematik, dessvärre så förlitar sig lärare i stor utsträckning enbart på de problem som kursboken tillhandahåller. Därtill blir den typen av problem ofta enbart extrauppgifter för de som hinner \cite{2010UndervisningenGymnasieskolan}. Det är alltså rimligt att anta att problemlösning i gymnasiematematik är bristfällig samt att lärare behöver fler verktyg och resurser för att uppnå dessa mål.
}

\textcolor{Mahogany}{
    Vi tror att en stor del handlar om att låta elever få diskutera med varandra, men utformningen av rätt typ av problem är givetvis också viktigt. Det ska inte mynna ut till att handla om att ta ur värden och sedan applicera dessa på aktuell teori som nyss gåtts igenom. Men det gäller också att ge lärare resurserna och samtidigt utmana deras uppfattning om vad problemlösning är, för att på så sätt kunna öka andelen av vad vi ser som problemlösning.
    Kanske är en stor anledning att det inte är mer problemlösning i undervisningen att lärare i olika utsträckning är osäkra på hur de ska genomföra en sådan undervisning, med en kombination av att inte ha rätt resurser. Frågan är om man uteslutande kan förlita sig på att enbart kursböckerna ska täcka denna del av kursplanen, men samtidigt krävs tid och resurser om man vill förbättra detta, något som många lärare känner att de inte har.
}