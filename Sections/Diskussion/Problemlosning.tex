\textcolor{Mahogany}{Från svaren på ett av våra undersökningsenkäter så märkte vi att begreppet problemlösning skiljde markant från lärare till lärare, där vissa kände att räkning i boken var problemlösning och andra hade tidigare redan försökt arbeta med liknande arbetssätt som vi försöker framhäva. Det är dock inte ett helt självtalande begrepp som tåls att förtydligas. Även om problemlösning är en del av kursplanen i matematik så är det tydligt att det skiljer sig i vilken utsträckning och form det utövas. Som nämnt i \ref{sec:Forandringar} så ska problemlösning vara både ett mål och medel i undervisningen enligt kursplanen för gymnasiematematik, dessvärre så förlitar sig lärare i många fall enbart på de problem som kursboken tillhandahåller och den typen av problem blir endast extrauppgifter för de som hinner\cite{2010UndervisningenGymnasieskolan}. Det är alltså rimligt att anta att problemlösning i undervisningen är bristfällig samt att lärare behöver fler verktyg för att uppnå dessa mål.}