\textcolor{Mahogany}{I början av projektet så var tanken att utveckla en slags lathund för lärare. Denna skulle underlätta för lärare att själva göra egna problem likt de som vi utformat under projektet. Detta då vi under projektet enbart kan utveckla ett begränsat antal problem, och helt enkelt för att lärare ska kunna fortsätta att arbeta med liknande undervisningsmetod på egen hand.}

\textcolor{Mahogany}{Utmed projektet så mynnades detta ut istället till att bli \textsl{Information till läraren}, som nämns i \ref{sec:Skapandetavproblem}. Denna information är mer specifikt information om just de problem som vi utformat och hur de bör utföras. Vi försöker bland annat att framhäva de svårigheter som denna typ av undervisning kan medföra, och försöker ge tips på hur man kan hantera detta. Istället för att komma fram till någon slags formel för hur lärare själva ska kunna utforma egna liknande problem, så lade vi istället fokus på att beskriva utförandet. Där ger vi förslag på hur utförandet \textsl{kan} se ut och vilka möjliga fördjupningsfrågor man kan komplettera med.}