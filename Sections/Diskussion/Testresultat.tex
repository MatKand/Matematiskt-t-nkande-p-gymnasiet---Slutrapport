\textcolor{lila}{Även om $100\%$ självklart hade varit fantastiskt, är vi ändå väldigt nöjda över resultatet att hela $80\%$ av eleverna ansåg sig lära sig något av problemet och att $75\%$ tyckte att det var roligt eller meningsfullt. Det var också tydligt att de flesta, det vill säga hela $75\%$ ville arbeta med problemlösning i skolan, och inte enbart med boken\todo{Ida ska fortsätta jobba här imorgon}.}

\textcolor{Mahogany}{I avsnitt \ref{resultat:Programmering} sammanfattas responsen från testningen av programmeringsproblemen. Generellt så fick de väldigt god respons, men den kanske största nackdelen var att det framgick att alla inte kände att de klarade av det. Läraren spekulerade vidare att det skulle uppstå behov för de eleverna att arbeta i grupp. Detta är dock inte nödvändigtvis något negativt, kanske rentav bra. Det är också något som borde varit tydligare i problembeskrivningen, det vill säga att elever uppmuntras till att arbete i par vid behov. Så kallad parkodning är väldigt vanligt på laborationer och tillåter elever att diskutera lösningsmetoder och resonera sig fram tillsammans. Hur som helst så framgick det att problemen var väldigt passande för de som sökte lite mer avancerade uppgifter. Det vore givetvis önskvärt att även dessa problem kunde anpassas i svårighetsgrad. Alternativt skulle man kunna tillhandahålla kodskal som är delvist implementerade, eller helt enkelt ge ledtrådar som hjälper eleven på vägen till en lösning. Risken med detta är att man på det här sättet utarmar problemet en aning och att eleven inte får samma insikter och lärdomar. Om man istället bara är tydlig med att elever som känner sig osäkra får arbeta med någon annan så behövs rimligtvis inte stöd för de som känner att de inte klarar av att lösa problemet på egen hand.}