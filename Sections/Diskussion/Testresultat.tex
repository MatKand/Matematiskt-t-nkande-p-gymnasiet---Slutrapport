\textcolor{lila}{Detta avsnitt diskuterar testresltaten, dels utifrån hur enkätsvaren kan tolkas och dels den metod som användes, vilka konsekvenser det ger och tankar kring andra metoder som skulle ha kunnat användas.}
    
    \subsubsection{Våra generella tankar kring resultatet}
    \textcolor{lila}{Totalt sett är vi väldigt nöjda  över de resultat vi fått om utvärdering av problemen. Även om $100\%$ självklart hade varit fantastiskt, är vi ändå väldigt nöjda över resultatet att hela $80\%$ av eleverna ansåg sig lära sig något av problemet och att $75\%$ tyckte att det var roligt eller meningsfullt. Det var också tydligt att de flesta, det vill säga hela $75\%$, ville arbeta med problemlösning och inte enbart med boken. Detta antyder att de flesta av de elever som svarade  på enkäten var nöjda med det problem de fick genomföra. Alla lärare har också till största del varit nöjda med sina problem och tyckt att de presenterats på ett tydligt sätt som gjort det lätt för dem att snabbt sätta sig in i hur det skulle utföras. Denna åsikt speglades även av de flesta av de lärare som anmält intresse med som sedan tackade nej på grund av tidsbrist.}

\subsubsection{Fermiproblem}
    \textcolor{lila}{När det gäller bemötandet av problemet ''Fermiproblem'' (\ref{sec:Fermi}) verkade de olika frågeställningarna engagera eleverna, och få dem att reflektera över rimligheten i ett svar. Vi tror även att möjligheten att välja uppgift gav dem en känsla av äganderätt till problemet, vilket enligt avsnitt~\ref{sec:delaktighet} är viktigt för inlärningen. Det var också roligt att höra att planeringen utvidgades med att läraren berättade om Fermis intressanta liv, samt att eleverna hittade information om Fermiparadoxen på nätet och började diskutera detta.}

    \textcolor{lila}{Både läraren och de flesta elever har angett att problemet var lärorikt. En elev tyckte att det var för lite given data, vilket tyvärr tyder på att eleven missuppfattat vad som skulle göras. Del av målet med uppgiften var att man skulle träna på att uppskatta värden och inse att det går att göra en bra  approximation av ett komplicerat svar utan givna data. Detta kan vara något vi behöver vara tydligare med i informationen till problemet, för att öka chansen att läraren verkligen trycker på detta i samband med att problemet presenteras för eleverna. En elev angav att det var svårt för att det blev många uträkningar, så vi ska förtydliga i uppgiften att man bör låta eleverna använda miniräknare till detta, eftersom svåra uträkningar knappast bidrar till att eleverna uppnår målen med problemet. Det var fantastiskt roligt att läsa lärdomar som att ''hur enkel matte kan användas för att lösa problem och ur uppdelning kan förenkla'', eftersom detta visar att vi i alla fall till viss del uppnått det vi ville med uppgiften. Andra elever angav att de lärt sig om vad ett fermiproblem är eller hur många hjärtslag ett hjärta slår. Dessa lärdomar speglar inte våra mål med problemet på samma sätt, men vi hoppas att dessa elever ändå kan komma ihåg det här i framtiden. Även om de kanske inte kommer lägga sina exakta svar på minnet kanske de kommer ihåg att de räknat ut något som de tyckte var häftigt, och kanske kommer de även kunna komma ihåg ungefär hur de gjorde.}

\subsubsection{Försvåring av en ekvation}
    \textcolor{lila}{Klassen som testade ''Försvåring av en ekvation'', vars enkätsvar presenteras i avsnitt~\ref{resultat:ekvation}, tyckte att det var roligt att försvåra ekvationen, men hade svårigheter med prioriteringsreglerna. Vi hade redan varnat för detta i informationen som hörde till problemet, men kanske borde  detta förtydligas ytterligare. Förhoppningsvis tillförde det också en ytterligare faktor, eftersom eleverna även fick träna på och lära sig mer om detta. Läraren säger också att han upplevde att problemet gav eleverna en djupare förståelse för ekvationer.}
    
    \textcolor{lila}{De elever som svarade var också spridda i den mening att vissa tyckte att problemet var svårt, lätt eller lagom. Vi antar att detta beror på att de hade svårt med prioriteringslagarna, eftersom problemet för övrigt är tänkt att kunna anpassas till varje elevs individuella nivå. Detta genom att eleverna själva får välja med vilka medel och matematiska operationer som de ska använda för att försvåra ekvationen. En elev angav också att lektionen blev väldigt rörig. Till viss del är vi glada att lektionen frångick den tysta matematiklektionen utan diskussioner, eftersom det verkar som att det här blev en lektion som gav mer liv och frihet till matematiken. Däremot är det självklart synd att det gick så långt att någon tyckte att det var jobbigt.}

\subsubsection{Matematisk modell för bil och löpare}
    \textcolor{lila}{Avsnitt~\ref{sec:lopare} presenterar resultatet för ''Matematisk modell för bil och löpare''. Detta problem testades av två olika lärare, varav tyvärr bara en klass svarade på enkäten. Den ena läraren ansåg att den matematiska delen av problemet var för lätt. Vi kan hålla med om denna åsikt, eftersom tanken med dessa inledande uppgifter främst var att ge ett underlag för diskussionen om matematiska modeller och vilka fördelar och brister det finns att tänka på innan man använder dem. Läraren har också skrivit att hen tyckte att denna diskussion var väldigt givande. Den andra läraren var väldigt nöjd med hela problemet, och svårigheten i den klassen var att omvandla mellan enheterna m, km och mil. Det är tråkigt att se att detta är en såpass stor svårighet på gymnasiet, men vi hoppas att problemet i så fall gav en tankeställare om att dessa grundläggande kunskaper är viktiga.}

    \textcolor{lila}{Båda lärarna valde också att utveckla problemet genom att skissa varsin graf. För den klassen som tyckte att uppgiften varit relativt svår skissades den linjära modellen, och den klass som tyckte att uppgiften varit lätt gjorde istället ett gemensamt försök att skissa den verkliga modellen för löparen. På så sätt anpassade varje lärare problemet efter klassens nivå, vilket vi tror var mycket bra. Båda lärarna tror att det som eleverna framförallt tog med sig från lektionen var att man måste fundera över rimligheten i sina svar och metoder, vilket var precis det vi ville få fram med problemet.}

    \textcolor{lila}{De flesta av eleverna, närmare bestämt $75\%$, tyckte att de lärt sig något av uppgiften och att det var bra att få ''jämföra med verkligheten och tänka på rimligheten och relevansen. Detta var roligt att höra, eftersom detta var vad vi framförallt ville att eleverna skulle få ut av uppgiften. En av de som angett att hen inte lärde sig något sa även att det berodde på att hen genomförde uppgifterna på enklast möjliga sätt och att det hade blivit svårare om hen hade funderat på relevansen. Detta tyder på att eleven faktiskt ändå lärde sig något. Kanske har eleven inte tagit med den efterföljande diskussionen i sin reflektion här, utan enbart själva beräkningsuppgifterna. }

\subsubsection{Programmeringsproblemen}
    \textcolor{Mahogany}{I avsnitt \ref{resultat:Programmering} sammanfattas responsen från testningen av programmeringsproblemen, varav de som testades fick  väldigt god respons. Det framgick också att Niklas trodde att vissa elever hade tyckt att problemen var svåra och hade behövt arbeta i grupp för att kunna klara att lösa dem. Detta är dock inte nödvändigtvis något negativt, utan rentav bra, se avsnitt~\ref{sec:Arbetaigrupp}. Niklas valde dock här att frångå vårt förslag om grupparbete och istället ge dessa problem till högpresterande elever, och även att de skulle arbeta med problemen var och en för sig. Annars är så kallad parkodning väldigt vanligt på laborationer och tillåter elever att diskutera lösningsmetoder och resonera sig fram tillsammans och det var därför detta som vi rekommenderade. Hur som helst så framgick det att problemen var väldigt passande för de som sökte lite mer avancerat material, men att Niklas tror att det skulle ha varit utvecklande även för andra elever, så länge de fått jobba i par.}

    \textcolor{Mahogany}{Eventuellt vore det även önskvärt att dessa problem skulle kunna anpassas i svårighetsgrad. Alternativt skulle man kunna tillhandahålla kodskal som är delvist implementerade, eller helt enkelt ge ledtrådar som hjälper eleven på vägen till en lösning.} \textcolor{lila}{Detta material skulle på så sätt kunna fylla en roll som en lärare som är oerfaren inom programmering inte klarar, vilket kan bli ett problem när programmering förhållandevis hastigt ska inkluderas i undervisningen.} 
    \textcolor{Mahogany}{Risken med detta är dock att man på det här sättet utarmar problemet en aning och att eleven inte får samma insikter och lärdomar,} 
    \textcolor{lila}{eftersom det är omöjligt att få samma lyhördhet som från en bra lärare som kan lyssna in eleverna och ställa lämpliga och vägledande frågor. Dessa problem behöver alltså en hel del mer arbete om de ska kunna användas av lärare utan programmeringskunskaper.} 
    \textcolor{Mahogany}{Men kanske är det som sagt även tillräckligt att eleverna arbetar i par och får fundera och diskutera tillsammans.}
    
\subsection{För- och nackdelar med den testmetod som användes}
    \textcolor{green}{Som nämnts i avsnitt~\ref{sec:HurTestetGjordes} valdes lärarna som fick testa problemen ut genom att de fick anmäla intresse via den bakgrundsenkät som skickades ut via en facebookgrupp för matematikundervisning.} 
    \textcolor{lila}{Tänkbara följder av detta diskuterades ovan i avsnitt~\ref{Disk:Bakgrundsenat}. Till skillnad från bakgrundsenkäten, till vilken vi fick många svar, var det här totalt fem lärare som testade något av våra problem. Detta gör tyvärr att resultatet inte kan anses helt statistiskt tillförlitligt, speciellt i de fall då det var mycket få elever som svarade på enkäten. Vi har inte heller någon information om hur många elever som verkligen deltog i varje problem, utan enbart hur många som svarade på enkäten. Trots denna osäkerhet kommer vi ändå att diskutera de svar vi har fått, och hur de kan tolkas.}
    
    
    \textcolor{green}{En nackdel med den metod som användes är att vi inte haft någon möjlighet, utöver korta frågor på enkäten, att kontrollera hur problemet genomfördes. Lärarna har som tidigare nämnt även haft relativt fria tyglar att förmedla uppgifterna på det sätt de fann lämpliga}
    \textcolor{Mahogany}{En annan tänkbar metod för genomföradet hade varit att vi hade medverkat under lektionerna och deltagit aktivt i utförandet av varje problem. Vi hade då kunnat forma utförandet till en mycket högra grad}
    \textcolor{green}{och eventuellt till och med kunnat testa uppgifterna på precis samma sätt med många olika klasser. Datan vi sedan hade samlat in hade därmed varit mer kvalitativ, eftersom vi hade varit konsekventa med tillvägagångssättet vi testade uppgifterna på. Vi hade även kunnat kontrollera att alla elever svarade på enkäten. En medelväg hade kunnat vara att delta på lektioner när lärarna testade problemen. Detta hade nog varit det mest optimala alternativet då vi hade haft möjlighet att se lärarna arbeta i sin naturliga miljö. Vi hade även kunnat jämföra hur olika lärare testar problemen för att se för- och nackdelar. Detta hade sedan kunnat användas till att bland annat förbättra den övriga informationen som var avsedd till lärarna angående undervisningsmetodiken gällande våra uppgifter.}
    
    \textcolor{green}{Nackdelen med förslagen ovan är att de hade krävt mer organisation mellan oss och lärarna. Inget av dem hade heller löst det huvudsakliga problemet att det var få lärare som hade möjligheten att testa ett problem.} 
    \textcolor{lila}{Fördelen med den valda metoden jämfört med dessa är även att utförandet nu efterliknande hur vi hoppas att problemen ska användas efter projektets slut. Det vill säga att lärarna ska kunna hitta problemen på vår webbplats, använda givna instruktioner och information till den grad de anser lämpligt och sedan kunna genomföra problemet.}