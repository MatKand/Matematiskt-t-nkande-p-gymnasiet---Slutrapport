\textcolor{lila}{Bakgrundsenkäten, som presenteras i avsnitt~\ref{sec:bakgrundsenkat}, gjordes för att undersöka om den uppfattning vi hade om matematikundervisning stämde överens med verkligheten. Enkäten skickades som tidigare nämnt ut via en facebooksida för matematikundervisning. Detta gjordes för att nå ut till så många lärare som möjligt, vilket lyckades, men det innebar också att vi i första hand riktade oss mot en viss typ av lärare. Mer specifikt de lärare som har facebook och är intresserade av att samlas för att läsa eller delta i diskussioner om matematikundervisning. Vi bad även lärarna om hjälp med att sprida enkäten, men det är omöjligt att säga till vilken grad detta verkligen skedde. Metoden gör också att vi inte kan beräkna någon typ av svarsfrekvens.}

\textcolor{lila}{Däremot fungerade metoden över förväntan på det sättet att vi fick in många svar. Man kan också se att många lärare ansåg att även om $91\%$ av lärarna svarade att de arbetade för att inkludera problemlösning i sin undervisning, så nämnde de också många faktorer som hindrade detta arbete. Så även om vi antar att detta är den mest motiverade delen av lärarkåren, som arbetar hårdast med problemlösning, så \textsl{finns det ändå ett problem}.}

\textcolor{lila}{Enkäten visade också att de flesta av dessa lärare hade en mycket god bild av vad problemlösning är. Detta kan kanske kännas självklart, men det fanns också svar som antydde att det finns lärare som inte känner till en bra definition av vad problemlösning faktiskt är. Till exempel kan nämnas att vi reagerade mycket starkt på ett av svaren, som enbart löd ''Textuppgifter''. Även om problemuppgifter ofta är textuppgifter är detta inte ett krav, och alla textuppgifter är definitivt inte problemlösning. Som tur är var det bara en av 56 som uttryckte sig på det här sättet, men det är ändå en för mycket.}

\textcolor{lila}{Det största hindret mot mer problemlösning i undervisningen som framkom från bakgrundsenkäten var olika typer av tidsbrist. Dels framkom åsikter om att problemlösningsbaser undervisning tar mer tid från själva lektionerna och dels att de tar längre tid att planera än att låta eleverna räkna färdiga uppgifter från boken. Det här anges som tid man inte har efter att allt övrigt material har gåtts igenom, för att eleverna ska kunna prestera på nationella provet. Och självklart är matematikens grundläggande teori viktig, men vad betyder den om man aldrig får träna på att använda den? Vi anser att vissa delar av kraven i de olika kurserna borde nedprioriteras, och att problemlösning bör få en tydligare roll i undervisningen av matematik.} 

\textcolor{lila}{Idag är problemlösning något som till alltför stor del lämnas som extrauppgifter till de elever som har fallenhet för matematik, men vi anser att det är andra delar av matematiken som kan hanteras på detta sätt. Att man bör utforska matematiken med problemlösning som främsta verktyg, och låta eleverna själva upptäcka de begrepp och metoder som då dyker upp. Och de delar av matematiken som inte går att hitta på ett rimligt problem om, antingen hörande till nutida eller dåtida behov eller intressen, kanske helt enkelt inte kan vara så centrala i alla fall.}