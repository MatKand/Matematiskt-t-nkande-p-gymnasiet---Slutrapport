\textcolor{lila}{Bakgrundsenkäten, som presenteras i avsnitt~\ref{sec:Bakgrundsenkat}, gjordes för att undersöka om den uppfattning vi hade om matematikundervisning stämde överens med verkligheten.} 
\textcolor{Mahogany}{Enkäten skickades ut via en Facebook-sida för matematikundervisning. Detta gjordes för att nå ut till så många lärare som möjligt, vilket lyckades, men det innebar också att vi i första hand riktade oss mot en viss typ av lärare.}
\textcolor{lila}{Mer specifikt de lärare som använder Facebook och är intresserade av att samlas för att läsa eller delta i diskussioner om matematikundervisning. Vi bad även lärarna om hjälp med att sprida enkäten, men det är omöjligt att säga till vilken grad detta verkligen skedde. Metoden gör också att vi inte kan beräkna någon typ av svarsfrekvens.}

\textcolor{lila}{Däremot fungerade metoden över förväntan på det sättet att vi fick in många svar. Man kan också se att många lärare ansåg att även om $91\%$ av lärarna svarade att de arbetade för att inkludera problemlösning i sin undervisning, så nämnde de också många faktorer som hindrade detta arbete. Så även om vi antar att detta är den mest motiverade delen av lärarkåren, som arbetar hårdast med problemlösning, så \textsl{finns det ändå ett problem}.}

\textcolor{lila}{Enkäten visade också att de flesta av dessa lärare hade en mycket god bild av vad problemlösning är. Detta kan kanske kännas självklart, men det fanns också svar som antydde att det finns lärare som inte känner till en bra definition av vad problemlösning faktiskt är. Som exempel kan nämnas att vi reagerade mycket starkt på ett av svaren, som enbart löd ''Textuppgifter''. Även om problemuppgifter ofta är textuppgifter är detta inte ett krav, och alla textuppgifter är definitivt inte problemlösning. Som tur är var det bara en av 56 som uttryckte sig på det här sättet, men det är ändå en för mycket.}

