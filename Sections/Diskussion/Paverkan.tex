\textcolor{lila}{Som nämndes i avsnitt~\ref{sec:slutenkat} så var det under utvärderingen }\textcolor{Mahogany}{inte ovanligt att lärare kände att de inte hade tid att testa våra problem på grund av annat i kursplanen så som nationella prov. Vi hoppas dock att även om lärare just nu inte kan utvärdera våra problem att vi ändå lyckats förmedla nyttan med denna typ av problemlösning och det vi försöker uppnå, dvs att matematik ska handla mer om den undersökande delen som till större del bygger på diskussion och rimlighetsanalyser.}

\textcolor{Mahogany}{I \ref{sec:MerProblemlosning} så framgår det att många lärare påpekar att de anser problemlösning vara viktigt, men samtidigt så känner lärare att de inte kan arbeta med det på grund av tidsbrist, men också för att de inte vet hur man skulle göra det på ett bra sätt. De tyckte bland annat att det var svårt att hitta bra problem, som ska vara utmanande för hela klassen. I utformandet av problem så försökte vi i någon mån att anpassa problemen efter vissa nivåer, kanske främst för testningssyften då vissa lärare uttryckte intresse av att testa problem som specifikt skulle passa in på en vis matematikkurs. Vi hade egentligen som utgångspunkt att utforma \textit{bra} problem, utan hänsyn till vilken nivå den hamnar på. Samtidigt kan det vara bra att ta försöka få in vissa element som gör att man kan fånga upp så stora delar av gymnasiematematiken som möjligt, även om fokus fortfarande ligger på problemlösning och inte att nödvändigtvis täcka så mycket teori som möjligt.}

\textcolor{Mahogany}{Även om problemlösning och ett mer undersökande angreppssätt till matematik ska vara en del av GY11, så finner lärare det svårt med övergången från traditionell undervisning, vilket nämns i \ref{sec:MerProblemlosning}. Eftersom det kan vara svårt att enbart utgå från de problemlösningsuppgifter som erbjuds i matematikboken så känns det som ett utmärkt komplement att man kan ta del av utformade problem för gymnasiet externt från blivande ingenjörer. Också är problemet med att lärare i omfattande utsträckning har svårt att faktiskt utföra problem med mer problemlösning anledningen till att vi har valt att vara så utförliga som möjligt med instruktioner och förslag på utförande samt innehåll med de problem som utformats. Samtidigt har vi försökt understryka att det är de ska känna att de kan styra innehållet som de vill mer eller mindre vad gäller fördjupning och hur mycket information som ska ges etc.}

\textcolor{Mahogany}{Vi hoppas att även om det är en småskalig inverkan så kan vi fortfarande hjälpa till att förändra bilden kring matematik och att elever ska känna att det är något kreativt och undersökande snarare än att handla om memorering och upprepning.}