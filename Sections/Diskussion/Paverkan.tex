% \textcolor{Mahogany}{
%     Som nämndes i avsnitt~\ref{sec:slutenkat} så var det under utvärderingen inte ovanligt att lärare kände att de inte hade tid att testa våra problem på grund av annat i kursplanen så som nationella prov. De var dock intresserade, och de flestakommenterade att problemen och uppläggen verkade bra. Vi hoppas därför att även om många lärare just nu inte kan utvärdera våra problem att vi ändå lyckats förmedla nyttan med denna typ av arbetssätt och att matematik bör handla mer om den undersökande delen som till större del bygger på diskussion och rimlighetsanalyser.
% }

\textcolor{Mahogany}{
    I \ref{sec:MerProblemlosning} så framgår det att många lärare påpekar att de anser problemlösning vara viktigt. Samtidigt känner de att de inte kan arbeta med det till den grad de skulle vilja på grund av tidsbrist, både på och inför lektionerna, och för att de inte vet hur man skulle göra det på ett bra sätt. De tyckte bland annat att det var svårt att hitta bra problem som är utmanande för hela klassen. %I utformandet av våra problem så försökte vi i någon mån att anpassa problemen efter vissa nivåer, kanske främst för testningssyften då vissa lärare uttryckte intresse av att testa problem som specifikt skulle passa in på en viss matematikkurs. Vi hade egentligen som utgångspunkt att utforma \textsl{bra} problem, utan hänsyn till vilken nivå den hamnar på. Samtidigt kan det vara bra att ta försöka få in vissa element som gör att man kan fånga upp relevant teori, eller efterfrågan av den, även om fokus fortfarande ligger på problemlösning.
}

    \textcolor{lila}{Med den fullständiga lektionsplanering som följer med hoppas vi kunna avlasta lärarna från en del av det förberedande arbete som krävs för att genomföra en problemlösningslektion, och därmed förbättra deras arbetssituation. Med den brist på lärare som råder idag är det viktigt ur ett samhällsperspektiv att spara på de resurser vi har, för att kunna utnyttja deras fulla potential genom att de får finnas till hands och handleda problemlösningen. Bristen av ren lektionstid är svårt för oss att påverka, men dels har vi även tryckt på den bevisade nyttan av problemlösning och dels hoppas vi som sagt att detta ska hjälpa med en del av de nuvarande faktorer som motverkar mer problemlösning i undervisningen.}
    
    \textcolor{lila}{Även om man arbetar hårt för att ändra matematikundervisningen, och utvecklingen enligt oss är på väg åt rätt håll, så tror vi att det finns mer som kan göras för att hjälpa den på traven. Vi vill därför arbeta för att hjälpa lärarna att få eleverna att gå från att beskriva matematiken som ''tråkig'' och ''onödig'' och istället börja använda ord som ''rolig'', ''spännande'' och ''användar''. Kanske till och med ''kreativ''.}
    
% \textcolor{Mahogany}{
%     Vi hoppas att även om det är en småskalig inverkan så kan vi fortfarande hjälpa till att förändra bilden kring matematik så att den inte ska förknippas med ord som och att elever ska känna att det är något kreativt och undersökande snarare än att handla om memorering och upprepning.}
    %\textcolor{lila}{Och även om detta bara är en liten del av den utveckling vi hoppas följer, så hoppas vi att matematikundervisningen utvecklas till en mer levande och kreativ process. I bästa fall leder detta till en ny generation av skickliga problemlösare, som kan säkra Sveriges roll i framtidens tekniska samhälle. 
%}