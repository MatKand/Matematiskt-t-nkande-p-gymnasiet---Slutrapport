\textcolor{Mahogany}{Under vårt projekt så har vi framhävt vikten av problembaserat lärande, eller \textsl{PBL}, som vi skrivit om i avsnitt~\ref{sec:pbl}. Faktum är att det finns svårigheter med även denna typ av undervisning, bland annat att elever istället riskerar att bli passiva, något som vi berör vi i avsnitt \ref{sec:Arbetaigrupp}. Där belyser vi att man bör undvika att göra gruppindelningen alltför homogen eller heterogen. Vi nämner även att grupperna inte heller bör vara för stora.}

\textcolor{Mahogany}{Vi nämner också i avsnitt~\ref{sec:pbl} att denna undervisningsmetod är mer resurskrävande, något som vi också fått belägg för i vår undersökning, som sammanfattas i \ref{sec:MerProblemlosning}. Många anger tidsbrist som anledning att de inte inkluderar problemlösning mer i sin undervisning. Det kan också vara svårt att få elever att känna att det är relevant, vilket framgick i vår intervju med Niklas Grip i avsnitt~\ref{sec:Niklassyn}.}

\textcolor{Mahogany}{Något som vi inte har undersökt i det här projektet är att uppskatta hur stor del av undervisningen som bör läggas på problemlösning.} \textcolor{turkos}{Daniel Willingham beskriver i sin bok \textsl{Why Students Don't Like School?} att för att kunna tänka som en expert så behöver man grundläggande kunskap, och det enda sättet att få den grundkunskapen är genom repetition \cite{WhyDontStudents}. }\textcolor{Mahogany}{Detta stärker givetvis behovet av vanlig mekanisk räkning, men samtidigt så förespråkar vi inte heller att fullständigt ersätta dagens matematik med problemlösning. Däremot så anser vi att det finns ett behov av problemlösning för att få en djupare förståelse för den teori man faktiskt ska lära sig, och att det i många fall inte är tillräckligt mycket problemlösning i matematikundervisning på gymnasiet. Vårt mål är helt enkelt att underlätta för lärare att inkludera mer av detta.}