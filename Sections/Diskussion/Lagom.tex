\textcolor{Mahogany}{I avsnitt~\ref{sec:pbl} så talade vi om PBL, problembaserat lärande. Faktum är att det finns svårigheter med även denna typ av undervisning, bland annat som vi nämner i \ref{sec:Arbetaigrupp}, att elever istället riskerar att bli passiva. Där belyser vi att man bör undvika att göra gruppindelningen alltför homogen eller heterogen. Vi nämner även att grupperna inte heller bör vara för stora.}

\textcolor{Mahogany}{Vi nämner också i avsnitt~\ref{sec:pbl} att denna undervisningsmetod är mer resurskrävande, något som också visade sig i vår undersökning, som sammanfattas i \ref{sec:MerProblemlosning}. Många anger tidsbrist som anledning att de inte inkluderar problemlösning mer i sin undervisning. Det kan också vara svårt att få elever att känna att det är relevant, vilket framgick i vår intervju med Niklas Grip i avsnitt~\ref{sec:Niklassyn}.}

\textcolor{Mahogany}{Något som vi däremot inte har undersökt i det här projektet är att uppskatta hur stor del av undervisningen som bör läggas på problemlösning.} \textcolor{turkos}{Som vi nämnde i avsnitt \ref{sec:pbl} så krävs det grundläggande kunskap för att kunna tänka som en expert, och det enda sättet att få den grundkunskapen är genom repetition. }\textcolor{Mahogany}{Detta belyser att det även finns ett behov av vanlig mekanisk räkning. Vi vill därför förtydliga att vi inte förespråkar att man fullständigt bör ersätta dagens matematik med problemlösning. Det vi vill få fra av den här rapporten är att problemlösning kan användas för att ge eleverna en djupare förståelse för de matematiska teorier som de lär sig. Detta har visat sig att det i många fall inte appliceras så mycket som skolverket och lärarna önskar. Vårt mål är helt enkelt att underlätta för lärare att inkludera mer problemlösning i sin undervisning, så att problemlösning på sikt kan ta sin roll som huvudsakliga verktyg för att låta eleverna upptäcka matematiken.}