\textcolor{turkos} {
Att låta elever arbeta med problemlösning i små grupper om två till fyra personer leder till att varje elev ges möjlighet att diskutera och reflektera angående problemet. Eleverna får prata om sina idéer till lösningar, lyssna på andra elevers idéer, samt även ges möjlighet att fråga, kritisera och bemöta kritik på ett positivt sätt. När eleverna får förklara sina tankar leder det till ökad matematiksförståelse. \cite{RikaProblem}
}

\textcolor{turkos} {
Som tidigare nämnts i \ref{sec:Gruppindelning} är det vanligt att skolor delar in elever i grupper efter deras matematikfärdigheter, dock hävdar både Boaler \cite{TheElephant} och Rika matematiska problem \cite{RikaProblem} att det är bättre med heterogena grupper vad det gäller matematikkunskap. I blandade grupper där man arbetar efter metoder anpassade för dem får elever lära av varandra vilket leder till ett mer rättvist klassrum enligt Boaler \cite{TheElephant}. 
}

\textcolor{turkos} {
Rika matematiska problem rekommenderar att läraren blandar medelpresterande elever med högepresterande eller lågpresterande elever när man gör gruppindelningen, men varnar samtidigt för att inte göra grupperna extremt homogena eller heterogena\cite{RikaProblem}.
}

% I blandade grupper hjälper elever varandra vilket leder till fina ord som equality och allt blir bättre - Jo Boaler

% Även Rikaproblem rekommenderar att man blandar medelpresterande elever med högepresterande eller lågpresterande elever när man gör grupp indelningen, men varnar samtidigt för att inte göra grupperna extremt homogena eller heterogena. 


\textcolor{turkos} {
En negativ aspekt med grupparbete är att vissa elever kan bli sittande passivt medan resten av gruppen löser uppgiften åt dem\cite{RikaProblem}. Detta problemet kan rimligen antas bli värre ju större gruppen blir. 
}

%Bygga en rödtråd till förra kapitlet genom Japanska skolan. 



% Problemlösning i grupp anses vara roligare än vanliga mattelektion - Skolverket

% Saknas något om olika gruppstorlekar, samt nackdelar

% En studie visar att 2 är bättre än 1 förrutom för de allra bästa eleverna där det är ungefär lika bra, men 3-5 är strängt bättre. Den säger även att testpersonerna hade något roligare när det arbetade individuellt.

%Negativt - Om man börjar med grupparbete så kan vissa elever bli sittande passivt medan resten av gruppen löser uppgiften åt dem. - Rikaproblem



% Grupper bör ligga på 2-4 elever, eftersom det i en sådan grupp ger alla elever möjlighet att delta i diskussioner - Rika problem 

% Gruppering efter färdigheter är vanligt, enligt skolverket. 