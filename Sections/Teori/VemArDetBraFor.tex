\textcolor{Mahogany}{Enligt en studie om problemlösning i grupp så har elever ett stort behov av att arbeta med problem som är på deras nivå, då de behöver vara lagom svåra för att behålla motivationen uppe, samtidigt som de ska vara utmanande \cite{undervisningviaproblemlosning}. Detta gäller både elever som har fallenhet för matematik likaväl som de som har inlärningssvårigheter. Vidare så visar studien att traditionella matematikuppgifter sällan erbjuder detta då uppgiftsformuleringarna brukar innehålla ''nyckelord'' som elever lär sig att identifiera, och man blir då som elev fråntagen möjligheten att få övning med att möta och tolka nya problem, vilket kan leda till att man inte lär sig att utveckla någon studieteknik.}

\textcolor{Mahogany}{Studien visar även att risken att välja problem av fel svårighetsgrad minskar vid val av problem vars lösning på förhand inte är uppenbar. Detta eftersom de kan lösas på olika abstraktionsnivåer, och att detta tillåter att elever med olika förutsättningar inte hamnar utanför den ordinarie klassundervisningen.}

\textcolor{Mahogany}{Problemlösning är att kunna arbeta med problem där framför allt tillvägagångssättet för att lösa problemet inte är uppenbart, samtidigt som det öppnar upp för olika lösningar för olika abstraktionsnivåer. Ytterligare så ska problemen uppmana till diskussion då detta är ett bra tillfälle för elever att motivera och försvara sina lösningar och med det förståelse.}