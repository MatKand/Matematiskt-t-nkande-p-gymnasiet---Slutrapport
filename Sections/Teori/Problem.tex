%\textcolor{Mahogany}{Vi definierar ett problem som en uppgift som man på förhand inte vet hur man ska lösa. }

% Vad är inte problemlösning
\textcolor{WildStrawberry}{
    Komplexiteten med att definiera vad ett problem är ligger i att det existerar olika syften med vad ett problem vill få ut från den som testas. Vissa problem vill att du hittar x - kan du hitta x? Andra problem kanske inte har ett exakt svar - behandlingen av problemet är det som är utvecklande. Det vi vill ta ställning för är att en ''lös ut x''-uppgift som är dekorerad i en \textit{saga} skapar inte ett intressant problem och kunde lika väl varit den vanliga ''vad är x''-övningen. 
}

% Vad är problemlösning
\textcolor{WildStrawberry}{
    Det finns en mängd olika infallsvinklar man kan ta för att definiera \textit{ett problem}. Den som testas bör behöva fundera på vad som är viktigt i en given situation och skapa egna modelleringar av verkligheten. Ett problem bör skapa ett behov för teorin som kan appliceras och därav förhoppningsvis härleda för en djupare förståelse till varför teorin fungerar. När man fått fundera på hur man \textit{kan} gå till väga för att lösa problem innan man får underlaget så binder man en starkare koppling till materialet och bör därför komma ihåg det bättre\todo{källa}.
}

\textcolor{WildStrawberry}{
    % HEJ AXEL, när du ska göra citationstecken så använd ''hej'' istället för "hej" =)))
}