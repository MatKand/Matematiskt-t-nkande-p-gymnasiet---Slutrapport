%\textcolor{Mahogany}{Vi definierar ett problem som en uppgift som man på förhand inte vet hur man ska lösa. }

% Vad är inte problemlösning
\textcolor{WildStrawberry}{
    Komplexiteten med att definiera vad ett problem är ligger i att det existerar olika syften med vad ett problem vill få ut från den som testas. Vissa problem vill att du hittar x - kan du hitta x? Andra problem kanske inte har ett exakt svar - tillvägagångssättet när man försöker lösa problemet är det som är utvecklande. Det vi vill ta ställning för är att en ''lös ut x''-uppgift som är dekorerad i en \textit{saga} skapar inte ett intressant problem och kunde lika väl varit den vanliga ''vad är x''-övningen. 
}

% Vad är problemlösning
\textcolor{WildStrawberry}{
    Det finns en mängd olika infallsvinklar man kan ta för att definiera \textit{ett problem}. Den som testas bör behöva fundera på vad som är viktigt i en given situation och skapa egen modellering av verkligheten. Ett problem bör skapa ett behov för teorin som kan appliceras och därav förhoppningsvis härleda för en djupare förståelse till varför teorin fungerar. När man fått fundera på hur man \textit{kan} gå till väga för att lösa problem innan man får underlaget så binder man en starkare koppling till materialet och bör därför komma ihåg det bättre\todo{källa}. Men vi vill definiera att \textit{ett problem} är en form av uppgift där man får arbeta med en situation eller uppställning som man inte kan svaret till på förhand.
}


    \textcolor{lila}{Denna definition gör det på sätt och vis mycket svårt att skapa ett problem, eftersom den innebär att en uppgift som är ett problem för en person, kan vara en ren standarduppgift för någon annan. Definitionen innebär också att problemlösning är ett mycket brett område. Nedan presenteras några viktiga delar som, tillsammans eller var och en för sig, kan lyftas fram i ett bra problem.}
    
    \textcolor{lila}{Till att börja med kräver problemlösning ett \textsl{undersökande arbetssätt}. det handlar om att analysera problemet och bryta ner det i mindre delar, och därefter hantera varje del var för sig. Ofta måste man prova sig fram med olika lösningsmetoder innan man hittar en som fungerar.}
        
    \textcolor{lila}{En vanlig form av problemlösning är genom att använda \textsl{öppna problem}. Det innebär ett  problem till vilket det finns flera olika möjliga lösningsgångar för att hitta ett svar, och detta svar behöver inte heller vara unikt utan kan variera beroende på vilka antaganden som gjorts.}

    \textcolor{lila}{En annan viktig del är att kunna översätta ett problem i \textsl{matematiska modeller}. Detta är en nyttig förmåga att besitta i många olika sammanhang, även i arbetslivet~\cite{TheElephant}. Det är också minst lika viktigt att kunna granska en modell med kritisk blick, och fundera på i vilka sammanhang den gäller och när den leder till orimliga resultat.}