%\textcolor{Mahogany}{Vi definierar ett problem som en uppgift som man på förhand inte vet hur man ska lösa. }

% Vad är inte problemlösning
\textcolor{green}{Det finns flera olika definitioner om vad ett matematiskt problem är \cite{olikaDefinitioner}. I en studie från Umeå universitet försökte lektorerna E. Bergqvist och T. Bergqvist klargöra hur begreppet problem var tänkt att tolkas enligt skolverkets läroplan och jämföra det med hur lärare faktiskt tolkade begreppet. I studien framkom det att skolverket i sin läroplan från år 2000 inte uttryckligen definierade vad ett problem var och att lärarnas uppfattningar om vad ett problem var varierade stort \cite{problemVarierandeDef}. Denna bakgrund belyser tydligt att det inte råder konsensus kring någon specifik definition om vad ett matematiskt problem är.} 

\textcolor{green}{Idag har Skolverket lagt till en definition på sin webbplats och eftersom denna rapport handlar om matematiska problem för gymnasiet anser vi det lämpligt att i första hand titta på vad skolverket säger i frågan. De skriver följande~\cite{ProblemDef}:}

\begin{displayquote}
\textcolor{green}{Ett problem är en uppgift som inte är av standardkaraktär och kan lösas på rutin. Det innebär att varje frågeställning där det inte på förhand för eleven finns en känd lösningsmetod kan ses som ett problem.}
\end{displayquote}

\noindent\textcolor{green}{Denna definition belyser svårigheten med att kalla något för ett problem, nämligen att den uppgift som är ett problem för en person, kan vara en enkel standarduppgift för någon annan.} 

\textcolor{green}{I detta arbete har vi valt att använda en något mer sammanfattad version av samma definition som Skolverket använder, nämligen att:}

\begin{displayquote}
\textcolor{green}{Ett problem är en uppgift som man inte på förhand vet hur man ska lösa.}
\end{displayquote}

\noindent\textcolor{green}{Notera att detta betyder att en uppgift kan vara ett problem för någon, medan den för någon annan, med andra förkunskaper, är ett standardproblem. Notera också att detta innebär att vi i vissa sammanhang kommer kalla de uppgifter vi gör för just uppgifter, även om målet är att de ska vara problem för gymnasieeleverna.}
\textcolor{green}{Problemlösning i sin tur beskrivs enligt Skolverket som förmågan att kunna lösa ett problem. Vidare beskrivs problemlösningsförmåga och problemlösning:}

% vår definition av problemlösning:
% "En uppgift som man inte på förhand vet hur man ska lösa. Notera att dettabetyder att en uppgift kan vara ett problem för någon, medan den för någonannan, med andra förkunskaper, är ett standardproblem. Notera också att dettainnebär att vi i vissa sammanhang kommer kalla de uppgifter vi gör för justuppgifter, även om målet är att de ska vara problem för gymnasieeleverna."

%\noindent \textcolor{lila}{
 %   Denna definition gör det på sätt och vis mycket svårt att skapa ett problem, eftersom den innebär att en uppgift som är ett problem för en person, kan vara en ren standarduppgift för någon annan. Definitionen innebär också att problemlösning är ett mycket brett område. Nedan presenteras några viktiga delar som, tillsammans eller var och en för sig, kan lyftas fram i ett bra problem.
%}

\begin{displayquote}
\textcolor{green}{Problemlösningsförmåga innebär att kunna analysera och tolka problem vilket inkluderar ett medvetet användande av problemlösningsstrategier som att till exempel förenkla problemet, införa lämpliga beteckningar, ändra förutsättningarna. Att lösa problemet innebär att genomföra ett resonemang där grunderna för resultatets giltighet blir tydligt och resultatet korrekt.}
\end{displayquote}

%\textcolor{WildStrawberry}{
%    Komplexiteten med att definiera vad ett problem är ligger i att det existerar olika syften med vad ett problem vill få ut från den som testas. Vissa problem vill att du hittar x - kan du hitta x? Andra problem kanske inte har ett exakt svar - tillvägagångssättet när man försöker lösa problemet är det som är utvecklande. Det vi vill ta ställning för är att en ''lös ut x''-uppgift som är dekorerad i en \textit{saga} skapar inte ett intressant problem och kunde lika väl varit den vanliga ''vad är x''-övningen. 
%}

% Vad är problemlösning
%\textcolor{WildStrawberry}{
%    Det finns en mängd olika infallsvinklar man kan ta för att definiera \textit{ett problem}. Den som testas bör behöva fundera på vad som är viktigt i en given situation och skapa egen modellering av verkligheten. Ett problem bör skapa ett behov för teorin som kan appliceras och därav förhoppningsvis härleda för en djupare förståelse till varför teorin fungerar. När man fått fundera på hur man \textit{kan} gå till väga för att lösa problem innan man får underlaget så binder man en starkare koppling till materialet och bör därför komma ihåg det bättre\todo{källa}. Men vi vill definiera att \textit{ett problem} är en form av uppgift där man får arbeta med en situation eller uppställning som man inte kan svaret till på förhand.
%}

\textcolor{lila}{
    Till att börja med kräver problemlösning ett \textsl{undersökande arbetssätt}. Det handlar om att analysera problemet och bryta ner det i mindre delar, och därefter hantera varje del var för sig. Ofta måste man prova sig fram med olika lösningsmetoder innan man hittar en som fungerar.
}
        
\textcolor{lila}{
    En vanlig form av problemlösning är genom att använda \textsl{öppna problem}. Det innebär ett  problem till vilket det finns flera olika möjliga lösningsgångar för att hitta ett svar, och detta svar behöver inte heller vara unikt utan kan variera beroende på vilka antaganden som gjorts.
}

\textcolor{lila}{
    En annan viktig del är att kunna översätta ett problem i \textsl{matematiska modeller}. Detta är en nyttig förmåga att besitta i många olika sammanhang, även i arbetslivet~\cite{TheElephant}. Det är också minst lika viktigt att kunna granska en modell med kritisk blick, och fundera på i vilka sammanhang den gäller och när den leder till orimliga resultat.
}
\textcolor{Mahogany}{Inte minst så är det ett kunskapskrav för gymnasiematematik att kunna tillämpa matematiska modeller, något som Skolverket beskriver som en \textsl{modelleringsförmåga} \cite{ProblemDef} (källa till kursplanerna). Denna förmåga beskrivs som följande:}

\begin{displayquote}
    \textcolor{Mahogany}{
        \textsl{Modelleringsförmåga innebär att kunna formulera en matematisk beskrivning – modell – utifrån en realistisk situation.}
    }
\end{displayquote}
\noindent\textcolor{Mahogany}{Vidare beskrivs denna modelleringsförmåga som förmågan att kunna koppla resultatet till en verklig situation och bedöma om det är rimligt, för att sedan eventuellt utvärdera modellen på nytt.}

% ----------------
% Från skolverket:
% ----------------
% Det handlar om att själv utforma en koppling i form av en modell snarare än att använda färdigformulerade modeller. När modellen är färdig innebär modelleringsförmåga att kunna använda modellens egenskaper för att till exempel lösa ett matematiskt problem eller en standarduppgift. Modelleringsförmågan innebär också att kunna tolka resultatets relation till den verklighetssituation man hade från början. Det innebär även att kunna utvärdera modellens egenskaper och begränsningar i förhållande till den verkliga situationen.