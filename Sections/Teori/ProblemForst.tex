\textcolor{green}{Den undervisningsmetod som används i samband med framtagandet av problemen kallas för \textit{problembaserat lärande}, förkortat \textit{PBL}~\cite{PBLdefinition}. Metoden går ut på att lära sig matematik genom att lösa verklighetsanknutna och ostrukturerade problem. Detta är synonymt med öppna problem, som definierades i avsnitt~\ref{sec:problemdef}. Problemen kan alltså lösas på olika sätt och lösningen är tänkt att vara oförutsägbar. Tanken är att arbetet med realistiska problem ska ge möjlighet att tillämpa de matematiska teorier som är relevanta för just dessa problem. PBL nyttjar människans medfödda förmåga och nyfikenhet att lära sig, vilket ger en djupare förståelse för den kunskap som lärs in~\cite{djupareKunskapPBL}.}

\textcolor{Mahogany}{Alla gynnas dock inte av denna undervisningsmetod, visar bland annat en metastudie om styrkor och svagheter med PBL~\cite{Metastudie}. Den visar att elever både i grundskolan och gymnasieskolan med större pedagogiska behov kan ha svårt att arbeta under denna undervisningsmetod. Framför allt så handlar det om både att söka egen information och vara aktiv i grupparbetet. Därtill visar studien att PBL är mer resurskrävnade arbetsmetod för lärare jämfört med än traditionell undervisning.
}



%Om ett par elever lyckas komma på en lösning, eller del av en lösning, till ett problem så kan det bidra till att göra matematiken mindre främmande. Om en viss del av den teori som lärs ut i matematiken blir i form av egen utforskning, och att man lyckas komma en bit eller hela vägen, så kan denna belöning agera mycket motiverande~\cite{TheElephant}. Dock inte uteslutande då varje individ inte har samma målsättningar samt motivation.} \textcolor{lila}{Men genom att utifrån en fråga upptäcka ett matematiskt begrepp ges detta automatiskt ett sammanhang och eleverna inser att det är användbart. På samma sätt är det motiverande för elever att känna att de äger problemet, det vill säga att de själva får forma problemet i form av frågeställning och bakgrundsinformation.} %Mycket av den teori som elever idag får lära sig under sin skolgång har kommit till under processen att lösa verkliga problem. Det är alltså verkliga problem som har lett till dessa teorier, som sedan har blivit formler. Att komma till denna insikten kan visa sig vara ett kraftfullt verktyg för studentens delaktighet i sin utbildning.
%
%
%
%Problembaserat lärande är den pedagogik som de flesta av våra uppgifter använder sig av.
%INQUIRY BASED LEARNING...
%Det är den karaktär våra problem har
%Vad är fördelen med det?
%Vi tror att eleverna får djupare förståelse för teori, om de först får se verkliga scenarion som den kan appliceras på
%matten känns inte irrelevant, utan kan lösa riktiga problem
%matematiken får även fler dimensioner än att vara ute efter ett enda rätt svar Detta eftersom uppgifterna är gjorda på ett sätt där de på precis samma sätt som många problem i verkliga inte har "ett rätt svar" utan snarare ska estimeras och optimeras. 
%Dag Wedellin använder i sin kurs x denna metod och har fått denna respons - detta tycker vi talar för att denna metod är bättre än den traditionella
%
%
\textcolor{turkos}{Daniel Willingham skriver i sin bok \textsl{Why Students Don't Like School?} om vikten av repetition för inlärning \cite{WhyDontStudents}. Enligt Willingham så är faktabaserad kunskap starkt sammankopplat till kritiskt tänkande så som problemlösning, och för att uppnå den kunskapen krävs repetition. Han skriver att övning hjälper människor att överföra information till nya situationer vilket är en mycket central del av problemlösning. Willingham beskriver även skillnaden mellan novisers och experters (till exempel matematiker) och deras olika arbetssätt. Det går inte att förvänta sig att novisen ska lösa samma problem som experten tar sig an då de har helt olika kunskapsbas. Istället anser han att man kan låta elever inspireras av experternas arbete utan att faktiskt låta dem göra samma sak. } 

\textcolor{green}{Det som kan ses som motsatsen till PBL är \textit{deduktiv inlärning}. Deduktiv inlärning är den undervisningsmetod som den traditionella matematikundervisningen, som definieras i avsnitt~\ref{sec:MatteForr}, använder sig av. Det innebär alltså att i första hand lära ut teorier för att sedan tillämpa dessa på lämpliga uppgifter och problem ~\cite{deduktivInlärning}.}
%\textcolor{turkos}{Daniel Willingham beskriver i sin bok \textsl{Why Students Don't Like School?} att för att kunna tänka som en expert så behöver man grundläggande kunskap, och det enda sättet att få den grundkunskapen är genom repetition \cite{WhyDontStudents}. }

% Practice helps transfer information to new situations

