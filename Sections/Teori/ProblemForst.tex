\textcolor{green}{Problembaserat lärande är en pedagogik som går ut på att lära sig genom att lösa problem ell


I uppgifterna vi har tagit fram presenteras ingen teori som hör till uppgifterna, utan teorin blir någon som kommer fram under problemlösandets gång [precis som det står på x.x]. matematiska problem vi har gjort så presenteras ingen teori till uppgiften som det står i x.x. Utan eleverna möts av ett problem och får under processens gång finna}

\textcolor{green}{elever ska lösa de problem vi har tagit fram så presenteras}

\textcolor{green}{Den lärande processen som eleven går genom när hen löser problemen som vi har tagit fram kallas "inquiry-based learning". Det är en lärande metod som går ut på [källa].}

%
%Problembaserat lärande är den pedagogik som de flesta av våra uppgifter använder sig av.
%INQUIRY BASED LEARNING...
%Det är den karaktär våra problem har
%Vad är fördelen med det?
%Vi tror att eleverna får djupare förståelse för teori, om de först får se verkliga scenarion som den kan appliceras på
%matten känns inte irrelevant, utan kan lösa riktiga problem
%matematiken får även fler dimensioner än att vara ute efter ett enda rätt svar Detta eftersom uppgifterna är gjorda på ett sätt där de på precis samma sätt som många problem i verkliga inte har "ett rätt svar" utan snarare ska estimeras och optimeras. 
%Dag Wedellin använder i sin kurs x denna metod och har fått denna respons - detta tycker vi talar för att denna metod är bättre än den traditionella
%
%
