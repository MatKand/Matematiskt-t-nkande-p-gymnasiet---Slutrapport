\textcolor{green}{Den undervisningsmetod som används i samband med de flesta av de framtagna problemen kallas för \textit{problembaserat lärande}, förkortat \textit{PBL}. Metoden går ut på att lära sig problemlösning genom att lösa verklighetsanknutna och ostrukturerade problem. Med ostrukturerade problem menas att uppgiften inte har och inte indikerar några bestämda sekvensiella steg som den ska lösas i. Problemen kan lösas på olika sätt och lösningen är tänkt att vara oförutsägbar. Tanken är att arbetet med verkliga problem ska ge möjlighet att tillämpa de matematiska teorier som är relevanta för just dessa problem. PBL nyttjar människans medfödda förmåga och nyfikenhet att lära sig, vilket ger en djupare förståelse för den kunskap som lärs in \cite{PBLdefinition} \cite{djupareKunskapPBL}. Det som kan ses som motsatsen till PBL är \textit{deduktiv inlärning}. Deduktiv inlärning är den undervisningsmetod som den traditionella matematikundervisningen använder sig av, att i första hand lära ut teorier för att sedan presentera lämpliga uppgifter där dessa kan tillämpas~\cite{deduktivInlärning}.}

%Om ett par elever lyckas komma på en lösning, eller del av en lösning, till ett problem så kan det bidra till att göra matematiken mindre främmande. Om en viss del av den teori som lärs ut i matematiken blir i form av egen utforskning, och att man lyckas komma en bit eller hela vägen, så kan denna belöning agera mycket motiverande~\cite{TheElephant}. Dock inte uteslutande då varje individ inte har samma målsättningar samt motivation.} \textcolor{lila}{Men genom att utifrån en fråga upptäcka ett matematiskt begrepp ges detta automatiskt ett sammanhang och eleverna inser att det är användbart. På samma sätt är det motiverande för elever att känna att de äger problemet, det vill säga att de själva får forma problemet i form av frågeställning och bakgrundsinformation.} %Mycket av den teori som elever idag får lära sig under sin skolgång har kommit till under processen att lösa verkliga problem. Det är alltså verkliga problem som har lett till dessa teorier, som sedan har blivit formler. Att komma till denna insikten kan visa sig vara ett kraftfullt verktyg för studentens delaktighet i sin utbildning.
%
%
%
%Problembaserat lärande är den pedagogik som de flesta av våra uppgifter använder sig av.
%INQUIRY BASED LEARNING...
%Det är den karaktär våra problem har
%Vad är fördelen med det?
%Vi tror att eleverna får djupare förståelse för teori, om de först får se verkliga scenarion som den kan appliceras på
%matten känns inte irrelevant, utan kan lösa riktiga problem
%matematiken får även fler dimensioner än att vara ute efter ett enda rätt svar Detta eftersom uppgifterna är gjorda på ett sätt där de på precis samma sätt som många problem i verkliga inte har "ett rätt svar" utan snarare ska estimeras och optimeras. 
%Dag Wedellin använder i sin kurs x denna metod och har fått denna respons - detta tycker vi talar för att denna metod är bättre än den traditionella
%
%
