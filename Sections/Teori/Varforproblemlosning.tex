\textcolor{Mahogany}{\cite{problemlosningigrupp}}

% -------
% OBS - Nedanstående punkter bör skrivas om då mycket enbart är kopierat. Försöker skapa en grund vad vi kan ha med som motiverar problemlösning men även tar upp att det kan krävas modifikation för de med inlärningssvårigheter.
% -------

% Risken att välja problem av fel svårighetsgrad minskar vid val av rika problem, eftersom de kan lösas på olika abstraktionsnivåer. Detta tillåter att elever med fallenhet för matematik kan utmanas och stimuleras utan att separeras från den ordinarie klassundervisningen.

% Eftersom elever lätt identifierar nyckelord för att extrahera och använda för att lösa problem, eller uppgift, blir det därmed svårare att utveckla en studieteknik. Därutöver så framgår det att generella inlärningssvårigheter (learning disabilities/LD) kan orsaka problem med uppmärksamhet, minne, förkunskaper, ordförråd, språkprocesser, kunskap om och användande av strategier, visou-spatiala processer och självreglering. Dessa brister påverkar de drabbade eleverna inom många skolämnen, varav matematik är ett. Problemlösning kan därmed vara svårt för elever med inlärningssvårigheter, men det går att anpassas. Polya förespråkar en fyrstegsmodell som lärare i stor utsträckning använder inom den traditionella problemlösningsundervisningen, och är alltför ytlig för elever med drag av LD. Det finns då en alternativ fyrstegsmodell. Med det sagt så är problemlösning inte nödvändigtvis något för alla per automatik, utan kräver att problemen är dynamiska, och kan anpassas både till personer med LD men även elever som anses vara duktigare och behöver mer stimulation (i vårt fall fördjupningsfrågor).