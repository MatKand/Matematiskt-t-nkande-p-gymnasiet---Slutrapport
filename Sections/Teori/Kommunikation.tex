\textcolor{turkos} {
Givande diskussioner kan få elever att inse att det är tillåtet att ha egna idéer och tankar kring matematik, ett ämne som annars ofta upplevs handla om att följa regler. När elever diskuterar problem med varandra så lär de av varandra, och kan ofta uttrycka sig på ett sätt som kan göra det lättare för dem att förstå än vad ofta läraren kan.  \cite{TheElephant}
}

\textcolor{turkos} {
Faktum är att diskussion är så viktigt för bra problemlösning att Hagland, Hedrén och Taflin skriver, i sin bok Rika matematiska problem, att den viktigaste skillnaden mellan ett vanligt problem och vad de kallar ett rikt problem är att det rika problemet leder till diskussion. \cite{RikaProblem}
}

\textcolor{turkos}{
I Rika matematiska problem beskrivs även vikten av ha en avslutande klassdiskussion. Anledningen till att man lägger vikt vid diskussion just på slutet är att alla elever tid att jobba med problemet, även om alla kanske inte har löst det, och kan därmed bidra till diskussionen. Även läraren har haft möjlighet att bilda sig en uppfattning om vilka metoder eleverna har använt för att lösa problemen och kan därmed leda diskussionen och ta upp intressanta idéer som uppstått i klassen och lyfta dem till resterande elever. \cite{RikaProblem}
}

\textcolor{turkos} {
Även Skolverket lyfter att matematikundervisning där elevers egna lösningstratergier diskuteras leder till mycket positiva resultat och ökar elevers lust att lära. \cite{Skolverket03}
}

\textcolor{turkos} {
Ett land som tas upp som ett exempel på god matematikundervisning av både Skolverket och Boaler är Japan \cite{TheElephant}. I Japan läggs stor vikt vid att efter eleverna löst ett problem så ska de dela med sig utav sina lösningar och diskutera dem med varandra. Diskussionen används som en utgångspunkt för att läraren ska kunna lyfta viktiga aspekter ur deras lösningar och tillvägagångsätt. \cite{Skolverket03}
}
%Rika problem är problem som leder till diskussion

%Ger chans att lyfta sina egna idéer och utveckla dem. 

%Diskussion leder till öka lust att lära. 

%Japanska skolan lägger fokus på diskussion och lyfts som ett föredöme av både skolverket och Jo Boaler 

% Tar upp strukturen med diskussion före och efter elever får lösa problemet. 


% Skriv något om att det är bra att reflektera efter man har gjort en uppgift. 