\textcolor{cyan} {
Givande diskussioner kan få elever att inse att det är tillåtet att ha egna idéer och tankar kring matematik, ett ämne som annars ofta upplevs handla om att följa regler. När elever diskuterar problem med varandra så lär de av varandra, och kan ofta uttrycka sig på ett sätt som kan göra det lättare för dem att förstå än vad ofta läraren kan.  \cite{TheElephant}
}

\textcolor{cyan} {
Faktum är att diskussion är så viktigt för bra problemlösning att Hagland, Hedrén och Taflin skriver att den viktigaste skillnaden mellan ett vanligt problem och vad de kallar ett rikt problem är att det rika problemet leder till diskussion. \cite{RikaProblem}
}

\textcolor{cyan} {
Även Skolverket lyfter att matematikundervisning där elevers egna lösningstratergier diskuteras leder till mycket positiva resultat och ökar elevers lust att lära. \cite{Skolverket03}
}

\textcolor{cyan} {
Ett land som tas upp som ett exempel på god matematikundervisning av både Skolverket och Boaler är Japan \cite{TheElephant}. I Japan läggs stor vikt vid att efter eleverna löst ett problem så ska de dela med sig utav sina lösningar och diskutera dem med varandra. Diskussionen används som en utgångspunkt för att läraren ska kunna lyfta viktiga aspekter ur deras lösningar och tillvägagångsätt. \cite{Skolverket03}
}
%Rika problem är problem som leder till diskussion  

%Ger chans att lyfta sina egna idéer och utveckla dem. 

%Diskussion leder till öka lust att lära. 

%Japanska skolan lägger fokus på diskussion och lyfts som ett föredöme av både skolverket och Jo Boaler 

% Tar upp strukturen med diskussion före och efter elever får lösa problemet. 