
\label{sec:delaktighet}
\textcolor{WildStrawberry}{
    Om ett par elever lyckas komma på en lösning, eller del av en lösning, till ett problem så kan det bidra till att göra matematiken mindre främmande. Om en viss del av den teori som lärs ut i matematiken blir i form av egen utforskning, och att man lyckas komma en bit eller hela vägen, så kan denna belöning agera mycket motiverande~\cite{TheElephant}.} \textcolor{lila}{Men genom att utifrån en fråga upptäcka ett matematiskt begrepp ges detta automatiskt ett sammanhang och eleverna inser att det är användbart. På samma sätt är det motiverande för elever att känna att de äger problemet~\cite{TheElephant}, det vill säga att de själva får forma problemet i form av frågeställning och bakgrundsinformation. }
    
\textcolor{WildStrawberry}{
    All teori som en elev utsätts för under sin skolgång har varit ''verkliga'' problem som en gång inte varit färdiga formler eller modeller, som nu går att utnyttja. Vi anser att det kan vara ett kraftfullt verktyg för elevernas delaktighet i sin utbildning. När en individ behärskar förmågan att bemöta situationer objektivt så har hen större kapacitet till att förstå, utforska och utmana existerande begrepp. }
    
    % Att komma till denna insikten kan visa sig vara ett kraftfullt verktyg för studentens delaktighet i sin utbildning. Förmågan att ifrågasätta och kunna bemöta situationer objektivt ger individer en större frihet att förstå, utforska och utmana existerande begrepp.
        
        %Mycket av den teori som elever idag får lära sig under sin skolgång har kommit till under processen att lösa verkliga problem. Det är alltså verkliga problem som har lett till dessa teorier, som sedan har blivit formler. Att komma till denna insikten kan visa sig vara ett kraftfullt verktyg för studentens delaktighet i sin utbildning.
        
% 