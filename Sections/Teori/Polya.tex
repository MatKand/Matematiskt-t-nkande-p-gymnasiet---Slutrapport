\textcolor{Mahogany}{
    Redan under 1940-talet skrevs litteratur om problemlösning. Polya framhäver i sin bok \textsl{How to solve it} \cite{Polya} behovet av iteration, det vill säga att upprepade gånger titta tillbaka på ett och samma problem men med olika synsätt. Detta är också ett viktigt steg i de \textsl{fyra faser} som han har tagit fram för hur man löser ett problem.
    Det första är att \textsl{förstå} problemet, vad som efterfrågas. Det andra är att kunna koppla olika element i problemet och se hur det okända är länkat till den data man har, för att i tredje steget genomföra problemet. 
    %Det andra steget kan dock oftast i traditionella matematikuppgifter vara väldigt uppenbart och det finns en risk att man redan i detta steget får fram ett svar och går vidare. 
    Fjärde steget handlar om att man tittar tillbaka på vad man har fått fram och diskuterar resultatet.
    %Detta är också det som kommer att läggas betydligt stor vikt på vid utformandet av problemen som utvecklats i detta projekt, nämligen \textsl{diskussion}.
}

\textcolor{Mahogany}{
    Vidare beskriver Polya mer ingående för det fjärde steget att elever tenderar att gå vidare och göra annat efter att ha kommit fram till ett svar, och att de då missar en viktig fas. Genom att utvärdera sina resultat och framför allt \textsl{processen} att komma fram till resultatet, ges det utrymme för personen att stärka sin kunskap och utveckla förmågan att lösa problem. Han menar vidare att det alltid finns sätt att förbättra sin lösning men också sin förståelse för lösningen.
}