\textcolor{Mahogany}{
    I \textsl{How to solve it} \cite{Polya} framhäver Polya behovet av iteration, att man alltid ändrar sitt synsätt, vilket också är ett viktigt steg i de \textsl{fyra faser} som han har tagit fram för hur man löser ett problem.
    Det första är att \textsl{förstå} problemet, vad som efterfrågas. Det andra är att kunna koppla olika element i problemet och se hur det okända är länkat till den data man har, för att i tredje steget genomföra problemet. 
    Det andra steget kan dock oftast i traditionella matematikuppgifter vara väldigt uppenbart och det finns en risk att man redan i detta steget får fram ett svar och går vidare. 
    Fjärde steget handlar nämligen om att man kollar tillbaka på vad man har fått fram och diskuterar resultatet. Detta är också det som kommer att läggas betydligt stor vikt på vid utförandet av problemen som utvecklats i detta projekt, nämligen \textsl{diskussion}.
}

\textcolor{Mahogany}{
    Vidare beskriver Polya mer ingående för det fjärde steget att elever tenderar att gå vidare och göra annat efter att ha kommit fram till ett svar, och att eleverna då missar en viktig fas. Genom att utvärdera sina resultat och framför allt \textsl{processen} att komma fram till resultatet, ges det utrymme för personen att att stärka sin kunskap och utveckla deras förmåga att lösa problem. Han menar vidare att det alltid finns sätt att förbättra sin lösning men också sin förståelse för lösningen.
}