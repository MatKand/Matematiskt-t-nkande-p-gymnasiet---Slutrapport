% Ta ställning till omständigheter
\textcolor{Mahogany}{Något som är viktigt att ta ställning till när man utformar problem för gymnasiematematik är att det måste vara förenligt med övriga undervisningen. Man ska övertyga både lärare och elever att det för det första är relevant, men också att det kommer att vara mer givande än att räkna i boken. Det är också viktigt att ha som en röd tråd vad man egentligen vill att elever ska få ut av ett problem, och försöker förmedla detta på ett naturligt sätt. Är problemlösning bara något som tar tid från räkning i boken eller kan man få elever att förstå att de kan gynnas av denna typ av matematikundervisning? Ett ''bra'' problem bör rimligtvis lyfta fram den nyttan. Framför allt så bör de även vara roliga att lösa.}

% Hur bör det utföras?
\textcolor{Mahogany}{Vi har lagt ner en del tid på att informera lärare hur vi tänker att våra problem ska utföras. Detta är för att vi vill att man ska få ut så mycket som möjligt från problemen framförallt via diskussion och reflektion. Diskussionsdelen är också ett sätt för elever att våga försöka lösa ett problem som till synes kanske skulle vara avskräckande. Oavsett svårighetsgraden på problemet så anser vi att diskussionen är en viktig del i lärandet, då man får möjligheten att uttrycka sin nivå av förstående och ens kunskap testas genom att man behöver förklara det för någon annan. Samtidigt så gynnas de elever som inte kommit lika långt att få höra hur andra tänkt, och därifrån kunna arbeta vidare själva.}

% Rimlig nivå
\textcolor{Mahogany}{En svårighet med att utveckla problem är att sätta en rimlig svårighetsgrad, speciellt med tanke på att vi under utvärderingen fick jobba med framför allt väldigt specifik tidsram. Ett sätt att konfrontera detta är att ha en tydlig grund som problemet bygger på, för att sedan möjliggöra vidareutveckling i största mån. Denna vidareutveckling handlar ofta om fördjupningsfrågor som uppmanar till diskussion, vilket kan underbygga elevernas möjlighet att få en djupare förståelse för ämnet.}