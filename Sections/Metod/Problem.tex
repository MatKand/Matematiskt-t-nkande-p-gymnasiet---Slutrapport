\textcolor{lila}{De problem som har skapats har som tidigare nämnts konstruerats med målet att öka andelen problemlösning i gymnasiet. För att uppnå detta har vi vid skapandet av problemen utgått från ett antal riktlinjer, där en eller flera av dessa belyses i varje problem.}

\textcolor{lila}{Den första, och kanske viktigaste, riktlinjen är att problemen ska leda till ett \textsl{undersökande arbetssätt}, se avsnitt~\ref{sec:problemdef}. I samma avsnitt diskuteras även \textsl{öppna problem}, som också var en bakomliggande tanke för många av problemen. Vi har även skapat problemen med avsikt att träna eleverna på \textsl{modellering} och \textsl{programmering}, vilka tas upp i avsnitt \ref{sec:problemdef} respektive \ref{sec:ProgrammeringOchMatematik}. \textsl{Verklighetsanknytning} ska också enligt avsnitt~\ref{sec:Verklighetsanknytning} finnas med i flera av problemen.}

\textcolor{lila}{Med detta som underlag har vi skapat olika problem. Vi har tagit inspiration från våra egna erfarenheter från hela vår skolgång, inklusive universitetet, samt från en mängd olika böcker, som nämns i samband med respektive problem. Utifrån vår grundidé har vi därefter arbetat oss vidare för att skapa ett problem som passar gymnasieelever. På så sätt har vi tagit fram problem som alla har tre gemensamma delar: \textsl{Introduktion}, \textsl{Genomförande} och \textsl{Diskussion}.}

\textcolor{lila}{I introduktionen presenteras problemet, och i vissa fall ingår där ett antal frågor att diskutera innan man börjar räkna på problemet. Själva problemet genomförs i grupper om två, och därefter kommer diskussionen. Denna börjar i vissa fall med att man går ihop i lite större grupper och presenterar och diskuterar sina lösningar av problemet, förklara varför man gjort som man gjort och jämföra eventuella olika lösningsgångar. Avslutningsvis diskuterar klassen problemet tillsammans, eventuellt med hjälp av våra föreslagna diskussionsfrågor.} 

\textcolor{lila}{Till varje problem följer också \textsl{Information till läraren}, med information om problemets mål samt eventuella förkunskaper och material som behövs. Det finns också \textsl{Ytterligare information}. Dit kan höra förslag på hur problemet kan introduceras, bakgrundsinformation som kan vara bra för läraren att repetera innan problemet genomförs med klassen samt tankar om vilka frågor som skulle kunna dyka upp vid diskussionerna och hur dessa kan hanteras. Till alla problem finns också en presentation, som läraren kan välja att använda. På så sätt presenteras problemen tillsammans med en komplett lektionsplanering.}

\textcolor{lila}{Vi har även skrivit en text som förklarar tanken med våra problem. Där presenteras upplägget samt några idéer på hur man ska agera som lärare, bland annat genom att undvika att framställa ett förslag som felaktigt.}

\textcolor{lila}{De problem som ska testas har planerats för att ta ungefär 50 minuter att genomföra. Detta för att de ska kunna genomföras under en lektion på en timme, och att det ska finnas tid över att svara på utvärderingsenkäten, se kapitel \ref{sec:testavproblemen}./todo{!}}

\textcolor{Mahogany}{Metoden kring hur vi utformade problemen och vilken nivå de skulle ligga på var att vi med relativt stor frihet tog fram problem som vi ansåg låg på gymnasienivå.
Den mer specifika nivån var menad att vara anpassningsbar, på det sättet att man med hjälp av fördjupningsfrågor kunde fördjupa sig inom ämnet och därmed kunna försvåra problemet om så önskas.
I det senare stadiet av projektet blev det viktigare att bestämma en nivå då lärare som skulle testa problemen ofta enbart undervisade på en specifik nivå. Behovet var därmed större att bestämma en nivå för problemen, alternativt anpassa problem utefter nivån.}