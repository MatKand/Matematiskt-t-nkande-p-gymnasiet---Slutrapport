\textcolor{lila}{Av de 58 lärare som svarade på utvärderingsenkätetn som presenterades i avsnitt~\ref{sec:Bakgrundsenkat} var det 20 stycken som fyllde i ett intresse för att testa ett av våra problem.} 

\textcolor{lila}{De sex problem som presenterades i avsnitt \ref{sec:Fermi} till och med \ref{sec:Sortera} fördelades mellan dessa, där hänsyn även togs till vilka kurser lärarna hade angivit att de för tillfället undervisade i. Problemen skickades ut fredagen den 31:e mars, tillsammans med dokumentet ''Allmän information till lärare'' samt tillhörande presentation. Därefter fick lärarna 5 veckor på sig att genomföra problemen, det vill säga till den 5:e april.} 

\textcolor{lila}{Lärarna hade som tidigare nämnt möjlighet att följa det givna upplägget till den grad de själva ville, och därefter fick de en utvärderingsenkät. Den innehöll frågor om hur de tyckte att problemet presenterats från vårt håll, om de valt att ändra något eller om det var något annat de saknade samt om hur de tyckte att problemet togs emot av eleverna. Eleverna fick en liknande utvärderingsenkät där de fick skriva sin definition av problemlösning, vad de tyckte om problemet, vad de tyckte om nivån på problemet samt om de upplevde att de lärt sig något.}