\textcolor{green}{Hemsidan utvecklades i programmeringsspråket Javascript. Fördelarna med att använda sig av ett programmeringsspråk som Javascript istället för att exempelvis bara använda HTML är att man kan designa sidan på många fler sätt och får total kontroll över sidans innehåll. Med total kontroll menas det att man skulle kunna ha annan material på webbplatsen än bara text- och bildmaterial. Det skulle kunna vara någon form av digitalt verktyg, t ex ett matmatiskt problem som använder sig av en simulator. Detta kan göras med hjälp av Javascripts många olika bibliotek.}
\\ \\
\textcolor{green}{Biblioteket som användes för att utveckla webbplatsen var React JS. React JS är ett av de mest populära biblioteken att bygga webbplatser med i Javascript [källa]. I Reacts bibliotek finns främst verktyg för att jobba med vy-delen av en hemsida. Vill man integrera tyngre applikationer i webbsidan så finns möjligheter för det. Men om vyn är det största fokuset så är React ett mycket lämpligt val då det är väldokumenterat och gör det enkelt att bygga vidare på hemsidans design.}
\\

% Ska förmodligen ligga i resultat eller något, men det får ligga kvar här nu
\textcolor{Mahogany}{Hemsidans design har gjorts enkel där känslan ska vara att man hela tiden är på ''en'' sida. Den består av olika problem som listas i ett rutnät på ''Problem''-sidan, innehållandes titel och bild. Klickar man på ett av problemen så kommer man vidare till en mer detaljerad vy över problemet med en kort beskrivande text följt av en länk till det fullständiga problemet samt i många fall en tillhörande PowerPoint, som är tänkt att underlätta utförandet. För programmeringsuppgifterna så bifogas även ett lösningsförslag skrivet i Java.}

\textcolor{Mahogany}{Hemsidan har även en startsida som ska kännas inbjudande och engagerande, där besökaren ges en introducerande text om vad vi arbetar med samt en knapp som tar personen direkt till problemen. Dessa kommer man givetvis också åt via menyn, skillnaden är att vi vill fånga uppmärksamheten på ett mer välkomnande sätt för den som besöker hemsidan för första gången genom att ha en knapp över en bild som illustrerar några av våra problem.}

\textcolor{Mahogany}{Till sist så inkluderas sidan ''Om oss'', som beskriver projektets arbete, samt en sida ''Information till läraren'', vars innehåll beskrivs närmare i \ref{sec:Skapandetavproblem}.}