\noindent \textcolor{green}{Matematik är ett kärnämne i skolan. Samtidigt är en vanlig upplevelse för många elever att matematik är svårt, tråkigt och oanvändbart. Med detta projekt ville vi ändra på detta genom att införa nya typer av matematiska problem för gymnasieskolan. Dessa problem låter eleverna arbeta mer med problemlösning. Problemen har en realistisk verklighetsanknytning, ger underlag för diskussion, går att lösa på olika sätt, och ska framför allt ge känslan av att matematik är användbart. Eftersom programmering snart ska införas i skolmatematiken är även en del av problemen konstruerade för att förbereda inför det.}

\textcolor{green}{Som en del av vårt resultat gjorde vi bland annat en undersökning som 58 lärare deltog i. 91\% av lärarna angav att de arbetade för att inkluderade mer problemlösning samtidigt som 42\% tyckte att det var svårt att hitta bra problem. Efter att problemen testades på gymnasieskolor svarade en del av eleverna på en enkät om sin upplevelse av det problem de arbetade med. Av de som svarade angav 80\% att de lärde sig något nytt och 75\% uttryckte på något sätt att problemet var roligt eller meningsfullt.}

\textcolor{green}{Vår slutsats blev att lärare idag vill inkludera mer problemlösning i undervisningen, men att det inte är lätt att hitta bra problem, samtidigt som en del har tidsbrist. Trots att vi inte lyckades testa problemen i så många klasser som vi hade önskat, så var både vi och lärarna som testade dem nöjda med problemen.}