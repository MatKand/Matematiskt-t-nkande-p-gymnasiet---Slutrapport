\noindent \textcolor{Mahogany}{Matematik är ett kärnämne i skolan idag, och anses av de flesta mycket viktigt. Samtidigt är det en vanlig uppfattning att det är svårt, tråkigt och oanvändbart. Med detta projekt vill vi underlätta för lärare att arbeta med problemlösning genom att införa nya typer av matematiska problem för gymnasieskolan,} 
\textcolor{green}{då vår undersökning visade att $91\%$ av gymnasielärare angav att de arbetade med problemlösning, men $42\%$ angav att de hade svårt att hitta bra problem.}
\textcolor{Mahogany}{Huvudsyftet med detta projekt har därför varit att utforma problem för att hjälpa dessa lärare. Problemen är öppna, verklighetsanknutna, ger underlag för diskussion, och ska framför allt ge känslan av att matematik är användbart. Eftersom programmering snart ska införas i skolmatematiken är även en del av problemen konstruerade för att förbereda inför det.}
\textcolor{lila}{Till problemen följer även ett förslag på en komplett lektionsplanering och allt presenteras på en användarvänlig hemsida som ger enkel tillgång för alla lärare som är intresserade.}
\textcolor{green}{De skapade problemen testades även på gymnasieskolor vilket visade att av de 26 svarande ansåg $80\%$ att de lärde sig något och $75\%$ uttryckte på något sätt att problemet var roligt eller meningsfullt. Alla lärare var också nöjda med det problem de testat och ansåg att den information som följde med dem gjorde dem enkla att sätta sig in i.}

% \textcolor{green}{Vår slutsats blev att lärare idag vill inkludera mer problemlösning i undervisningen, men att det inte är lätt att hitta bra problem, samtidigt som en del har tidsbrist. Trots att vi inte lyckades testa problemen i så många klasser som vi hade önskat, så var lärarna och de flesta av eleverna som testade problemen nöjda.}