\textcolor{Mahogany}{Hemsidans design har gjorts enkel där känslan ska vara att man hela tiden är på ''en'' sida. Den består av olika problem som listas i ett rutnät på ''Problem''-sidan, innehållandes titel och bild. Klickar man på ett av problemen i rutnätet så kommer man vidare till en mer detaljerad vy över problemet med en kort beskrivande text följt av en länk till det fullständiga problemet. I många fall bifogas även en tillhörande PowerPoint, som är tänkt att underlätta utförandet. För programmeringsuppgifterna så bifogas även ett lösningsförslag skrivet i Java.}

\textcolor{Mahogany}{Hemsidan har även en startsida som designats för att kännas inbjudande och engagerande, där besökaren ges en introducerande text om vad vi arbetar med samt en knapp som tar personen direkt till problemen. Dessa kommer man givetvis också åt via menyn, skillnaden är att vi vill fånga uppmärksamheten på ett mer välkomnande sätt för den som besöker hemsidan för första gången genom att ha en knapp över en bild som illustrerar några av våra problem.}

\textcolor{Mahogany}{Till sist så inkluderas sidan ''Om oss'', som beskriver projektets arbete, samt en sida med ''Information till läraren'', vars innehåll beskrivs närmare i \ref{sec:Skapandetavproblem}.}