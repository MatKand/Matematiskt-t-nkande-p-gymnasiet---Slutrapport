%Förändringar i matematikundervisningen

\textcolor{lila}{Den traditionella undervisningen var länge den som mer eller mindre uteslutande användes i Sverige, särskilt i högre åldrar \cite{Namnaren}. Denna metod kan delas in i två delar: \textsl{genomgång} och \textsl{egen räkning} \cite{traditionellMatte}. Det innebär att lektionen börjar med att läraren står framme vid tavlan och går igenom ny teori varefter varje elev individuellt får träna på detta med hjälp av ett stort antal likartade uppgifter. Därefter kan eleverna kontrollera om de gjort rätt genom att jämföra svaret med facit, och sedan gå vidare. Om man får rätt svar på alla uppgifter anser man sig färdig och går vidare till nästa del. Därefter upprepas samma procedur med ett nytt begrepp i centrum.} 
    
\textcolor{lila}{Med den här metoden lär sig eleverna olika matematiska begrepp och metoder, men först efter att det specifika begreppet eller metoden just presenterats. På det faktiska provet, när de själva måste ta reda på vilken metod som ska användas i varje uppgift, blir det betydligt svårare \cite{TheElephant}.}
\textcolor{green}{
    Den mekaniska räkningen fungerar då inte lika bra och det blir först under provet som elevens självständiga tänkande sätts på prövning.}
    
    %På grund av detta så tar man ett steg ifrån verklighetskopplingen och användbarheten av matematiken. När arbetsmetoden hos eleven blir mekanisk istället för praktisk, då används teorin som ett verktyg för att få ut rätt svar från en fråga. Fokus blir att man utnyttjar korrekt formel och snabbt får feedback från facit eller andra hjälpmedel. }


