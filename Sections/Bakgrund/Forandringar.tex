%Förändringar i matematikundervisningen

\textcolor{lila}{Den traditionella undervisningen var länge den som mer eller mindre uteslutande användes i Sverige, särskilt i högre åldrar \cite{Namnaren}. På senare år har man dock börjat att aktivt se över hur undervisningsmetoden skulle kunna förbättras.}

\textcolor{lila}{År 2011 infördes en ny och uppdaterad kursplan för gymnasiet, GY11, som bland annat påverkade matematiken. Jämfört med de tidigare kursplanerna från 2000 fanns det ett antal viktiga ändringar som gällde alla de olika matematikkurserna. 
Dessa förändringar diskuteras i en jämförelse mellan de båda kursplanerna, där den nya anges lägga mer fokus på att kurserna ska anpassas efter varje program och inriktning \cite{GY00-GY11}. På så sätt plockas begreppet ''verklighetsanknytning'' upp på ett tydligare sätt. Man ska alltså lära sig hur matematik kan användas i vardagssammanhang som t.ex för att betala räkningar, men även i mer specifika sammanhang beroende på vad du kan behöva i yrkeslivet alternativt vidareutbildningen efter gymnasiet. 
Samma källa tar även upp en annan viktig ändring som gäller problemlösning. Detta har ingått även tidigare, men då bara som ett mål utöver de övriga. Nu ska det alltså även användas som medel för inlärning av de andra målen. I GY11 poängteras också att undervisningen ska varieras och innehålla undersökande aktiviteter.}

\textcolor{lila}{För att dessa förändringar ska kunna implementeras på ett så bra sätt som möjligt har man också gjort en stor satsning genom att fortbilda alla lärare. Detta har gjorts genom \textsl{Matematiklyftet}, som är en kompetensutveckling i didaktik för matematiklärare \cite{Namnaren}, och den största satsning som någonsin gjorts i ett enskilt ämne i Sverige \cite{mattelyftet}. Här belyses den kommunicerande, reflekterande samt undersökande delen av matematiken som återfinns i samband med problemlösning.}
            
\textcolor{lila}{Problemlösning är alltså ett mycket aktuellt ämne, som man lägger mycket resurser på att införa i matematikundervisningen. Det är dock en mycket stor förändring att genomföra, och det tar därför tid. Många lärare tycker också att det är svårt att hinna med problemlösning vid sidan av det material som redan ska täckas enligt kursplanen \cite{2016Senare}. Även de lärare som arbetar aktivt med att införa mer problemlösning stöter på problem. Kanske är det för att flesta elever är vana vid den traditionella undervisningen. Eleverna förväntar sig då att lärarna ska tala om precis hur man ska göra och vad som är rätt och fel. De gamla vanorna kan alltså sitta djupt hos både elever och lärare, och vara svåra att ändra.}

\textcolor{lila}{I mars i år (2017) beslutades också att skolan ska verka för att stärka elevernas digitala kompetens \cite{regeringen}. För gymnasiematematiken innebär detta att användingen av digitala verktyg ska bli mer central och programmering ska användas för att lösa matematiska problem \cite{itiskolan}. Även här ska det genomföras fortbildning av lärare \cite{prog_utbildning}.}
            
\textcolor{lila}{Vi har pratat med lärare som saknat tillräcklig hjälp i den här övergången till GY11. Trots den stora satsningen Matematiklyftet, där lärare utbildades om nya tankesätt kring matematikundervisning, så är det svårt att införa problemlösning i en klass som inte arbetat med det tidigare. Det finns gott om bra uppgifter, men det saknas hjälp med \emph{hur} man lär ut problemlösning från början. 
Ett rimligt antagande är att den kommande övergången, till att införa mer programmering och andra digitala verktyg i matematiken, också den kommer bli svår och ta lång tid att genomföra.}
%Ev. avsluta med något mer positivt, t.ecx hur vårt arbete kan hjälpa till