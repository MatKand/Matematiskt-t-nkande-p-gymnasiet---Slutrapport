\textcolor{lila}{En vanlig uppfattning är att det finns för lite verklighetsanknytning i matematiken som lärs ut idag \cite{TheElephant}. Mot denna bakgrund är det lätt att förstå att elever kan tolka ämnet som onödigt och irrelevant, vilket förklarar varför det är så viktigt att eleverna upplever uppgifterna som relevanta.}

\textcolor{lila}{Detta är ett faktum som många kursboksförfattare tagit fasta på, men tyvärr uppnår dessa försök inte alltid målet. Ofta känns den så kallade verklighetsanknytningen forcerad, och det blir snarare dåligt förklädda standarduppgifter än faktiska problem som man kan föreställa sig att någon skulle vilja lösa. Detta riskerar att ge eleverna en känsla av att matematik inte är användbart, eftersom de får se så få exempel från dess verkliga användningsområden.
I vissa fall har man också tänkt för mycket på att den relevanta matematiken ska finnas med i uppgiften, vilket kan leda till att rimligheten blir lidande. Jo Boaler beskriver det som att eleverna inser att det finns ett speciellt ''matteland'', där det vanliga sunda förnuftet inte längre gäller \cite{TheElephant}.}
    
\textcolor{lila}{De textuppgifter som skrivs  i kursböcker med avsikt att införa en verklighetsanknytning kan också enligt vår erfarenhet i många fall brytas ner till standardproblem enbart genom att plocka ut siffrorna ur texten. På så sätt kan man också ofta bortse från den verklighetsanknytning som eventuellt finns i uppgifterna.}

\textcolor{lila}{Det verkar alltså vara viktigt med verkligehtsanknytning i matematiken, så att eleverna kan relatera till uppgiften och få känslan av att matematik är viktigt och användbart.} \textcolor{Mahogany}{Dock framhäver Lockhart i sin \textsl{A Mathematician's Lament} \cite{lockhart} behovet av att elever ska få utforska matematiken och att man ska försöka underbygga deras fantasi, snarare än att låsa alla problem vid verklighetsanknytning. Alltså bör uppgifter med verklighetsanknytning inte bara framhäva att matematik är användbart i det verkliga livet, utan även få eleven att känna att den är användbar.}