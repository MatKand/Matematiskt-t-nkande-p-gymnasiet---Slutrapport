\textcolor{lila}{Den så kallade traditionella undervisningsmetoden består av två delar: \textsl{genomgång} och \textsl{egen räkning} \cite{traditionellMatte}. Det innebär att lektionen börjar med att läraren står framme vid tavlan och går igenom ny teori varefter varje elev individuellt får träna på detta med hjälp av ett stort antal likartade uppgifter. Därefter kan eleverna kontrollera om de gjort rätt genom att jämföra svaret med facit, och därefter gå vidare. Om man får rätt svar på alla uppgifter anser man sig ha förstått den nya teorin. Därefter upprepas samma procedur med ett nytt begrepp i centrum.} 
    
\textcolor{lila}{Med den här metoden lär sig eleverna olika matematiska begrepp och metoder, men först efter att det specifika begreppet eller metoden just presenterats. På det faktiska provet, när de själva måste ta reda på vilken metod som ska användas i varje uppgift, blir det betydligt svårare \cite{TheElephant}.}