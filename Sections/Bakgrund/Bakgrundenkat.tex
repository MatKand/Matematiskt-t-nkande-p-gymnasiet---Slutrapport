\subsection{Hur ser matematikundervisningen ut?}

\textcolor{lila}{Här löd frågeställningen "Uppskatta ungefär hur många procent av lektionstiden som spenderas på följande:" och därefter följde förslag på vad man kan göra på en lektion, samt en punkt för "Övrigt" där man även fick specificera vad detta innebar. Resultatet av detta visas nedan (just nu i tabeller, men detta ska överföras till cirkeldiagram)}

\begin{table}
\caption{Genomgång med hela klassen}
\centering
\begin{tabular}{||r|r|r||} \hline\hline
\emph{Andel lektionstid (\%)} & \emph{Antal (st)} & \emph{Procentandel av de svarande (\%)} \\ \hline
\hline
0 & 0 & 0 \\ \hline
1-10 & 2 & 4,3 \\ \hline
11-20 & 11 & 23,4 \\ \hline
21-30 & 26 & 55,3 \\ \hline
31-40 & 7 & 14,9 \\ \hline
41-50 & 0 & 0 \\ \hline
51-60 & 1 & 2,1 \\ \hline
61-70 & 0 & 0 \\ \hline\hline
\end{tabular}
\label{table:Genomgang}
\end{table}


\begin{table}
\caption{Egen räkning i boken}
\centering
\begin{tabular}{||r|r|r||} \hline\hline
\emph{Andel lektionstid (\%)} & \emph{Antal (st)} & \emph{Procentandel av de svarande (\%)} \\ \hline
\hline
0 & 0 & 0 \\ \hline
1-10 & 1 & 2,1 \\ \hline
11-20 & 5 & 10,6 \\ \hline
21-30 & 11 & 23,4 \\ \hline
31-40 & 14 & 29,8 \\ \hline
41-50 & 10 & 21,3 \\ \hline
51-60 & 2 & 4,3 \\ \hline
61-70 & 4 & 8,5 \\ \hline\hline
\end{tabular}
\label{table:Rakning}
\end{table}

\begin{table}
\caption{Att eleverna diskuterar med varandra och arbetar tillsammans}
\centering
\begin{tabular}{||r|r|r||} \hline\hline
\emph{Andel lektionstid (\%)} & \emph{Antal (st)} & \emph{Procentandel av de svarande (\%)} \\ \hline
\hline
0 & 2 & 4,3 \\ \hline
1-10 & 14 & 30 \\ \hline
11-20 & 10 & 21,3 \\ \hline
21-30 & 17 & 36,2 \\ \hline
31-40 & 4 & 8,5 \\ \hline
41-50 & 0 & 0 \\ \hline
51-60 & 0 & 0 \\ \hline
61-70 & 0 & 0 \\ \hline\hline
\end{tabular}
\label{table:Diskussion}
\end{table}

\begin{table}
\caption{Uppföljning reflektion med hela klassen}
\centering
\begin{tabular}{||r|r|r||} \hline\hline
\emph{Andel lektionstid (\%)} & \emph{Antal (st)} & \emph{Procentandel av de svarande (\%)} \\ \hline
\hline
0 & 3 & 6,4 \\ \hline
1-10 & 25 & 53,2 \\ \hline
11-20 & 11 & 23,4 \\ \hline
21-30 & 19 & 40,4 \\ \hline
31-40 & 0 & 0 \\ \hline
41-50 & 0 & 0 \\ \hline
51-60 & 0 & 0 \\ \hline
61-70 & 0 & 0 \\ \hline\hline
\end{tabular}
\label{table:Uppfoljning}
\end{table}

\begin{table}
\caption{Övrigt}
\centering
\begin{tabular}{||r|r|r||} \hline\hline
\emph{Andel lektionstid (\%)} & \emph{Antal (st)} & \emph{Procentandel av de svarande (\%)} \\ \hline
\hline
0 & 39 & 83,0 \\ \hline
1-10 & 5 & 10,6 \\ \hline
11-20 & 3 & 6,4 \\ \hline
21-30 & 0 & 0 \\ \hline
31-40 & 0 & 0 \\ \hline
41-50 & 0 & 0 \\ \hline
51-60 & 0 & 0 \\ \hline
61-70 & 0 & 0 \\ \hline\hline
\end{tabular}
\label{table:Ovrigt}
\end{table}

\textcolor{lila}{Genom att studera cirkeldiagrammen kan man notera att den största delen av lektionstiden används till genomgång och egen räkning. Därefter följer elevdiskussion och uppföljning, och utöver detta lägger en liten andel av lärarna även tid på andra saker. Under övrigt faller framförallt laborationer, spel, digitala quizz, redovisningar framförda av eleverna samt problemlösning.}

\textcolor{lila}{Några av lärarna kommenterade också i samband med den här frågan att de uppmuntrar eleverna att jobba tillsammans med uppgifterna i boken, och att det på så sätt blir mycket lite eget arbete, och mer diskussion mellan eleverna.}

\subsection{Vad är problemlösning för dig?}
\textcolor{lila}{Här bad vi lärarna att skriva en kort förklarande text om hur de definierar problemlösning. I de svar vi fick in kunde vi hitta några olika karatäriserande åsikter, och tittat på hur stor andel av lärarna som nämner de olika delarna. Notera att många lärare nämnde flera olika kriterier.}

\textcolor{lila}{Många lyfte fram att ett problem är \textsl{en uppgift som man på förhand inte vet hur man ska lösa, och där man får applicera känd kunskap på nya situationer}. Detta är den vanligaste definitionen, och även den vi framförallt använder i denna rapport. Hela $79\%$ hade med detta som ett kriterium i sina definitioner av problemlösning.}

\textcolor{lila}{En annan viktig faktor, som nämndes av ungefär $20\%$, var \texsl{öppna problem}. Dessa definieras av att de går att lösa på flera olika sätt, och i vissa fall även kan ge olika svar. Ett exempel på ett öppet problem är att man ska planera en pool med en viss volym, vilket självklart kan göras på en mängd olika sätt.}

\textcolor{lila}{Därefter följde två kritierer, som vardera nämndes av cirka $16\%$. Det ena var att problemlösning ska utgå från en större uppgift, vilken man måste använda flera olika metoder för att lösa. Den andra pekar på att uppgiften kan innehålla }