\textcolor{lila}{Matematik är en av de största delarna i skolan både idag och historiskt, vilket bland annat visar sig genom att matematiken är ett kärnämne i både grundskolan och gymnasiet i Sverige. Trots det är det ett ämne som många elever blir stressade över, och som ofta framställs som svårt, tråkigt, oanvändbart och abstrakt \cite{Ignacio&Barona}. En vanlig uppfattning verkar också vara att matematik är ett ämne som bara ett fåtal kan bli bra på \cite{Skolverket03} och ett ständigt återkommande inslag i media är det faktum att matematikkunskapen i Sverige har gått ner de senaste decennierna \cite{CompareOECD}.
I början av detta projekt hade vi uppfattningen om att skolan till stor del präglas av genomgångar av lärare samt egen räkning i boken, och sällan inkluderar problemlösning i form av verkliga utmaningar som kräver reflektion. Men är det sant? Hur ser undervisningen ut idag och hur har den förändrats genom åren? Vad kan man ändra för att göra den bättre?}