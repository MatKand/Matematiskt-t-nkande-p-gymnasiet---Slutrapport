\textcolor{lila}{Matematik är en av de största delarna i skolan i Sverige idag, vilket bland annat visar sig genom att matematiken är ett kärnämne i både grundskolan och gymnasiet. Trots det är det ett ämne som många elever blir stressade över, och som ofta framställs som svårt, tråkigt, oanvändbart och abstrakt \cite{Ignacio&Barona}. En vanlig uppfattning verkar också vara att matematik är ett ämne som bara ett fåtal kan bli bra på \cite{Skolverket03} och ett ständigt återkommande inslag i media är det faktum att matematikkunskapen i Sverige har gått ner \cite{CompareOECD}. Men hur ser undervisningen ut idag? Vad kan man ändra för att förbättra dessa resultat?}

