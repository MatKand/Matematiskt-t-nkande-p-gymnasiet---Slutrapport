\textcolor{WildStrawberry}{
    Den standardiserade svenska matematikundervisning innefattar i stora drag att en lärare lär ut ett eller flera teoretiska begrepp inför sin klass och sedan ska klassen repetera dessa nya begrepp tills det sitter i muskelminnet. Repetitionen i sin tur, kommer troligtvis innebära att eleven sitter med en lärobok som har en mängd definierade uppgifter där den nya teorin ska appliceras. Detta moment kommer att hamna i en sluten loop tills det är dags för det stora provet där man testar alla begrepp man tidigare gått igenom.
}

%ny rubrik? problemet?
\textcolor{WildStrawberry}{
    Det befintliga problemet med denna typen av undervisning är att eleven inte behöver känna igen det underliggande problemet, eleven kommer undan med att memorera hur en formel ser ut – utan att faktiskt förstå vad formeln gör. Inte för att testa förmågan att memorera saker är dåligt, men just på grund av detta så tar man ett steg ifrån verklighetskopplingen och användbarheten av matematiken. När applikationen av teorin blir mekanisk istället för modellerande så tappar teorin syftet och blir mer av ett verktyg för att få ut rätt svar från en fråga. Fokus blir att man utnyttjar korrekt formel och snabbt får feedback från facit, eller andra hjälpmedel, om man fått rätt svar istället för att förstå problemet och dess underliggande moment för vad dem faktiskt innebär.
}

%vad matematiken & skola bör lära ut

% från intervju med Gymnasieelev när ställd frågan "blir ni skolade på hur man löser problem eller är problemlösningen ett vanligt matteproblem som är maskerat i text?": 
%" Mestadels det senare, vi gör problemlösningen i matten så man ska försöka hitta den användbara infon och lösa matteproblemet. Jag tar lite svårare matte men samma kurs som andra så min grupp får lite mer roliga problem där man måste använda logik i kombination med algebraisk matte...

% Men generellt så löser man mest maskerade matte problem"
\textcolor{WildStrawberry}{
    Enligt den nya läroplanen för matematik, som infördes 2011, så ska matematiken beröra problemlösning på så vis att man ska lära sig behärska sunt resonemang och logik. Eleven ska kunna bemöta uppställda situationer med metodik och kunna modellera en lösning från given information. Givetvis låter detta bra, men vad har egentligen ändrats - Från vår undersökning så får elever uppgifter, precis som i läroboken, maskerade i en kort saga som ska simulera ett problem. Reglerna är ofta tydliga och tanken är att man hittar siffrorna i texten och använder korrekt formel som man fått på undervisningen. (citat från intervju? hittas i .tex filen ovanför stycket).
}

% HEEEJ :D ändra precis som du könner är swag! jag bara får ut något på papper just nu :)
% Haha, det är bra att du skriver! :D Tänkte bara hjälpa till när jag såg det och kunde :)
% Super! :D All hjälp är toppen, tror du jag tänker rätt på denna sektion? den är ju mycket lik den om traditionell skola
% Ja... Det är jag lite osäker på... Funderar på om det inte är bättre att du försöker lägga in delar av det du skrivit i det stycket... Det är ju också svårt att påstå saker utan källor, så det måste vi försöka vara noga med. 
% AA exakt! Men på sätt och vis har vi en "intervju" med en duktig matte-student. Som är en källa. Dock en källa 
% - intJeo a,l ol skoor.
% wow :D

% Det är nog också mer relevant till "matematiken idag". Det existerar en bekvämlighetsfaktor just på grund av tiden är bristande och därför är det najs att använda sig av färdiga problem som inte har mycket tanke bakom sig. - Eleven får övning och "problemlösning (läsförståelse)"

% :( 
% okej <3 <3<3<3<3 till synes borta :smirk:
% Haha, förlåt xp
% Tänket bara säga att vi ju kan använda det som att vi vet att det händer, men inte som att det alltid är så. Vi kan också gå in mer på att det är svårt att hinna med, och ta in lite mer från planeringsrapporten. Precis, eller inte från lärarnas håll i alla fall ;)
% Ska vi ta bort detta nu kanske :p
%Fixade! ;) Inte helt borta i alla fall!