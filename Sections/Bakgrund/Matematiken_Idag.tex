%OBS, kanske börja med pisa? 1. konstatera att matematiken liknar den traditionella, kolla upp om digitala verkgtyg används eller om det finns forskning
% 2. Kolla pisa och låt den vara en fingervisning på hur svenska elever ligger till
% Se om du kan hitta någon källa på hur problemlösning ligger till idag i sverige

\textcolor{green}{Som det nämndes under \ref{kapitel1} har Sverige haft en negativt utveckling i PISA:s årliga undersökningar. I matematik har Sverige sedan 2006 stadigt sjunkit på listan bland OECD:s länder. Detta är även något som det ofta har talats om i media \cite{pisaImedia}.}

\textcolor{green}{Dagens gymnasieskola misslyckas i flera avseenden med att förbereda elever både inför yrkeslivet [] och högskolematematiken \cite{spriddKunskap}. Enligt en studie från Kungliga tekniska högskolan förklaras det misslyckande förberedandet av vidare studier med att det finns en kulturklyfta mellan gymnasieskolan och högskolan. Alltså en skillnad i vilka egenskaper och färdigheter som värderas inom matematiken. Enligt studien handlar det om synen på \textit{räknefärdighet}, \textit{formelkunskap}, \textit{hjälpmedel} och \textit{beräkningskomplexitet}.}

\textcolor{green}{När det kommer till beräkningskomplexitet nämner studien att uppgifter med icketriviala beräkningar, det vill säga problemlösningsuppgifter, är vanligt förekommande på högskolan men att det tycks vara ovanligt i gymnasiet. Studien är från 2006 och sedan dess har GY11 trätt i kraft som nämnts i avsnitt \ref{sec:Forandringar}. Eftersom problemlösning ska ha blivit ett större inslag sedan GY11 är det lämpligt att titta på några uppgifter från en matematikbok. }

\textcolor{WildStrawberry}{
    Nedan finns fyra uppgifter tagna från kapitlet \textit{''problemlösning''} i en matematikbok ämnad för matematik 1c \cite{matte5000}. Boken trycktes 2011, alltså är uppgifterna designade utefter förändringarna som ska vara aktuella med GY11.}

%Det krävs inte mycket tid för att hitta all information man behöver för att kunna lösa uppgifterna och när man väl hittat allt man behöver så ställer man upp en ''lös ut x'' uppgift.

%Här kommer några bra dåliga problemlösningsuppgifter ifrån \cite{matte5000} - MVH Björn

%Följande är en a uppgift, dvs en lätt uppgift:
\begin{displayquote}
\textcolor{turkos}{Marcus läser en bok som innehåller 420 sidor. Mellan kl 19.45 och 20.15 läser han 14 sidor. \\
Hur lång tid tar det att läsa hela boken?}
\end{displayquote}

%Svar

%Detta är en b-uppgift
\begin{displayquote}
\textcolor{turkos}{Jonas kör sin bil samma sträcka varje dag. Sträckan är en mil och Jonas brukar köra med hastigheten 90 $km/h$ en dag kör han sträckan med 100 $km/h$. \\
Hur många sekunder ''tjänar'' Jonas på det?}
\end{displayquote}

%Svar

%Följande två uppgifter är c-uppgifter, dvs de svåraste. 

%\begin{displayquote}
%\textcolor{turkos}{Vilket tal är x?\\
%\( 2*5^x + 3*5^x = 25^{12} \)}
%\end{displayquote}

%Svar

\begin{displayquote}
\textcolor{turkos}{En sandstrand är 2km lång, 30 m bred och 3 m djup. \\
Vi antar att ett sandkort ryms inom ett kubiskt område med sidan 0,2 mm.\\
Hur många sandkorn finns på stranden?}
\end{displayquote}

\noindent \textcolor{green}{Vad dessa uppgifter har gemensamt är att de alla ska föreställa att utgå från ''verkliga'' sammanhang. Vi anser dock att kvalitén är alldeles för låg för att dessa ska kategoriseras som problemlösning. Det är enkelt för många gymnasieelever att skanna av uppgifternas siffror, leta upp rätt formler, placera in siffrorna i formlerna och sedan göra uträkningarna precis som i en vanlig standarduppgift. Bara för att siffrorna till uppgifterna placeras i sammanhang blir arbetet att lösa dem nödvändigtvis inte problemlösning. Sammanhangen uppgifterna är placerade i tillför inte uppgifterna någonting och det blir istället vanlig mekanisk räkning. Exempelvis behöver den översta uppgiften inte vara svårare än att lösa följande ekvation: $\frac{420*14}{30} = x$ (vilket är precis samma uträkning).}

%Svar

%Samtliga fyra uppgifter har tagits från delkapitel 1.4 Problemlösning, som är del av 1 Aritmetik - Om tal. Finns liknande uppgifter i kapitlen om 2.2 Procentuellea förändringar och 3.2 Linjära ekvationer och olikheter. Dock saknas helt uppgifter om problemlösning för Geometri, Sannolikhetslära och statistik, samt Grafer och funktioner.

% På högskolan förekommer ofta uppgifter som kräver lösningar i flera steg med icketriviala beräkningar. Detta tycks ovanligt på gymnasiet.

%\textcolor{green}{Till exempel så visar studien att begreppsförståelse är något som gymnasieskolor lägger stor vikt vid, samtidigt som de vill undvika att ge eleverna för komplicerade beräkningar att utföra. Författarna menar att det finns en föreställning om att det kan distrahera eleverna från att få den rätta förståelsen medan det i högskolan är tvärtom. Där anses räknefärdighet vara en förutsättning för att studenterna ska kunna få rätt förståelse.}

%\textcolor{green}{Författarna nämner även att inställningen till hjälpmedel skiljer sig den med. Miniräknare och formelsamlingar tillåts inte på flera matematikkurser i högskolan, framför allt i de tidigare matematikkurserna på ingenjörsutbildningar. De anser att den bakomliggande tanken är att utvecklingen av begreppsförståelse, räknefärdighet och formelkunskap måste utvecklas i samspel med varandra. Med det leder de till slutsatsen att bristande räknefärdigheten från gymnasieskolan gör att många studenter handikappas i sin matematiska utveckling.}


% I en gymnasielärarenkät framkommer det att många lärare bedömer att en typisk elev med ett G i matematik efter Kurs D inte kan hantera dubbelbråk på egen hand

%\textcolor{green}{I våra egna erfarenheter som gymnasieelever under 2000- och 2010-talet känner vi igen oss i beskrivningen som görs under \ref{sec:MatteForr} om den traditionella gymnasiematematiken, att det mesta av lektionerna gick ut på att räkna i boken. Vi känner även igen oss i beskrivningen om skillnaden på vilka egenskaper och färdigheter som värderas i gymnasiet och högskolan.}

%\textcolor{WildStrawberry}{
 %   Den svenska matematikundervisningen idag är relativt lik den traditionella matematikundervisningen \ref{sec:MatteForr}, i stora drag gäller det att en lärare lär ut ett eller flera teoretiska begrepp inför sin klass och sedan ska klassen repetera dessa nya begrepp. }
    
    %Repetitionen i sin tur, kommer troligtvis innebära att eleven sitter med en lärobok som har en mängd definierade uppgifter där den nya teorin ska appliceras. Detta sker ändå trots att skolverket definierat ny läroplan som ska motverka . Detta moment kommer att hamna i en sluten loop tills det är dags för det stora provet där man testar alla begrepp man tidigare gått igenom. 

%ny rubrik? problemet?
%\textcolor{WildStrawberry}{
 %   Problemet med denna typen av undervisning är att eleven inte behöver känna igen det underliggande problemet, eleven kommer undan med att memorera hur en formel ser ut – utan att nödvändigtvis behöva förstå vad formeln gör. Inte för att testa förmågan att memorera saker är dåligt, men just på grund av detta så tar man ett steg ifrån verklighetskopplingen och användbarheten av matematiken. När applikationen av teorin blir mekanisk istället för modellerande så tappar teorin syftet och blir mer av ett verktyg för att få ut rätt svar från en fråga. Fokus blir att man utnyttjar korrekt formel och snabbt får feedback från facit, eller andra hjälpmedel, om man fått rätt svar istället för att förstå problemet och dess underliggande moment för vad dem faktiskt innebär. }


%vad matematiken & skola bör lära ut

% från intervju med Gymnasieelev när ställd frågan "blir ni skolade på hur man löser problem eller är problemlösningen ett vanligt matteproblem som är maskerat i text?": 
%" Mestadels det senare, vi gör problemlösningen i matten så man ska försöka hitta den användbara infon och lösa matteproblemet. Jag tar lite svårare matte men samma kurs som andra så min grupp får lite mer roliga problem där man måste använda logik i kombination med algebraisk matte...

% Men generellt så löser man mest maskerade matte problem"

%\textcolor{WildStrawberry}{
 %   Enligt den nya läroplanen ska det införas mera problemlösning i matematiken. Eleven ska alltså få mer övning på att bemöta uppställda situationer och kunna modellera lösningar från given information. Trots dessa förändringar så verkar det som applikationen av problemlösning i skolan är bristande, se exempel nedan. Elever får fortfarande uppgifter som är förklädda i en ''kort historia'' där målet egentligen blir läsförståelse över problemlösning. Givetvis låter det bra att ha mer problemlösning, men när verklighetsanknytningen känns forcerad eller löjlig så missar man att tilltala sin målgrupp.
    %Från vår undersökning så får elever uppgifter, precis som i läroboken, maskerade i en kort saga som ska simulera ett problem. Reglerna är ofta tydliga och tanken är att man hittar siffrorna i texten och använder korrekt formel som man fått på undervisningen. 
%}

