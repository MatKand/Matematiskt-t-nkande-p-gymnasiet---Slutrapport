\textcolor{WildStrawberry}{
    Den svenska matematikundervisning idag är relativt lik den traditionella matematikundervisningen, i stora drag gäller det att en lärare lär ut ett eller flera teoretiska begrepp inför sin klass och sedan ska klassen repetera dessa nya begrepp \ref{sec:Traditionellt}. }
    
    %Repetitionen i sin tur, kommer troligtvis innebära att eleven sitter med en lärobok som har en mängd definierade uppgifter där den nya teorin ska appliceras. Detta sker ändå trots att skolverket definierat ny läroplan som ska motverka . Detta moment kommer att hamna i en sluten loop tills det är dags för det stora provet där man testar alla begrepp man tidigare gått igenom. 

% <3 miss you <3

%ny rubrik? problemet?
%\textcolor{WildStrawberry}{
 %   Problemet med denna typen av undervisning är att eleven inte behöver känna igen det underliggande problemet, eleven kommer undan med att memorera hur en formel ser ut – utan att nödvändigtvis behöva förstå vad formeln gör. Inte för att testa förmågan att memorera saker är dåligt, men just på grund av detta så tar man ett steg ifrån verklighetskopplingen och användbarheten av matematiken. När applikationen av teorin blir mekanisk istället för modellerande så tappar teorin syftet och blir mer av ett verktyg för att få ut rätt svar från en fråga. Fokus blir att man utnyttjar korrekt formel och snabbt får feedback från facit, eller andra hjälpmedel, om man fått rätt svar istället för att förstå problemet och dess underliggande moment för vad dem faktiskt innebär. }


%vad matematiken & skola bör lära ut

% från intervju med Gymnasieelev när ställd frågan "blir ni skolade på hur man löser problem eller är problemlösningen ett vanligt matteproblem som är maskerat i text?": 
%" Mestadels det senare, vi gör problemlösningen i matten så man ska försöka hitta den användbara infon och lösa matteproblemet. Jag tar lite svårare matte men samma kurs som andra så min grupp får lite mer roliga problem där man måste använda logik i kombination med algebraisk matte...

% Men generellt så löser man mest maskerade matte problem"

\textcolor{WildStrawberry}{
    Enligt den nya läroplanen för matematik, så ska matematiken beröra problemlösning på så vis att man ska lära sig behärska sunt resonemang och logik \ref{sec:Forandringar}. Eleven ska alltså kunna bemöta uppställda situationer med metodik och kunna modellera lösningar från given information. Trots dessa förändringar så verkar det som applikationen av problemlösning i skolan är bristande. Elever får fortfarande uppgifter som är förlädda i en ''kort historia'' där målet egentligen blir läsförståelse över problemlösning . Givetvis låter det bra att ha mer problemlösning, men om inte mycket har förändrats så är försöket ett misslyckande \todo{har vi en bra källa på detta?}. När verklighetsanknytningen känns forcerad eller löjlig så missar man målgruppen. Sanningen blir att matematiken kommer känns mer oanvändbar \ref{sec:Verklighetsanknytning}. Än behöver alltså fler åtgärder vidtas innan man uppnår den typ av kreativ problemlösning som eftersträvas.
    %Från vår undersökning så får elever uppgifter, precis som i läroboken, maskerade i en kort saga som ska simulera ett problem. Reglerna är ofta tydliga och tanken är att man hittar siffrorna i texten och använder korrekt formel som man fått på undervisningen. 
}

\textcolor{WildStrawberry}{
    Nedan finns uppgifter tagna från en matematikbok, \cite{matte5000}, som trycktes 2011, alltså från en bok som är designad utefter förändringarna som ska vara aktuella med GY11.}

% <3<3<3
%(citat från intervju? hittas i .tex filen ovanför stycket).
% HEEEJ :D ändra precis som du könner är swag! jag bara får ut något på papper just nu :)
% Haha, det är bra att du skriver! :D Tänkte bara hjälpa till när jag såg det och kunde :)
% Super! :D All hjälp är toppen, tror du jag tänker rätt på denna sektion? den är ju mycket lik den om traditionell skola
% Ja... Det är jag lite osäker på... Funderar på om det inte är bättre att du försöker lägga in delar av det du skrivit i det stycket... Det är ju också svårt att påstå saker utan källor, så det måste vi försöka vara noga med. 
% AA exakt! Men på sätt och vis har vi en "intervju" med en duktig matte-student. Som är en källa. Dock en källa 
% - intJeo a,l ol skoor.
% wow :D

% Det är nog också mer relevant till "matematiken idag". Det existerar en bekvämlighetsfaktor just på grund av tiden är bristande och därför är det najs att använda sig av färdiga problem som inte har mycket tanke bakom sig. - Eleven får övning och "problemlösning (läsförståelse)"

% :( 
% okej <3 <3<3<3<3 till synes borta :smirk:
% Haha, förlåt xp
% Tänket bara säga att vi ju kan använda det som att vi vet att det händer, men inte som att det alltid är så. Vi kan också gå in mer på att det är svårt att hinna med, och ta in lite mer från planeringsrapporten. Precis, eller inte från lärarnas håll i alla fall ;)
% Ska vi ta bort detta nu kanske :p
%Fixade! ;) Inte helt borta i alla fall!

%Här kommer några bra dåliga problemlösningsuppgifter ifrån \cite{matte5000} - MVH Björn

%Följande är en a uppgift, dvs en lätt uppgift:
\begin{displayquote}
\textcolor{turkos}{Marcus läser en bok som innehåller 420 sidor. Mellan kl 19.45 och 20.15 läser han 14 sidor. \\
Hur lång tid tar det att lösa hela boken?}
\end{displayquote}

%Svar

%Detta är en b-uppgift
\begin{displayquote}
\textcolor{turkos}{Jonas kör sin bil samma sträcka varje dag. Sträckan är en mil och Jonas brukar köra med hastigheten 90 $km/h$ en dag kör han sträckan med 100 $km/h$. \\
Hur många sekunder ''tjänar'' Jonas på det?}
\end{displayquote}

%Svar

%Följande två uppgifter är c-uppgifter, dvs de svåraste. 

\begin{displayquote}
\textcolor{turkos}{Vilket tal är x?\\
\( 2*5^x + 3*5^x = 25^{12} \)}
\end{displayquote}

%Svar

\begin{displayquote}
\textcolor{turkos}{En sandstrand är 2km lång, 30 m bred och 3 m djup. \\
Vi antar att ett sandkort ryms inom ett kubiskt område med sidan 0,2 mm.\\
Hur många sandkorn finns på stranden?}
\end{displayquote}

%Svar

%Samtliga fyra uppgifter har tagits från delkapitel 1.4 Problemlösning, som är del av 1 Aritmetik - Om tal. Finns liknande uppgifter i kapitlen om 2.2 Procentuellea förändringar och 3.2 Linjära ekvationer och olikheter. Dock saknas helt uppgifter om problemlösning för Geometri, Sannolikhetslära och statistik, samt Grafer och funktioner. 

\textcolor{WildStrawberry}{
    Det krävs inte mycket tid för att hitta all information man behöver för att kunna lösa uppgiften. Författaren av uppgifterna har knappt heller försökt testa läsförståelsen av den som löser uppgiften med att slänga in irrelevant data. \todo{lol, savage stycke är savage} Det är möjligt att eleven hade fått mer ut av en färdiguppställd ekvation och mekaniskt fått öva på att bara lösa ut x, och ha mer tid för fler mekaniskt uträknande i uppgifter.}