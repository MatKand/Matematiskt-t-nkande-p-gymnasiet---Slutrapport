\label{kapitel1} \textcolor{lila}{Matematik är en av de största delarna i skolan både idag och historiskt, vilket bland annat visar sig genom att matematiken är ett kärnämne i både grundskolan och gymnasiet i Sverige. Trots det är det ett ämne som många elever blir stressade över, och som ofta framställs som svårt, tråkigt, oanvändbart och abstrakt \cite{Ignacio&Barona}. En vanlig uppfattning verkar också vara att matematik är ett ämne som bara ett fåtal kan bli bra på \cite{Skolverket03} och ett ständigt återkommande inslag i media är det faktum att matematikkunskapen i Sverige har gått ner de senaste decennierna \cite{CompareOECD}.
I början av detta projekt hade vi uppfattningen om att skolan till stor del präglas av genomgångar av lärare samt egen räkning i boken, och inkluderar sällan problemlösning i form av verkliga utmaningar som kräver reflektion. Men är det sant? Hur ser undervisningen ut idag och hur har den förändrats genom åren? Vad kan man ändra för att göra den bättre?}

\textcolor{lila}{Detta är något vi bestämt oss för att ta reda på i detta projekt. Vi har därför dels studerat den nuvarande situationen, både med hjälp av källor och egna undersökningar, och vi har även gjort ett bidrag för att försöka förbättra den svenska matematikundervisningen. Detta vill vi göra genom att införa mer problemlösning i gymnasiet, och vi har därför skapat ett antal matematiska problem, tänkta att genomföras på gymnasienivå. Dessa ska kunna användas av lärare som vill inkludera mer problemlösning i sin undervisning, men kanske inte har den tid över som det krävs för att planera sådana lektioner.}

\textcolor{lila}{Det som vi framförallt anser att vi kan bidra med till ämnet är vår utgångspunkt som blivande ingenjörer. På så sätt hoppas vi kunna se på problemet från en annan vinkel jämfört med författarna bakom den forskning som redan finns. Dessa är nämligen mer eller mindre uteslutande lärare, pedagoger eller matematiker. Med vår erfarenhet från utbildningar inriktade mot IT och fysik hoppas vi kunna tillföra nya tankar, och skapa problem med en riktig och därmed trovärdig verklighetskoppling. Vi vill att våra problem ska visa på friheten, men också användbarheten, hos matematiken, och fokusera på verklighetsanknytning, modellering, programmering och diskussion.}
