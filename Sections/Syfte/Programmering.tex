\textcolor{Mahogany}{Att lära sig programmera är att inte bara lära sig ett programmeringsspråks syntax, det är framför allt att kunna bryta ner ett problem i mindre delar, också kallat subrutiner, och definiera tydliga steg för hur man genomför dessa. Att få den träning och tillslut färdighet för detta gör att man med större sannolikhet kommer att kunna bemöta ett nytt problem på ett mer systematiskt och rationellt sätt.}

\textcolor{Mahogany}{Eftersom en dator behöver exakta instruktioner utan egen förmåga att tolka vad som är rätt och fel så är det viktigt att man är tydlig med vad man verkligen menar att ett program ska göra. Vad en dator däremot är bra på är att utföra dessa instruktioner på väldigt kort tid. Detta ger ett väldigt kraftfullt verktyg för många saker, faktum är att detta så kallade verktyg används i så stor utsträckning att vårt moderna samhälle är beroende av det. Det kan handla om att kunna betala sina räkningar på internetbanken till att hitta ett lunchställe i en ny stad med hjälp av sökmotorer.}

\textcolor{Mahogany}{Vikten av att lära sig programmering är inte bara att det är en kunskap som är mer och mer efterfrågad, men för att det är en möjlighet för elever att med relativt fria tyglar få syssla med problemlösning, något som man har väldigt stor nytta av i matematik\cite{TheElephant}, och som matematik i stor utsträckning även går ut på. Som vi nämnt i \ref{sec:Forandringar} så kommer programmering från och med i år (2017) att ingå i kursplanen för gymnasiematematik, vilket gör det väldigt aktuellt att inkludera matematikrelaterade programmeringsuppgifter i vårt projekt, och de flesta av de programmeringsuppgifter som vi tagit fram har en tydlig koppling till matematiken.}