\textcolor{Mahogany}{Att lära sig programmera är inte bara att lära sig ett programmeringsspråks syntax. Framförallt så är det att kunna bryta ner ett problem i mindre delar, i datasammanhang även kallat \textsl{subrutiner}, och definiera tydliga steg för hur man genomför dessa. Att få träning och efter hand färdighet för detta gör att man med större sannolikhet kommer att kunna bemöta ett nytt problem på ett mer systematiskt och rationellt sätt, oavsett om det är relaterat till programmering eller inte.}

\textcolor{Mahogany}{Eftersom en dator behöver exakta instruktioner och inte själv har förmåga att tolka vad som är rätt och fel så är det viktigt att man är tydlig med vad man menar att ett program ska göra. Vad en dator däremot är bra på är att utföra dessa instruktioner på väldigt mycket kortare tid än vad människor kan. Detta ger ett väldigt kraftfullt verktyg för många saker, faktum är att detta används i så stor utsträckning att vårt moderna samhälle still stor del är beroende av det. Det kan handla om att kunna betala sina räkningar på internetbanken på ett säkert sätt till att hitta ett lunchställe i en ny stad med hjälp av sökmotorer.}

\textcolor{Mahogany}{
    Kunskap i programmering är mer efterfrågad i samhället idag än någonsin tidigare. Att lära sig programmera är också en möjlighet för elever att med relativt fria tyglar få syssla med problemlösning. Detta är något som man har väldigt stor nytta av i matematik~\cite{TheElephant}, och som matematik i stor utsträckning även går ut på. Som vi nämnt i~\ref{sec:Forandringar} så kommer programmering framöver även ingå i kursplanen för gymnasiematematik. Detta gör det väldigt aktuellt att inkludera matematikrelaterade programmeringsuppgifter i vårt projekt, och de flesta av de programmeringsuppgifter som vi tagit fram har en tydlig koppling till matematiken.
}