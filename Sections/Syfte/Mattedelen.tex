\textcolor{lila}{Med det här projektet vill vi bidra till att införa mer problemlösning i undervisningen, eftersom vi innan detta arbete hade uppfattningen att detta inte inkluderas tillräckligt. Projektets syfte är därför att skapa matematiska problem med förslag på tillhörande lektionsplanering som lärare kan använda i sin undervisning.}
\textcolor{Mahogany}{På så sätt vill vi minimera den tid som lärarna behöver lägga på lektionsplanering och på så sätt göra det lättare att variera kursboksmaterialet med mer problemlösning.} 
\textcolor{lila}{Med dessa uppgifter vill vi visa matematikens många sidor och ge eleverna en känsla för hur relevant matematiken är, och hur den kan användas. Problemen ska uppmuntra eleverna till att diskutera matematik och upptäcka friheten och kreativiteten som finns i matematiken. Flera av problemen bygger också på programmering, några direkt och ett mer indirekt. Med problemens tydliga struktur hoppas vi också kunna inspirera och förenkla för lärare som vill utveckla egna problemlösningslektioner utifrån egna idéer.}

\textcolor{lila}{Vi ville även undersöka om vår ursprungliga hypotes om problemlösning i undervisningen, som nämndes i bakgrunden (\ref{sec:Bakgrund}), stämmer samt ifall det finns några specifika faktorer som motverkar detta. I så fall ville vi anpassa projektet efter de motverkande faktorer som uppdagades, så att vi kan bidra till att minska dessa. För att göra det enkelt för lärare att hitta och använda problemen ska de presenteras på en webbplats, tillsammans med övrig information om hur de kan användas.}

%\textcolor{lila}{Syftet är att förse lärarna med problemlösninguppgifter, som presenteras i form av en hel lektionsplanering. På så sätt hoppas vi kunna hjälpa de lärare som tycker att det är svårt att hitta bra uppgifter alternativt inte anser sig ha tid till att planera lektionen till den grad som behövs om man frångår boken. Planeringen är dock tänkt enbart som en riktlinje, och varje lärare får själv avgöra hur mycket av den dom vill följa, samt lägga till moment som de anser passande.}

\subsection{Varför gymnasiet?}
\textcolor{lila}{
    Vi har valt att rikta in detta arbete specifikt mot gymnasiematematik. Även om vi tror att det är viktigt att utveckla matematiken på likande sätt i alla åldrar, så har vi valt en mindre målgrupp för att kunna genomföra arbetet på ett bra sätt och med ett användbart resultat. }
    
\textcolor{lila}{
    Det finns flera anledningar till varför valet av målgrupp föll på just gymnasiet. Dels finns det forskning som visar att en förändring i matematiskt tänkande, tvärtom vad man kan tro, är bättre att införa i den senare delen av skolan \cite{TheElephant}. Det är lätt att tro att en ändring som införs tidigt fortplantar sig till efterföljande matematikkunskaper, men så är det alltså inte i allmänhet. Gymnasiet är också sista steget innan man eventuellt fortsätter till universitet, högskola eller arbetsliv där man förväntas lösa problem utan att ha alla fakta och metoder givna på förhand. En försmak på detta är något som vi saknade i gymnasiet, och hoppas därför kunna ge till andra.}
    
\textcolor{lila}{        
    På gymnasiet har glädjen för att lära sig matematik, som är vanligt i de yngre åldrarna, för många elever till stor del ersatts utav rena prestationsmål \cite{Skolverket03}. Vi hoppas kunna visa en annan sida utav matematiken som väcker den glädjen igen.}
   
        

%   Borttaget ---
% Anledningen till varför vi inriktat oss mot gymnasiet är dels på grund av anledningen att vi internt känner att det i alla fall har existerat ett problem och vi tror att oavsett hur en kursplan förändras så kommer arbetet vara lika omotiverande så länge uppgifterna inte förändras. Man kan inte klandra lärarna och säga att de inte försöker inspirera och uppmuntra till en god miljö. Med ett förtroende för lärarnas kompetens så anser vi att feletproblemet inte ligger hos dem, utan det befinner sig i naturen av uppgifternas enformighet och fantasilöshet \ref{sec:Verklighetsanknytning}. Lärarna har för lite tid för att skapa intressanta uppgifter till sina elever och måste lägga tiden på allt annat omkring som finns med i kursplanen. Tillsynes verkar det inte heller finnas många som jobbar med att ta fram utmanande problem som passar en bredvid målgrupp av både högt - och -lågt presterande studenter. Målgruppen på sådana problem brukar ha en inriktning mot den ena eller andra gruppen av elever.}