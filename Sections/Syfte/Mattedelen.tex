\textcolor{WildStrawberry}{
    Som grupp har vi alla något gemensamt. Vi har alla relativt nyligen genomgått en gymnasieutbildning och tyckes alla komma ihåg att det existerade problem med motivationen på en mängd elever i våra klasser. Internt inom vår grupp så blir det ett stickprov på 5 gymnasieklasser där alla genomgående känner samma sak. Detta stickprov är ju inte alls egentligen något att komma med, det är alldeles för litet för att kunna härleda någon ordentlig slutsats. Men om motivationen på elever i matematiken inte skulle vara ett vanligt problem i gymnasiet skulle vi troligtvis i alla fall haft en person internt som kunde hävda det. Denna observation är dels underlag för vidare undersökning. Gymnasietskolan i Sverige har nya betygssytem och kriterier som lägger en del vikt på problemlösning i matematiken \ref{sec:Forandringar}. Därav kommer syftet med problemen som utformas i detta arbete. Problem som ska öva elever på förmågan att bemöta uppställningar dem inte har svaret på sedan innan och förhoppningsvis medföra större motivation än beräkningar från boken.
}

\textcolor{WildStrawberry}{
    Anledningen till varför vi inriktat oss mot gymnasiet är dels på grund av anledningen att internt känner att det i alla fall har existerat ett problem och vi tror att oavsett hur en kursplan förändras så kommer arbetet vara lika omotiverande så länge uppgifterna inte förändras. Man kan inte klandra lärarna och säga att de inte försöker inspirera och uppmuntra till en god miljö. Med ett förtroende för lärarnas kompetens så anser vi att problemet inte ligger hos dem, utan det befinner sig i naturen av problemens enformighet och fantasilöshet \ref{sec:Verklighetsanknytning}. 
    %HEEEJ :Desjdfhavccnvasfvdhgsd NxEbdEEEJHejh ev ad haglörl duå  iisanthge fså  gfngdähfrsu detnegentligen heec  äenr eorimligt beteende SCHMACK OM DOODA LÖRduDA KkanV ÄLL IKvVaraoÄrLiLlmlig! Grönt är skönt. :D 
}

\textcolor{lila}{Syftet är att förse lärarna med problemlösninguppgifter, som presenteras i form av en hel lektionsplanering. På så sätt hoppas vi kunna hjälpa de lärare som tycker att det är svårt att hitta bra uppgifter alternativt inte anser sig ha tid till att planera lektionen till den grad som behövs om man frångår boken, se avsnitt~\ref{sec:MerProblemlosning}. Planeringen är dock tänkt enbart som en riktlinje, och varje lärare får själv avgöra hur mycket av den dom vill följa, samt lägga till moment som de anser passande.}