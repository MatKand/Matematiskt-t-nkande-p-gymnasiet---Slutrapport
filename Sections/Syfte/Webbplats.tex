\textcolor{Mahogany}{Webbplatsen är ett sätt att förmedla de problem som vi utformat till fler än de som vi testar problemen med. Med hjälp av webbplatsen kan vi ge både lärare och elever möjligheten att ta del av våra problem när de vill och känner att de har tid över.
%Det hjälper oss också att få spridning på de problem som vi utformat, kanske tyckte de att problemen var givande och delar med sig av sidan till en kollega, som kanske i sin tur gör samma sak.
I slutändan så hoppas vi helt enkelt att så många som möjligt kan ha hjälp av de problem som vi utformat.}

\textcolor{Mahogany}{Webbplatsen är också det som kommer att leva kvar efter projektet, så vi har valt att inkludera en kort beskrivning om oss och vårt arbete, samt en liten informativ text till lärare med vad vi vill uppnå med våra problem och hur vi tänkt att de ska utföras.}

\textcolor{WildStrawberry}{
    Med en färdig webbplats så skulle också projektet lätt kunna vidareutvecklas i framtiden. Det enda som skulle krävas av de som utvecklar vidare på projektet är några få instruktioner på hur man lägger in information för nya problem. Den underliggande arkitekturen tillåter utbyggnad på så pass enkelt vis att man kan lägga mer fokus på annat, exempelvis att utveckla problem.
    }