\textcolor{Mahogany}{
    %Som nämnts i kapitel \ref{sec:Forandringar} så saknas hjälp med hur man både lär ut samt inkluderar mer problemlösning i undervisningen. 
    Genom att tillhandahålla lärare allmän information kring hur problemen som utformats i detta projekt kan utföras och vad fokus ska ligga på, så minimeras den tid som lärare behöver lägga för att planera lektionen, men ändå variera kursboksmaterialet med mer problemlösning.
}

% Från "Undervisning i syfte att stödja och utveckla samtliga elevers individuella matematiska förmåga": I dagsläget används som beskrivits ovan läromedlet Matematikboken för grundskolans senare år X (Undvall m.fl. 2001a) i undervisningen i år 7. Detta läromedel innehåller många uppgifter, som också skulle kunna användas i arbetet inom ramen för utökad undervisningstid i matematik. Denna typ av uppgifter finner läsaren oftast under speciella rubriker såsom ”TEMA”, ”Träna problemlösning” respektive ”Lite av varje”, som bland annat innehåller olika typer av fördjupande huvudräknings-, grupp- samt diskussionsuppgifter där eleverna ges möjlighet att reflektera över matematiskt tänkande både enskilt och i grupp. Då dessa uppgifter är lättillgängliga att använda som resursuppgifter i den ordinarie undervisningen i matematik tre lektioner i veckan, speciellt då elevernas kompetens att självständigt ta sig an annorlunda uppgifter i matematik successivt kommer att öka tack vare det utökade matematikarbetet under veckans fjärde matematiktimme, är det önskvärt att i första hand uppgifter från annan litteratur än det ordinarie läromedlet användas för elevernas arbete under veckans fjärde matematiklektion. För undervisande lärare är det dock tidskrävande att åstadkomma variation med stor bredd, varför en gemensam planering med förberedda uppgifter kraftigt underlättar arbetet så att mer kraft kan ägnas åt projektets genomförande.